\documentclass[10pt, letterpaper]{article}

% Inhaltsverzeichnis für Pakettypen (nur für Übersicht im Header, wird nicht im Dokument angezeigt)
% 1. Seitenlayout und Ränder
% 2. Sprache und Zeichensatz
% 3. Mathematik und Theorem-Umgebungen
% 4. Eigene Makros
% 5. Diagramme und Grafiken
% 6. Tabellen und Aufzählungen
% 7. Inhaltsverzeichnis
% 8. Abschnittsüberschriften
% 9. Abstrakt-Umgebung
% 10. Todos/Notizen
% 11. Rahmen/Box-Umgebungen
% 12. Python-Integration
% 13. Literaturverwaltung
% 14. Hyperlinks
% 15. Absatzeinstellungen
% 16. Umgebungen
% 17. Titel und Autor

% --- 1. Seitenlayout und Ränder ---
\usepackage[margin=3cm]{geometry}

% --- 2. Sprache und Zeichensatz ---
\usepackage[english]{babel}
\usepackage[T1]{fontenc}
\usepackage[utf8]{inputenc}

% --- 3. Mathematik und Theorem-Umgebungen ---
\usepackage{amsmath, amssymb, amsthm}
\usepackage{mathrsfs}
\DeclareMathOperator{\WF}{WF}

% --- 4. Eigene Makros ---
\usepackage{xcolor}
\newcommand{\SKP}{\langle\cdot,\cdot\rangle}
\newcommand{\R}{\mathbb{R}}
\newcommand{\N}{\mathbb{N}}
\newcommand{\Q}{\mathbb{Q}}
\newcommand{\Z}{\mathbb{Z}}
\newcommand{\C}{\mathbb{C}}
\newcommand{\entwurf}[1]{\textcolor{red}{#1}}

% --- 5. Diagramme und Grafiken ---
\usepackage{graphicx}
\usepackage{tikz}
\usetikzlibrary{decorations.pathreplacing, arrows.meta, positioning}
\usepackage{tikz-cd}

% --- 6. Tabellen und Aufzählungen ---
\usepackage{enumitem}
\setlist[itemize]{left=0.5cm}

\newenvironment{romanenum}[1][]
  {%
    \ifx&#1&
    \else
      \textbf{#1}\quad
    \fi
    \begin{enumerate}[label=\roman*)]
  }
  {%
    \end{enumerate}%
  }

% --- 7. Inhaltsverzeichnis ---
\usepackage{tocloft}
\renewcommand{\cftsecfont}{\footnotesize}
\renewcommand{\cftsubsecfont}{\footnotesize}
\renewcommand{\cftsubsubsecfont}{\footnotesize}
\renewcommand{\cftsecpagefont}{\footnotesize}
\renewcommand{\cftsubsecpagefont}{\footnotesize}
\renewcommand{\cftsubsubsecpagefont}{\footnotesize}
\usepackage{etoc}

% --- 8. Abschnittsüberschriften ---
\usepackage{titlesec}
\titleformat{\section}{\normalfont\large\bfseries}{\thesection}{1em}{}
\titleformat{\subsection}{\normalfont\normalsize\bfseries}{\thesubsection}{0.5em}{}
\titleformat{\subsubsection}{\normalfont\normalsize\bfseries}{\thesubsubsection}{0.5em}{}
\setcounter{secnumdepth}{4}

% --- 9. Abstrakt-Umgebung ---
\usepackage{changepage}
\renewenvironment{abstract}
  {
    \begin{adjustwidth}{1.5cm}{1.5cm}
    \small
    \textsc{Abstract. –}%
  }
  {
    \end{adjustwidth}
  }

% --- 10. Todos/Notizen ---
\usepackage{todonotes}

% --- 11. Rahmen/Box-Umgebungen ---
\usepackage{mdframed}
\usepackage{tcolorbox}
\colorlet{shadecolor}{gray!25}

\newenvironment{customTheorem}
  {\vspace{10pt}%
   \begin{mdframed}[
     backgroundcolor=gray!20,
     linewidth=0pt,
     innertopmargin=10pt,
     innerbottommargin=10pt,
     skipabove=\dimexpr\topsep+\ht\strutbox\relax,
     skipbelow=\topsep,
   ]}
  {\end{mdframed}
   \vspace{10pt}%
  }

% --- 12. Python-Integration ---
% (Deaktiviert in dieser Version, aktiviere bei Bedarf)
% \usepackage{pythontex}
% \usepackage[makestderr]{pythontex}

% --- 13. Literaturverwaltung ---
\usepackage{csquotes}
\usepackage[backend=biber, style=alphabetic, citestyle=alphabetic]{biblatex}
\addbibresource{bibliography.bib}

% --- 14. Hyperlinks ---
\usepackage{hyperref}
\hypersetup{
  colorlinks   = true,
  urlcolor     = blue,
  linkcolor    = blue,
  citecolor    = blue,
  frenchlinks  = true
}

% --- 15. Absatzeinstellungen ---
\usepackage[parfill]{parskip}
\sloppy

% --- 16. Umgebungen ---
\usepackage{thmtools}

\newcommand{\CustomHeading}[3]{%
  \par\medskip\noindent%
  \textbf{#1 #2} \textnormal{(#3)}.\enskip%
}

\newenvironment{DEF}[2]{\CustomHeading{Definition}{#1}{#2}}{}
\newenvironment{PROP}[2]{\CustomHeading{Proposition}{#1}{#2}}{}
\newenvironment{THEO}[2]{\CustomHeading{Theorem}{#1}{#2}}{}
\newenvironment{LEM}[2]{\CustomHeading{Lemma}{#1}{#2}}{}
\newenvironment{KORO}[2]{\CustomHeading{Corollar}{#1}{#2}}{}
\newenvironment{REM}[2]{\CustomHeading{Remark}{#1}{#2}}{}
\newenvironment{EXA}[2]{\CustomHeading{Example}{#1}{#2}}{}
\newenvironment{STUD}[2]{\CustomHeading{Study}{#1}{#2}}{}
\newenvironment{CONC}[2]{\CustomHeading{Concept}{#1}{#2}}{}
\newenvironment{OTH}[2]{\CustomHeading{Other}{#1}{#2}}{}
\newenvironment{EXE}[2]{\CustomHeading{Exercise}{#1}{#2}}{}
\newenvironment{MOT}[2]{\CustomHeading{Motivation}{#1}{#2}}{}
\newenvironment{PROOF}[2]{\CustomHeading{Proof}{#1}{#2}}{}



% --- Unit Umgebung ---
\usepackage{mdframed}
\newmdenv[
  linewidth=1pt,
  topline=false,
  bottomline=false,
  rightline=false,
  leftmargin=0cm,
  rightmargin=0cm,
  skipabove=10pt,
  skipbelow=10pt,
  innertopmargin=0.5\baselineskip,
  innerbottommargin=0.5\baselineskip,
  backgroundcolor=gray!10,
  linecolor=gray
]{unitbox}

\newenvironment{unit}[1]
  {\begin{unitbox}\textbf{Unit #1}\par\smallskip}
  {\end{unitbox}}


% --- 17. Titel und Autor ---
\title{Mein Titel}
\author{Tim Jaschik}
\date{\today}

\begin{document}

\maketitle
\rule{\textwidth}{0.5pt}
\begin{abstract}
Kurze Beschreibung …
\end{abstract}
\rule{\textwidth}{0.5pt}
\vspace{0.5cm}

\tableofcontents

\pagebreak

\section{TESTSection}

\subsection{TESTSubsection}

\begin{PROP}{RG-A11-16-26}{Zeitableitung der Metrik-abhängigen Volumenform}
Suppose $g(t)$ is a smooth one-parameter family of metrics on a manifold $M$ with $\frac{\partial}{\partial t} g=h$. Then the volume form $d \mu(g(t))$ evolves by
$$
\frac{\partial}{\partial t} d \mu=\frac{1}{2} \operatorname{tr} h d \mu .
$$
\end{PROP}

\begin{PROOF}{RG-A11-16-27}{P: Zeitableitung der Metrik-abhängigen Volumenform}
In local coordinates ( $x^{i}$ ) the volume form can be written as $d \mu=\sqrt{\operatorname{det} g} d x^{1} \wedge \ldots \wedge x^{n}$. So by (4.12) and the chain rule,
$$
\begin{aligned}
\frac{\partial}{\partial t} \sqrt{\operatorname{det} g} & =\frac{1}{2} \frac{1}{\sqrt{\operatorname{det} g}} \frac{\partial}{\partial t} \operatorname{det} g \\
& =\frac{1}{2} \frac{1}{\sqrt{\operatorname{det} g}} \frac{\partial \operatorname{det} g}{\partial g_{i j}} \frac{\partial g_{i j}}{\partial t} \\
& =\frac{1}{2} \sqrt{\operatorname{det} g}\left(g^{-1}\right)_{j i} h_{i j} \\
& =\frac{1}{2} g^{i j} h_{i j} \sqrt{\operatorname{det} g}=\frac{1}{2} \operatorname{tr} h \sqrt{\operatorname{det} g}
\end{aligned}
$$
Thus,
$$
\frac{\partial}{\partial t} d \mu=\frac{\partial \sqrt{\operatorname{det} g}}{\partial t} d x^{1} \wedge \ldots \wedge x^{n}=\frac{1}{2} \operatorname{tr} h d \mu
$$
\end{PROOF}

\begin{THEO}{RG-A11-16-28}{Evolutionsformel für Riemannischen $(4,0)$ Krümmungstensor abhängig von Lösungen des Ricci-Flusses}
Suppose $g(t)$ is a solution of the Ricci flow, the $(4,0)$ Riemannian tensor $R$ evolves by
$$
\begin{aligned}
\frac{\partial}{\partial t} R_{i j k \ell}= & \Delta R_{i j k \ell}+2\left(B_{i j k \ell}-B_{i j \ell k}-B_{i \ell j k}+B_{i k j \ell}\right) \\
& -g^{p q}\left(R_{p j k \ell} R_{q i}+R_{i p k \ell} R_{q j}+R_{i j k p} R_{q \ell}+R_{i j p \ell} R_{q k}\right) .
\end{aligned}
$$
\end{THEO}

\begin{PROOF}{RG-A11-16-29}{P: Evolutionsformel für Riemannischen $(4,0)$ Krümmungstensor abhängig von Lösungen des Ricci-Flusses}
By Proposition 4.8, with $\frac{\partial}{\partial t} g_{i j}=-2 R_{i j}$, the time derivative of $R_{i j k \ell}$ satisfies
$$
\begin{aligned}
& \nabla_{i, \ell}^{2} R_{j k}+\nabla_{j, k}^{2} R_{i \ell}-\nabla_{i, k}^{2} R_{j \ell}-\nabla_{j, \ell}^{2} R_{i k} \\
& \quad=-\frac{\partial}{\partial t} R_{i j k \ell}-g^{p q}\left(R_{i j k p} R_{q \ell}+R_{i j p \ell} R_{q k}\right) .
\end{aligned}
$$
By Proposition 4.2, with indices $k$ and $\ell$ switched, the Laplacian of $R$ satisfies
$$
\begin{aligned}
& \nabla_{i, \ell}^{2} R_{j k}-\nabla_{j, \ell}^{2} R_{i k}+\nabla_{j, k}^{2} R_{i \ell}-\nabla_{i, k}^{2} R_{j \ell} \\
& \quad=\Delta R_{i j \ell k}+2\left(B_{i j \ell k}-B_{i j k \ell}-B_{i k j \ell}+B_{i \ell j k}\right) \\
& \quad-g^{p q}\left(R_{q j \ell k} R_{p i}+R_{i q \ell k} R_{p j}\right) .
\end{aligned}
$$
Combining these equations gives
$$
\begin{aligned}
-\Delta R_{i j k \ell}=\Delta R_{i j \ell k}= & -\frac{\partial}{\partial t} R_{i j k \ell}-2\left(B_{i j \ell k}-B_{i j k \ell}-B_{i k j \ell}+B_{i \ell j k}\right) \\
& +g^{p q}\left(R_{q j \ell k} R_{p i}+R_{i q \ell k} R_{p j}\right) \\
& -g^{p q}\left(R_{i j k p} R_{q \ell}+R_{i j p \ell} R_{q k}\right)
\end{aligned}
$$
\end{PROOF}

\begin{PROOF}{RG-A11-16-30}{P: Evolutionsformel für Riemannischen $(4,0)$ Krümmungstensor abhängig von Lösungen des Ricci-Flusses}
By Proposition 4.8, with $\frac{\partial}{\partial t} g_{i j}=-2 R_{i j}$, the time derivative of $R_{i j k \ell}$ satisfies
$$
\begin{aligned}
& \nabla_{i, \ell}^{2} R_{j k}+\nabla_{j, k}^{2} R_{i \ell}-\nabla_{i, k}^{2} R_{j \ell}-\nabla_{j, \ell}^{2} R_{i k} \\
& \quad=-\frac{\partial}{\partial t} R_{i j k \ell}-g^{p q}\left(R_{i j k p} R_{q \ell}+R_{i j p \ell} R_{q k}\right) .
\end{aligned}
$$
By Proposition 4.2, with indices $k$ and $\ell$ switched, the Laplacian of $R$ satisfies
$$
\begin{aligned}
& \nabla_{i, \ell}^{2} R_{j k}-\nabla_{j, \ell}^{2} R_{i k}+\nabla_{j, k}^{2} R_{i \ell}-\nabla_{i, k}^{2} R_{j \ell} \\
& \quad=\Delta R_{i j \ell k}+2\left(B_{i j \ell k}-B_{i j k \ell}-B_{i k j \ell}+B_{i \ell j k}\right) \\
& \quad-g^{p q}\left(R_{q j \ell k} R_{p i}+R_{i q \ell k} R_{p j}\right) .
\end{aligned}
$$
Combining these equations gives
$$
\begin{aligned}
-\Delta R_{i j k \ell}=\Delta R_{i j \ell k}= & -\frac{\partial}{\partial t} R_{i j k \ell}-2\left(B_{i j \ell k}-B_{i j k \ell}-B_{i k j \ell}+B_{i \ell j k}\right) \\
& +g^{p q}\left(R_{q j \ell k} R_{p i}+R_{i q \ell k} R_{p j}\right) \\
& -g^{p q}\left(R_{i j k p} R_{q \ell}+R_{i j p \ell} R_{q k}\right)
\end{aligned}
$$
\end{PROOF}

\begin{KORO}{RG-A11-16-31}{Evolution von Connection Koeffizienten unter Ricci Flow}
Under the Ricci flow, the connection coefficients evolve by
$$
\frac{\partial}{\partial t} \Gamma_{i j}^{k}=-g^{k \ell}\left(\left(\nabla_{j} \operatorname{Ric}\right)\left(\partial_{i}, \partial_{\ell}\right)+\left(\nabla_{i} \operatorname{Ric}\right)\left(\partial_{j}, \partial_{\ell}\right)-\left(\nabla_{\ell} \operatorname{Ric}\right)\left(\partial_{i}, \partial_{j}\right)\right)
$$
\end{KORO}

\begin{KORO}{RG-A11-16-32}{Evolution der Volumenform unter Ricci Flow}
Under the Ricci flow, the volume form of $g$ evolves by
$$
\frac{\partial}{\partial t} d \mu=-\operatorname{Scal} d \mu
$$
\end{KORO}

\begin{KORO}{RG-A11-16-33}{Evolution des Ricci-Krümmungstensors unter Ricci Flow}
Under the Ricci flow,
$$
\begin{aligned}
\frac{\partial}{\partial t} R_{i k} & =\Delta R_{i k}+\nabla_{i k}^{2} \mathrm{Scal}-g^{p q}\left(\nabla_{q, i}^{2} R_{k p}+\nabla_{q, k}^{2} R_{i p}\right)
\end{KORO}

\begin{KORO}{RG-A11-16-34}{Evolution des skalaren Krümmung unter Ricci Flow}
Under the Ricci flow,
\frac{\partial}{\partial t} \mathrm{Scal} & =2 \Delta \mathrm{Scal}-2 g^{i j} g^{p q} \nabla_{q, j}^{2} R_{p i}+2|\mathrm{Ric}|^{2}
\end{aligned}
$$
\end{KORO}

\begin{KORO}{RG-A11-16-35}{Evolution des Ricci-Krümmungstensor unter Ricci Flow 2}
Under the Ricci flow,
$$
\frac{\partial}{\partial t} R_{i k}=\Delta R_{i k}+2 g^{p q} g^{r s} R_{p i k r} R_{q s}-2 g^{p q} R_{i p} R_{q k}
$$
\end{KORO}

\begin{PROOF}{RG-A11-16-36}{P: Evolution des Ricci-Krümmungstensor unter Ricci Flow 2}
By (4.9) the time derivative of $R_{i k}=g^{j \ell} R_{i j k \ell}$ is $\frac{\partial}{\partial t} R_{i k}=g^{j \ell} \frac{\partial}{\partial t} R_{i j k \ell}$ $-2 g^{j p} g^{\ell q} R_{p q} R_{i j k \ell}$. Substituting the expression for $\frac{\partial}{\partial t} R_{i j k \ell}$ in Theorem 4.14 (with $g^{j \ell} \Delta R_{i j k \ell}=\Delta R_{i k}$ from (2.15)) results in

$$\begin{aligned}
\frac{\partial}{\partial t} R_{i k}= & \Delta R_{i k}+2 g^{j \ell}\left(B_{i j k \ell}-B_{i j \ell k}-B_{i \ell j k}+B_{i k j \ell}\right) \\
& -g^{j \ell} g^{p q}\left(R_{p j k \ell} R_{q i}+R_{i p k \ell} R_{q j}+R_{i j p \ell} R_{q k}+R_{i j k p} R_{q \ell}\right)
\end{aligned}$$

As we find that

$$\begin{aligned}
& 2 g^{j \ell}\left(B_{i j k \ell}-B_{i j \ell k}-B_{i \ell j k}+B_{i k j \ell}\right) \\
& \quad=2 g^{j \ell} B_{i j k \ell}-2 g^{j \ell}\left(B_{i \ell j k}+B_{i j \ell k}\right)+2 g^{p r} g^{q s} R_{p i q k} R_{r s} \\
& \quad=2 g^{j \ell} B_{i j k \ell}-4 g^{j \ell} B_{i j \ell k}+2 g^{p r} g^{q s} R_{p i q k} R_{r s} \\
& \quad=2 g^{j \ell}\left(B_{i j k \ell}-2 B_{i j \ell k}\right)+2 g^{p r} g^{q s} R_{p i q k} R_{r s}
\end{aligned}$$
and

$$\begin{aligned}
& g^{j \ell} g^{p q}\left(R_{p j k \ell} R_{q i}+R_{i p k \ell} R_{q j}+R_{i j p \ell} R_{q k}+R_{i j k p} R_{q \ell}\right) \\
& \quad=2 g^{p q} R_{p i} R_{q k}+g^{j \ell} g^{p q} R_{i p k \ell} R_{q j}+g^{j \ell} g^{p q} R_{i j k p} R_{q \ell} \\
& \quad=2 g^{p q} R_{p i} R_{q k}+2 g^{p r} g^{q s} R_{p i q k} R_{r s},
\end{aligned}$$
it follows that
$$\frac{\partial}{\partial t} R_{i k}=\Delta R_{i k}+2 g^{j \ell}\left(B_{i j k \ell}-2 B_{i j \ell k}\right)+2 g^{p r} g^{q s} R_{p i q k} R_{r s}-2 g^{p q} R_{p i} R_{q k}$$
The desired result now follows from the following claim.

Claim 4.19. For any metric $g_{i j}$, the tensor $B_{i j k \ell}$ satisfies the identity
$$g^{j \ell}\left(B_{i j k \ell}-2 B_{i j \ell k}\right)=0$$

Proof of Claim. 

Using the Bianchi identities,

$$\begin{aligned}
g^{j \ell} B_{i j k \ell} & =g^{j \ell} g^{p r} g^{q s} R_{p i q j} R_{r k s \ell} \\
& =g^{j \ell} g^{p r} g^{q s} R_{p q i j} R_{r s k \ell} \\
& =g^{j \ell} g^{p r} g^{q s}\left(R_{p i q j}-R_{p j q i}\right)\left(R_{r k s \ell}-R_{r \ell s k}\right) \\
& =2 g^{j \ell}\left(B_{i j k \ell}-B_{i j \ell k}\right)
\end{aligned}$$
\end{PROOF}

\begin{KORO}{RG-A11-16-37}{Evolution des skalaren Krümmung unter Ricci Flow 2}
Under the Ricci flow,
$$
\frac{\partial}{\partial t} \text { Scal }=\Delta \text { Scal }+2  |\text{Ric}|^{2}
$$
\end{KORO}

\begin{PROOF}{RG-A11-16-38}{P: Evolution des skalaren Krümmung unter Ricci Flow 2}
By (4.9) the time derivative of $R_{i k}=g^{j \ell} R_{i j k \ell}$ is $\frac{\partial}{\partial t} R_{i k}=g^{j \ell} \frac{\partial}{\partial t} R_{i j k \ell}$ $-2 g^{j p} g^{\ell q} R_{p q} R_{i j k \ell}$. Substituting the expression for $\frac{\partial}{\partial t} R_{i j k \ell}$ in Theorem 4.14 (with $g^{j \ell} \Delta R_{i j k \ell}=\Delta R_{i k}$ from (2.15)) results in
$$\begin{aligned}
\frac{\partial}{\partial t} R_{i k}= & \Delta R_{i k}+2 g^{j \ell}\left(B_{i j k \ell}-B_{i j \ell k}-B_{i \ell j k}+B_{i k j \ell}\right) \\
& -g^{j \ell} g^{p q}\left(R_{p j k \ell} R_{q i}+R_{i p k \ell} R_{q j}+R_{i j p \ell} R_{q k}+R_{i j k p} R_{q \ell}\right)
\end{aligned}$$
As we find that
$$\begin{aligned}
& 2 g^{j \ell}\left(B_{i j k \ell}-B_{i j \ell k}-B_{i \ell j k}+B_{i k j \ell}\right) \\
& \quad=2 g^{j \ell} B_{i j k \ell}-2 g^{j \ell}\left(B_{i \ell j k}+B_{i j \ell k}\right)+2 g^{p r} g^{q s} R_{p i q k} R_{r s} \\
& \quad=2 g^{j \ell} B_{i j k \ell}-4 g^{j \ell} B_{i j \ell k}+2 g^{p r} g^{q s} R_{p i q k} R_{r s} \\
& \quad=2 g^{j \ell}\left(B_{i j k \ell}-2 B_{i j \ell k}\right)+2 g^{p r} g^{q s} R_{p i q k} R_{r s}
\end{aligned}$$
and
$$\begin{aligned}
& g^{j \ell} g^{p q}\left(R_{p j k \ell} R_{q i}+R_{i p k \ell} R_{q j}+R_{i j p \ell} R_{q k}+R_{i j k p} R_{q \ell}\right) \\
& \quad=2 g^{p q} R_{p i} R_{q k}+g^{j \ell} g^{p q} R_{i p k \ell} R_{q j}+g^{j \ell} g^{p q} R_{i j k p} R_{q \ell} \\
& \quad=2 g^{p q} R_{p i} R_{q k}+2 g^{p r} g^{q s} R_{p i q k} R_{r s},
\end{aligned}$$
it follows that
$$\frac{\partial}{\partial t} R_{i k}=\Delta R_{i k}+2 g^{j \ell}\left(B_{i j k \ell}-2 B_{i j \ell k}\right)+2 g^{p r} g^{q s} R_{p i q k} R_{r s}-2 g^{p q} R_{p i} R_{q k}$$
The desired result now follows from the following claim.

Claim 4.19. For any metric $g_{i j}$, the tensor $B_{i j k \ell}$ satisfies the identity
$$g^{j \ell}\left(B_{i j k \ell}-2 B_{i j \ell k}\right)=0$$
Proof of Claim. 

Using the Bianchi identities,
$$\begin{aligned}
g^{j \ell} B_{i j k \ell} & =g^{j \ell} g^{p r} g^{q s} R_{p i q j} R_{r k s \ell} \\
& =g^{j \ell} g^{p r} g^{q s} R_{p q i j} R_{r s k \ell} \\
& =g^{j \ell} g^{p r} g^{q s}\left(R_{p i q j}-R_{p j q i}\right)\left(R_{r k s \ell}-R_{r \ell s k}\right) \\
& =2 g^{j \ell}\left(B_{i j k \ell}-B_{i j \ell k}\right)
\end{aligned}$$
Note
\end{PROOF}

\subsection{TESTSubsection2}

\end{document}