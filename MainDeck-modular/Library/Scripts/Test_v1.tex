\documentclass[10pt, letterpaper]{article}

% Inhaltsverzeichnis für Pakettypen (nur für Übersicht im Header, wird nicht im Dokument angezeigt)
% 1. Seitenlayout und Ränder
% 2. Sprache und Zeichensatz
% 3. Mathematik und Theorem-Umgebungen
% 4. Eigene Makros
% 5. Diagramme und Grafiken
% 6. Tabellen und Aufzählungen
% 7. Inhaltsverzeichnis
% 8. Abschnittsüberschriften
% 9. Abstrakt-Umgebung
% 10. Todos/Notizen
% 11. Rahmen/Box-Umgebungen
% 12. Python-Integration
% 13. Literaturverwaltung
% 14. Hyperlinks
% 15. Absatzeinstellungen
% 16. Umgebungen
% 17. Titel und Autor

% --- 1. Seitenlayout und Ränder ---
\usepackage[margin=3cm]{geometry}

% --- 2. Sprache und Zeichensatz ---
\usepackage[english]{babel}
\usepackage[T1]{fontenc}
\usepackage[utf8]{inputenc}

% --- 3. Mathematik und Theorem-Umgebungen ---
\usepackage{amsmath, amssymb, amsthm}
\usepackage{mathrsfs}
\DeclareMathOperator{\WF}{WF}

% --- 4. Eigene Makros ---
\usepackage{xcolor}
\newcommand{\SKP}{\langle\cdot,\cdot\rangle}
\newcommand{\R}{\mathbb{R}}
\newcommand{\N}{\mathbb{N}}
\newcommand{\Q}{\mathbb{Q}}
\newcommand{\Z}{\mathbb{Z}}
\newcommand{\C}{\mathbb{C}}
\newcommand{\entwurf}[1]{\textcolor{red}{#1}}

% --- 5. Diagramme und Grafiken ---
\usepackage{graphicx}
\usepackage{tikz}
\usetikzlibrary{decorations.pathreplacing, arrows.meta, positioning}
\usepackage{tikz-cd}

% --- 6. Tabellen und Aufzählungen ---
\usepackage{enumitem}
\setlist[itemize]{left=0.5cm}

\newenvironment{romanenum}[1][]
  {%
    \ifx&#1&
    \else
      \textbf{#1}\quad
    \fi
    \begin{enumerate}[label=\roman*)]
  }
  {%
    \end{enumerate}%
  }

% --- 7. Inhaltsverzeichnis ---
\usepackage{tocloft}
\renewcommand{\cftsecfont}{\footnotesize}
\renewcommand{\cftsubsecfont}{\footnotesize}
\renewcommand{\cftsubsubsecfont}{\footnotesize}
\renewcommand{\cftsecpagefont}{\footnotesize}
\renewcommand{\cftsubsecpagefont}{\footnotesize}
\renewcommand{\cftsubsubsecpagefont}{\footnotesize}
\usepackage{etoc}

% --- 8. Abschnittsüberschriften ---
\usepackage{titlesec}
\titleformat{\section}{\normalfont\large\bfseries}{\thesection}{1em}{}
\titleformat{\subsection}{\normalfont\normalsize\bfseries}{\thesubsection}{0.5em}{}
\titleformat{\subsubsection}{\normalfont\normalsize\bfseries}{\thesubsubsection}{0.5em}{}
\setcounter{secnumdepth}{4}

% --- 9. Abstrakt-Umgebung ---
\usepackage{changepage}
\renewenvironment{abstract}
  {
    \begin{adjustwidth}{1.5cm}{1.5cm}
    \small
    \textsc{Abstract. –}%
  }
  {
    \end{adjustwidth}
  }

% --- 10. Todos/Notizen ---
\usepackage{todonotes}

% --- 11. Rahmen/Box-Umgebungen ---
\usepackage{mdframed}
\usepackage{tcolorbox}
\colorlet{shadecolor}{gray!25}

\newenvironment{customTheorem}
  {\vspace{10pt}%
   \begin{mdframed}[
     backgroundcolor=gray!20,
     linewidth=0pt,
     innertopmargin=10pt,
     innerbottommargin=10pt,
     skipabove=\dimexpr\topsep+\ht\strutbox\relax,
     skipbelow=\topsep,
   ]}
  {\end{mdframed}
   \vspace{10pt}%
  }

% --- 12. Python-Integration ---
% (Deaktiviert in dieser Version, aktiviere bei Bedarf)
% \usepackage{pythontex}
% \usepackage[makestderr]{pythontex}

% --- 13. Literaturverwaltung ---
\usepackage{csquotes}
\usepackage[backend=biber, style=alphabetic, citestyle=alphabetic]{biblatex}
\addbibresource{bibliography.bib}

% --- 14. Hyperlinks ---
\usepackage{hyperref}
\hypersetup{
  colorlinks   = true,
  urlcolor     = blue,
  linkcolor    = blue,
  citecolor    = blue,
  frenchlinks  = true
}

% --- 15. Absatzeinstellungen ---
\usepackage[parfill]{parskip}
\sloppy

% --- 16. Umgebungen ---
\usepackage{thmtools}

\newcommand{\CustomHeading}[3]{%
  \par\medskip\noindent%
  \textbf{#1 #2} \textnormal{(#3)}.\enskip%
}

\newenvironment{DEF}[2]{\CustomHeading{Definition}{#1}{#2}}{}
\newenvironment{PROP}[2]{\CustomHeading{Proposition}{#1}{#2}}{}
\newenvironment{THEO}[2]{\CustomHeading{Theorem}{#1}{#2}}{}
\newenvironment{LEM}[2]{\CustomHeading{Lemma}{#1}{#2}}{}
\newenvironment{KORO}[2]{\CustomHeading{Corollar}{#1}{#2}}{}
\newenvironment{REM}[2]{\CustomHeading{Remark}{#1}{#2}}{}
\newenvironment{EXA}[2]{\CustomHeading{Example}{#1}{#2}}{}
\newenvironment{STUD}[2]{\CustomHeading{Study}{#1}{#2}}{}
\newenvironment{CONC}[2]{\CustomHeading{Concept}{#1}{#2}}{}

\newenvironment{PROOF}
  {\begin{proof}}%
{\end{proof}}


% --- Unit Umgebung ---
\usepackage{mdframed}
\newmdenv[
  linewidth=1pt,
  topline=false,
  bottomline=false,
  rightline=false,
  leftmargin=0cm,
  rightmargin=0cm,
  skipabove=10pt,
  skipbelow=10pt,
  innertopmargin=0.5\baselineskip,
  innerbottommargin=0.5\baselineskip,
  backgroundcolor=gray!10,
  linecolor=gray
]{unitbox}

\newenvironment{unit}[1]
  {\begin{unitbox}\textbf{Unit #1}\par\smallskip}
  {\end{unitbox}}


% --- 17. Titel und Autor ---
\title{Mein Titel}
\author{Tim Jaschik}
\date{\today}

\begin{document}

\maketitle
\rule{\textwidth}{0.5pt}
\begin{abstract}
Kurze Beschreibung …
\end{abstract}
\rule{\textwidth}{0.5pt}
\vspace{0.5cm}

\tableofcontents

\pagebreak

\section{Test}

\subsection{Test}

% Auto‑generiert aus Library.tex
% UnitID: A-1-03-01
% Titel : Ring mit Eins

\begin{DEF}{A-1-03-01}{Ring mit Eins}
% TODO: Inhalt ergänzen (Tex)
\end{DEF}

% Auto‑generiert aus Library.tex
% UnitID: A-1-03-02
% Titel : Ring ohne Eins

\begin{DEF}{A-1-03-02}{Ring ohne Eins}
% TODO: Inhalt ergänzen (Tex)
\end{DEF}

% Auto‑generiert aus Library.tex
% UnitID: A-1-03-03
% Titel : Kommutativer Ring

\begin{DEF}{A-1-03-03}{Kommutativer Ring}
Sei $\Omega \subset \mathbb{R}^{N}$ ein beschränktes Gebiet, $q \in L^{\infty}(\Omega)$ nichtnegativ und $f \in L^{2}(\Omega)$. Dann hat das Dirichletproblem
$$
-\Delta u+q(x) u=f \quad \text { in } \Omega, \quad u=0 \quad \text { auf } \partial \Omega
$$
eine eindeutig bestimmte schwache Lösung $u \in H_{0}^{1}(\Omega)$. Ist ferner $f \in L^{\infty}(\Omega)$, so gilt:

(i) $u \in C^{1}(\Omega) \cap L^{\infty}(\Omega)$.

(ii) Ist $\Omega^{\prime} \subset \subset \Omega$, so existiert eine nur von $\|q\|_{\infty}$ und $\Omega^{\prime}$ abhängige Konstante $C_{1}>0 \mathrm{mit}$
$$
\|u\|_{C^{1}\left(\overline{\Omega^{\prime}}\right)} \leq C_{1}\left(\|u\|_{L^{\infty}(\Omega)}+\|f\|_{L^{\infty}(\Omega)}\right)
$$

(iii) Erfüllt $\Omega$ eine gleichmäßige äußere Sphärenbedingung, so gilt $u \in C_{0}(\Omega)$, und es existiert eine nur von $\|f\|_{L^{\infty}(\Omega)}$ abhängige Konstante $C_{2}$ mit
$$
|u(x)| \leq C_{2} \operatorname{dist}(x, \partial \Omega) \quad \text { für } x \in \Omega \text {. }
$$

Erinnerung: $u \in H_{0}^{1}(\Omega)$ heißt schwache Lösung von (1.1), falls
$$
a_{L}(u, \varphi):=\int_{\Omega}(\nabla u \nabla \varphi+q(x) u \varphi) d x=\int_{\Omega} f u d x \quad \text { für alle } \varphi \in H_{0}^{1}(\Omega) \text {. }
$$
Wir testen auf Veränderung.
\end{DEF}

% Auto‑generiert aus Library.tex
% UnitID: A-1-03-04
% Titel : Körper sind Ringe

\begin{EXA}{A-1-03-04}{Körper sind Ringe}
% TODO: Inhalt ergänzen (Tex)
\end{EXA}

% Auto‑generiert aus Library.tex
% UnitID: A-1-03-05
% Titel : $(\Z,+,*)$ kommutaiver Ring

\begin{EXA}{A-1-03-05}{$(\Z,+,*)$ kommutaiver Ring}
% TODO: Inhalt ergänzen (Tex)
\end{EXA}

% Auto‑generiert aus Library.tex
% UnitID: A-1-03-06
% Titel : Ring der Funktionen

\begin{EXA}{A-1-03-06}{Ring der Funktionen}
% TODO: Inhalt ergänzen (Tex)
\end{EXA}

% Auto‑generiert aus Library.tex
% UnitID: A-1-03-07
% Titel : Matrizenringe über Körper

\begin{EXA}{A-1-03-07}{Matrizenringe über Körper}
% TODO: Inhalt ergänzen (Tex)
\end{EXA}

% Auto‑generiert aus Library.tex
% UnitID: A-1-03-08
% Titel : $(End_k(V),+,\circ)$ Ring

\begin{EXA}{A-1-03-08}{$(End_k(V),+,\circ)$ Ring}
% TODO: Inhalt ergänzen (Tex)
\end{EXA}

% Auto‑generiert aus Library.tex
% UnitID: A-1-03-09
% Titel : Matrizenring über Ring

\begin{EXA}{A-1-03-09}{Matrizenring über Ring}
% TODO: Inhalt ergänzen (Tex)
\end{EXA}

% Auto‑generiert aus Library.tex
% UnitID: A-1-03-10
% Titel : Nullring

\begin{EXA}{A-1-03-10}{Nullring}
% TODO: Inhalt ergänzen (Tex)
\end{EXA}

% Auto‑generiert aus Library.tex
% UnitID: A-1-03-11
% Titel : Produktring

\begin{EXA}{A-1-03-11}{Produktring}
% TODO: Inhalt ergänzen (Tex)
\end{EXA}

% Auto‑generiert aus Library.tex
% UnitID: A-1-03-12
% Titel : Gruppenring mit Koeffizienten aus Körper

\begin{EXA}{A-1-03-12}{Gruppenring mit Koeffizienten aus Körper}
% TODO: Inhalt ergänzen (Tex)
\end{EXA}

% Auto‑generiert aus Library.tex
% UnitID: A-1-03-13
% Titel : Eins eines Ringes mit Eins ist eindeutig

\begin{REM}{A-1-03-13}{Eins eines Ringes mit Eins ist eindeutig}
% TODO: Inhalt ergänzen (Tex)
\end{REM}

% Auto‑generiert aus Library.tex
% UnitID: A-1-03-14
% Titel : Rechenregeln für Ringe mit Eins

\begin{LEM}{A-1-03-14}{Rechenregeln für Ringe mit Eins}
% TODO: Inhalt ergänzen (Tex)
\end{LEM}

% Auto‑generiert aus Library.tex
% UnitID: A-1-03-15
% Titel : Wenn Ring mit $0=1$, dann Nullring

\begin{LEM}{A-1-03-15}{Wenn Ring mit $0=1$, dann Nullring}
% TODO: Inhalt ergänzen (Tex)
\end{LEM}

% Auto‑generiert aus Library.tex
% UnitID: A-1-03-16
% Titel : Ringhomomorphismus

\begin{DEF}{A-1-03-16}{Ringhomomorphismus}
% TODO: Inhalt ergänzen (Tex)
\end{DEF}

% Auto‑generiert aus Library.tex
% UnitID: A-1-03-17
% Titel : Ringhomomorphismen induzieren Gruppenhomomorphismen zwischen abelschen Gruppen

\begin{REM}{A-1-03-17}{Ringhomomorphismen induzieren Gruppenhomomorphismen zwischen abelschen Gruppen}
% TODO: Inhalt ergänzen (Tex)
\end{REM}

% Auto‑generiert aus Library.tex
% UnitID: A-1-03-18
% Titel : Pullback-Ringhomomorphismus

\begin{EXA}{A-1-03-18}{Pullback-Ringhomomorphismus}
% TODO: Inhalt ergänzen (Tex)
\end{EXA}

% Auto‑generiert aus Library.tex
% UnitID: A-1-03-19
% Titel : Einschränkung als Pullback der Inklusion

\begin{EXA}{A-1-03-19}{Einschränkung als Pullback der Inklusion}
% TODO: Inhalt ergänzen (Tex)
\end{EXA}

% Auto‑generiert aus Library.tex
% UnitID: A-1-03-20
% Titel : Auswertungshomomorphismus für Punkt-Inklusion

\begin{EXA}{A-1-03-20}{Auswertungshomomorphismus für Punkt-Inklusion}
% TODO: Inhalt ergänzen (Tex)
\end{EXA}

% Auto‑generiert aus Library.tex
% UnitID: A-1-03-21
% Titel : R-Linearkombination in Ringen

\begin{DEF}{A-1-03-21}{R-Linearkombination in Ringen}
% TODO: Inhalt ergänzen (Tex)
\end{DEF}

% Auto‑generiert aus Library.tex
% UnitID: A-1-03-22
% Titel : Unterring eines Ringes

\begin{DEF}{A-1-03-22}{Unterring eines Ringes}
% TODO: Inhalt ergänzen (Tex)
\end{DEF}

% Auto‑generiert aus Library.tex
% UnitID: A-1-03-23
% Titel : Bild von Ringhomomorphismen ist ein Unterring

\begin{EXA}{A-1-03-23}{Bild von Ringhomomorphismen ist ein Unterring}
% TODO: Inhalt ergänzen (Tex)
\end{EXA}

% Auto‑generiert aus Library.tex
% UnitID: A-1-03-24
% Titel : Einheiten in Ringen

\begin{DEF}{A-1-03-24}{Einheiten in Ringen}
% TODO: Inhalt ergänzen (Tex)
\end{DEF}

% Auto‑generiert aus Library.tex
% UnitID: A-1-03-25
% Titel : Einheitsgruppe: Menge der Einheiten in Ringen sind Gruppe bzgl. Multiplikation in R

\begin{PROP}{A-1-03-25}{Einheitsgruppe: Menge der Einheiten in Ringen sind Gruppe bzgl. Multiplikation in R}
% TODO: Inhalt ergänzen (Tex)
\end{PROP}

% Auto‑generiert aus Library.tex
% UnitID: A-1-03-26
% Titel : Einheitengruppe von ganzen Zahlen

\begin{EXA}{A-1-03-26}{Einheitengruppe von ganzen Zahlen}
% TODO: Inhalt ergänzen (Tex)
\end{EXA}

% Auto‑generiert aus Library.tex
% UnitID: A-1-03-27
% Titel : Einheitengruppe von Gruppenringe

\begin{EXA}{A-1-03-27}{Einheitengruppe von Gruppenringe}
% TODO: Inhalt ergänzen (Tex)
\end{EXA}

% Auto‑generiert aus Library.tex
% UnitID: A-1-03-28
% Titel : Einheiten von Matrizenringe mit Koeffizienten in Körper

\begin{EXA}{A-1-03-28}{Einheiten von Matrizenringe mit Koeffizienten in Körper}
% TODO: Inhalt ergänzen (Tex)
\end{EXA}

% Auto‑generiert aus Library.tex
% UnitID: A-1-03-29
% Titel : Ringhomomorphismen bilden Einheiten auf Einheiten ab und induzieren G-Hom auf Einheitsgruppen

\begin{PROP}{A-1-03-29}{Ringhomomorphismen bilden Einheiten auf Einheiten ab und induzieren G-Hom auf Einheitsgruppen}
% TODO: Inhalt ergänzen (Tex)
\end{PROP}

% Auto‑generiert aus Library.tex
% UnitID: A-1-03-30
% Titel : Schiefkörper als Ring mit Einheitsgruppe $= R$ ohne $0$

\begin{DEF}{A-1-03-30}{Schiefkörper als Ring mit Einheitsgruppe $= R$ ohne $0$}
% TODO: Inhalt ergänzen (Tex)
\end{DEF}

% Auto‑generiert aus Library.tex
% UnitID: A-1-03-31
% Titel : Körper als abelscher Schiefkörper

\begin{DEF}{A-1-03-31}{Körper als abelscher Schiefkörper}
% TODO: Inhalt ergänzen (Tex)
\end{DEF}

% Auto‑generiert aus Library.tex
% UnitID: A-1-03-32
% Titel : Quaternionen als nichtkommutativer Schiefkörper

\begin{EXA}{A-1-03-32}{Quaternionen als nichtkommutativer Schiefkörper}
% TODO: Inhalt ergänzen (Tex)
\end{EXA}

\end{document}