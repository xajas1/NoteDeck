
% … (dein gesamter Header bleibt hier unverändert)


\documentclass[10pt, letterpaper]{article}

% Inhaltsverzeichnis für Pakettypen (nur für Übersicht im Header, wird nicht im Dokument angezeigt)
% 1. Seitenlayout und Ränder
% 2. Sprache und Zeichensatz
% 3. Mathematik und Theorem-Umgebungen
% 4. Eigene Makros
% 5. Diagramme und Grafiken
% 6. Tabellen und Aufzählungen
% 7. Inhaltsverzeichnis
% 8. Abschnittsüberschriften
% 9. Abstrakt-Umgebung
% 10. Todos/Notizen
% 11. Rahmen/Box-Umgebungen
% 12. Python-Integration
% 13. Literaturverwaltung
% 14. Hyperlinks
% 15. Absatzeinstellungen
% 16. Umgebungen
% 17. Titel und Autor

% --- 1. Seitenlayout und Ränder ---
\usepackage[margin=3cm]{geometry}

% --- 2. Sprache und Zeichensatz ---
\usepackage[english]{babel}
\usepackage[T1]{fontenc}
\usepackage[utf8]{inputenc}

% --- 3. Mathematik und Theorem-Umgebungen ---
\usepackage{amsmath, amssymb, amsthm}
\usepackage{mathrsfs}
\DeclareMathOperator{\WF}{WF}

% --- 4. Eigene Makros ---
\usepackage{xcolor}
\newcommand{\SKP}{\langle\cdot,\cdot\rangle}
\newcommand{\R}{\mathbb{R}}
\newcommand{\N}{\mathbb{N}}
\newcommand{\Q}{\mathbb{Q}}
\newcommand{\Z}{\mathbb{Z}}
\newcommand{\C}{\mathbb{C}}
\newcommand{\entwurf}[1]{\textcolor{red}{#1}}

% --- 5. Diagramme und Grafiken ---
\usepackage{graphicx}
\usepackage{tikz}
\usetikzlibrary{decorations.pathreplacing, arrows.meta, positioning}
\usepackage{tikz-cd}

% --- 6. Tabellen und Aufzählungen ---
\usepackage{enumitem}
\setlist[itemize]{left=0.5cm}

\newenvironment{romanenum}[1][]
  {%
    \ifx&#1&
    \else
      \textbf{#1}\quad
    \fi
    \begin{enumerate}[label=\roman*)]
  }
  {%
    \end{enumerate}%
  }

% --- 7. Inhaltsverzeichnis ---
\usepackage{tocloft}
\renewcommand{\cftsecfont}{\footnotesize}
\renewcommand{\cftsubsecfont}{\footnotesize}
\renewcommand{\cftsubsubsecfont}{\footnotesize}
\renewcommand{\cftsecpagefont}{\footnotesize}
\renewcommand{\cftsubsecpagefont}{\footnotesize}
\renewcommand{\cftsubsubsecpagefont}{\footnotesize}
\usepackage{etoc}

% --- 8. Abschnittsüberschriften ---
\usepackage{titlesec}
\titleformat{\section}{\normalfont\large\bfseries}{\thesection}{1em}{}
\titleformat{\subsection}{\normalfont\normalsize\bfseries}{\thesubsection}{0.5em}{}
\titleformat{\subsubsection}{\normalfont\normalsize\bfseries}{\thesubsubsection}{0.5em}{}
\setcounter{secnumdepth}{4}

% --- 9. Abstrakt-Umgebung ---
\usepackage{changepage}
\renewenvironment{abstract}
  {
    \begin{adjustwidth}{1.5cm}{1.5cm}
    \small
    \textsc{Abstract. –}%
  }
  {
    \end{adjustwidth}
  }

% --- 10. Todos/Notizen ---
\usepackage{todonotes}

% --- 11. Rahmen/Box-Umgebungen ---
\usepackage{mdframed}
\usepackage{tcolorbox}
\colorlet{shadecolor}{gray!25}

\newenvironment{customTheorem}
  {\vspace{10pt}%
   \begin{mdframed}[
     backgroundcolor=gray!20,
     linewidth=0pt,
     innertopmargin=10pt,
     innerbottommargin=10pt,
     skipabove=\dimexpr\topsep+\ht\strutbox\relax,
     skipbelow=\topsep,
   ]}
  {\end{mdframed}
   \vspace{10pt}%
  }

% --- 12. Python-Integration ---
% (Deaktiviert in dieser Version, aktiviere bei Bedarf)
% \usepackage{pythontex}
% \usepackage[makestderr]{pythontex}

% --- 13. Literaturverwaltung ---
\usepackage{csquotes}
\usepackage[backend=biber, style=alphabetic, citestyle=alphabetic]{biblatex}
\addbibresource{bibliography.bib}

% --- 14. Hyperlinks ---
\usepackage{hyperref}
\hypersetup{
  colorlinks   = true,
  urlcolor     = blue,
  linkcolor    = blue,
  citecolor    = blue,
  frenchlinks  = true
}

% --- 15. Absatzeinstellungen ---
\usepackage[parfill]{parskip}
\sloppy

% --- 16. Umgebungen ---
\usepackage{thmtools}

\newcommand{\CustomHeading}[3]{%
  \par\medskip\noindent%
  \textbf{#1 #2} \textnormal{(#3)}.\enskip%
}

\newenvironment{DEF}[2]{\CustomHeading{Definition}{#1}{#2}}{}
\newenvironment{PROP}[2]{\CustomHeading{Proposition}{#1}{#2}}{}
\newenvironment{THEO}[2]{\CustomHeading{Theorem}{#1}{#2}}{}
\newenvironment{LEM}[2]{\CustomHeading{Lemma}{#1}{#2}}{}
\newenvironment{KORO}[2]{\CustomHeading{Corollar}{#1}{#2}}{}
\newenvironment{REM}[2]{\CustomHeading{Remark}{#1}{#2}}{}
\newenvironment{EXA}[2]{\CustomHeading{Example}{#1}{#2}}{}
\newenvironment{STUD}[2]{\CustomHeading{Study}{#1}{#2}}{}
\newenvironment{CONC}[2]{\CustomHeading{Concept}{#1}{#2}}{}

\newenvironment{PROOF}
  {\begin{proof}}%
{\end{proof}}

% --- 17. Titel und Autor ---
\title{Mein Titel}
\author{Tim Jaschik}
\date{\today}

\begin{document}

\maketitle
\rule{\textwidth}{0.5pt}
\begin{abstract}
Kurze Beschreibung …
\end{abstract}
\rule{\textwidth}{0.5pt}
\vspace{0.5cm}

\tableofcontents

\pagebreak




%-- AUTO-UNITS-START

\begin{DEF}{A-T12-Ri-29}{Ring ohne Eins}
% TODO: Inhalt ergänzen (Tex)
\end{DEF}

\begin{DEF}{A-T12-03-01}{Ring mit Eins}
% TODO: Inhalt ergänzen (Tex)
\end{DEF}

\begin{DEF}{A-T12-03-02}{Ring ohne Eins}
% TODO: Inhalt ergänzen (Tex)
\end{DEF}

\begin{DEF}{A-T12-03-03}{Kommutativer Ring}
% TODO: Inhalt ergänzen (Tex)
\end{DEF}

\begin{EXA}{A-T12-03-04}{Körper sind Ringe}
% TODO: Inhalt ergänzen (Tex)
\end{EXA}

\begin{EXA}{A-T12-03-05}{$(\Z,+,*)$ kommutaiver Ring}
% TODO: Inhalt ergänzen (Tex)
\end{EXA}

\begin{EXA}{A-T12-03-06}{Ring der Funktionen}
% TODO: Inhalt ergänzen (Tex)
\end{EXA}

\begin{EXA}{A-T12-03-07}{Matrizenringe über Körper}
% TODO: Inhalt ergänzen (Tex)
\end{EXA}

\begin{EXA}{A-T12-03-08}{$(End_k(V),+,\circ)$ Ring}
% TODO: Inhalt ergänzen (Tex)
\end{EXA}

\begin{EXA}{A-T12-03-09}{Matrizenring über Ring}
% TODO: Inhalt ergänzen (Tex)
\end{EXA}

\begin{EXA}{A-T12-03-10}{Nullring}
% TODO: Inhalt ergänzen (Tex)
\end{EXA}

\begin{EXA}{A-T12-03-11}{Produktring}
% TODO: Inhalt ergänzen (Tex)
\end{EXA}

\begin{EXA}{A-T12-03-12}{Gruppenring mit Koeffizienten aus Körper}
% TODO: Inhalt ergänzen (Tex)
\end{EXA}

\begin{REM}{A-T12-03-13}{Eins eines Ringes mit Eins ist eindeutig}
% TODO: Inhalt ergänzen (Tex)
\end{REM}

\begin{LEM}{A-T12-03-14}{Rechenregeln für Ringe mit Eins}
% TODO: Inhalt ergänzen (Tex)
\end{LEM}

\begin{LEM}{A-T12-03-15}{Wenn Ring mit $0=1$, dann Nullring}
% TODO: Inhalt ergänzen (Tex)
\end{LEM}

\begin{DEF}{A-T12-03-16}{Ringhomomorphismus}
% TODO: Inhalt ergänzen (Tex)
\end{DEF}

\begin{REM}{A-T12-03-17}{Ringhomomorphismen induzieren Gruppenhomomorphismen zwischen abelschen Gruppen}
% TODO: Inhalt ergänzen (Tex)
\end{REM}

\begin{EXA}{A-T12-03-18}{Pullback-Ringhomomorphismus}
% TODO: Inhalt ergänzen (Tex)
\end{EXA}

\begin{EXA}{A-T12-03-19}{Einschränkung als Pullback der Inklusion}
% TODO: Inhalt ergänzen (Tex)
\end{EXA}

\begin{EXA}{A-T12-03-20}{Auswertungshomomorphismus für Punkt-Inklusion}
% TODO: Inhalt ergänzen (Tex)
\end{EXA}

\begin{DEF}{A-T12-03-21}{R-Linearkombination in Ringen}
% TODO: Inhalt ergänzen (Tex)
\end{DEF}

\begin{DEF}{A-T12-03-22}{Unterring eines Ringes}
% TODO: Inhalt ergänzen (Tex)
\end{DEF}

\begin{EXA}{A-T12-03-23}{Bild von Ringhomomorphismen ist ein Unterring}
% TODO: Inhalt ergänzen (Tex)
\end{EXA}

\begin{DEF}{A-T12-03-24}{Einheiten in Ringen}
% TODO: Inhalt ergänzen (Tex)
\end{DEF}

\begin{PROP}{A-T12-03-25}{Einheitsgruppe: Menge der Einheiten in Ringen sind Gruppe bzgl. Multiplikation in R}
% TODO: Inhalt ergänzen (Tex)
\end{PROP}

\begin{EXA}{A-T12-03-26}{Einheitengruppe von ganzen Zahlen}
% TODO: Inhalt ergänzen (Tex)
\end{EXA}

\begin{EXA}{A-T12-03-27}{Einheitengruppe von Gruppenringe}
% TODO: Inhalt ergänzen (Tex)
\end{EXA}

\begin{EXA}{A-T12-03-28}{Einheiten von Matrizenringe mit Koeffizienten in Körper}
% TODO: Inhalt ergänzen (Tex)
\end{EXA}

\begin{REM}{A-T12-04-01}{Eindeutige Darstellung in Polynomringen}
% TODO: Inhalt ergänzen (Tex)
\end{REM}

\begin{REM}{A-T12-04-02}{Polynomring als Unterring der R-Linearkombinationen}
% TODO: Inhalt ergänzen (Tex)
\end{REM}

\begin{REM}{A-T12-04-03}{Eigenschaften der Gradfunktion von Leitkoeffizienten}
% TODO: Inhalt ergänzen (Tex)
\end{REM}

\begin{REM}{A-T12-04-04}{Identifikation von R als Unterring von Polynomring mit Koeff in R}
% TODO: Inhalt ergänzen (Tex)
\end{REM}

\begin{PROP}{A-T12-04-05}{Universelle Eigenschaft des Polynomringes: Auswertungs-Ringhomomorphismus}
% TODO: Inhalt ergänzen (Tex)
\end{PROP}

\begin{EXA}{A-T12-04-06}{Auswertungshomomorphismus für Abbildung von Körper in Matrzenring}
% TODO: Inhalt ergänzen (Tex)
\end{EXA}

\begin{EXA}{A-T12-04-07}{Auswertungshomomorphismus für Abbildung von Körper in Abbildungsring der $End_V$}
% TODO: Inhalt ergänzen (Tex)
\end{EXA}

\begin{DEF}{A-T12-04-08}{Polynomring in n-Variablen mit Koeffizienten aus Ring}
% TODO: Inhalt ergänzen (Tex)
\end{DEF}

\begin{LEM}{A-T12-04-09}{Eindeutige Darstellung in Polynomringen in n-Variablen}
% TODO: Inhalt ergänzen (Tex)
\end{LEM}

\begin{REM}{A-T12-04-10}{Multiindex-Schreibweise}
% TODO: Inhalt ergänzen (Tex)
\end{REM}

\begin{REM}{A-T12-04-11}{Induzierter Ringautomorphismus auf Polynomring durch Permutation}
% TODO: Inhalt ergänzen (Tex)
\end{REM}

\begin{LEM}{A-T12-04-12}{Gruppenhomomorphismus zwischen Symmetrische Gruppe und Gruppe der Ring-Automorphismen des Polynomringes in n-Variablen}
% TODO: Inhalt ergänzen (Tex)
\end{LEM}

\begin{DEF}{A-T12-04-13}{Symmetrisches Polynom}
% TODO: Inhalt ergänzen (Tex)
\end{DEF}

\begin{EXA}{A-T12-04-14}{Elementarsymmetrische Polynom in n-Variablen}
% TODO: Inhalt ergänzen (Tex)
\end{EXA}

\begin{PROP}{A-T12-04-15}{Vieta-Formel}
% TODO: Inhalt ergänzen (Tex)
\end{PROP}

\begin{PROP}{A-T12-04-16}{Jedes symmetrische Polynom ist ein Polynom in den elementarsymmetrischen Polynomen}
% TODO: Inhalt ergänzen (Tex)
\end{PROP}

\begin{REM}{A-T12-05-01}{Kern von Ringhomomorphismen nicht i.A. Unterring}
% TODO: Inhalt ergänzen (Tex)
\end{REM}

\begin{DEF}{A-T12-05-02}{Ideal eines Ringes}
% TODO: Inhalt ergänzen (Tex)
\end{DEF}

\begin{LEM}{A-T12-05-03}{Charakterisierung von Idealen}
% TODO: Inhalt ergänzen (Tex)
\end{LEM}

\begin{EXA}{A-T12-05-04}{$Rx$ sind Ideale von R}
% TODO: Inhalt ergänzen (Tex)
\end{EXA}

\begin{PROP}{A-T12-05-05}{Kern eines R-Homs ist ein Ideal}
% TODO: Inhalt ergänzen (Tex)
\end{PROP}

\begin{PROP}{A-T12-05-06}{R-Hom ist injektiv gdw $Kern = 0$}
% TODO: Inhalt ergänzen (Tex)
\end{PROP}

\begin{DEF}{A-T12-05-07}{Von Teilmengen erzeugte Ideale}
% TODO: Inhalt ergänzen (Tex)
\end{DEF}

\begin{REM}{A-T12-05-08}{Warum ist die Menge der erzeugten R-Linearkombinationen eine Ideal?}
% TODO: Inhalt ergänzen (Tex)
\end{REM}

\begin{LEM}{A-T12-05-09}{Schnitte von Idealen sind Ideale}
% TODO: Inhalt ergänzen (Tex)
\end{LEM}

\begin{DEF}{A-T12-05-10}{Erzeugendensysteme von Ideale}
% TODO: Inhalt ergänzen (Tex)
\end{DEF}

\begin{EXA}{A-T12-05-11}{$n\Z$}
% TODO: Inhalt ergänzen (Tex)
\end{EXA}

\begin{EXA}{A-T12-05-12}{$\{0\}$ und $\{1\}$ in jedem Ring sind Ideale}
% TODO: Inhalt ergänzen (Tex)
\end{EXA}

\begin{EXA}{A-T12-05-13}{$(2,X)$ im Polynomring $\Z(X)$}
% TODO: Inhalt ergänzen (Tex)
\end{EXA}

\begin{LEM}{A-T12-05-14}{Vereinigung von aufsteigend inkludierten Idealen sind Ideale}
% TODO: Inhalt ergänzen (Tex)
\end{LEM}

\begin{PROP}{A-T12-05-15}{Faktorring als Quotientenring bzgl, Ideale}
% TODO: Inhalt ergänzen (Tex)
\end{PROP}

\begin{KORO}{A-T12-05-16}{Jedes Ideal ist Kern eines geeigneten R-Homs}
% TODO: Inhalt ergänzen (Tex)
\end{KORO}

\begin{DEF}{A-T12-05-17}{Quotienten für Ideale in Ringen mit Quotientenabbildung}
% TODO: Inhalt ergänzen (Tex)
\end{DEF}

\begin{PROP}{A-T12-05-18}{Faktorringe für Ideale mit kanonischer Projektion sind Quotienten}
% TODO: Inhalt ergänzen (Tex)
\end{PROP}

\begin{REM}{A-T12-05-19}{Quotientenabbildung ist surjektiv}
% TODO: Inhalt ergänzen (Tex)
\end{REM}

\begin{PROP}{A-T12-05-20}{Urbild von Idealen längs R-Homs ist Ideal}
% TODO: Inhalt ergänzen (Tex)
\end{PROP}

\begin{PROP}{A-T12-05-21}{Bilder von Idealen längs surjektiven R-Homs sind Ideale}
% TODO: Inhalt ergänzen (Tex)
\end{PROP}

\begin{PROP}{A-T12-05-22}{Homomorphisatz}
% TODO: Inhalt ergänzen (Tex)
\end{PROP}

\begin{REM}{A-T12-05-23}{Struktur von Faktorring bestimmen durch raten eines Isomorphismus zw S und $R\backslash I$ und $I=ker(f)$}
% TODO: Inhalt ergänzen (Tex)
\end{REM}

\begin{EXA}{A-T12-05-24}{Komplexen Zahlen isomorph zu Faktorring des Polynomringes in reellen Zahlen Mod $(X^2+1)$}
% TODO: Inhalt ergänzen (Tex)
\end{EXA}

\begin{PROP}{A-T12-05-25}{Erster Isomorphiesatz}
% TODO: Inhalt ergänzen (Tex)
\end{PROP}

\begin{PROP}{A-T12-05-26}{Zweiter Isomorphiesatz}
% TODO: Inhalt ergänzen (Tex)
\end{PROP}

\begin{DEF}{A-T12-07-01}{Modul zu einem Ring}
% TODO: Inhalt ergänzen (Tex)
\end{DEF}

\begin{DEF}{A-T12-07-02}{R-Modulhomomorphismus}
% TODO: Inhalt ergänzen (Tex)
\end{DEF}

\begin{DEF}{A-T12-07-03}{Untermodul eines Moduls}
% TODO: Inhalt ergänzen (Tex)
\end{DEF}

\begin{DEF}{A-T12-07-04}{Durch Teilmengen eines R-Moduls erzeugte Untermoduln}
% TODO: Inhalt ergänzen (Tex)
\end{DEF}

\begin{DEF}{A-T12-07-05}{Kern und Bild eines R-Modulhomomorphismus}
% TODO: Inhalt ergänzen (Tex)
\end{DEF}

\begin{DEF}{A-T12-07-06}{Innere Direkte Summe von Untermoduln}
% TODO: Inhalt ergänzen (Tex)
\end{DEF}

\begin{DEF}{A-T12-07-07}{Direkte Summe von Moduln}
% TODO: Inhalt ergänzen (Tex)
\end{DEF}

\begin{DEF}{A-T12-07-08}{Direkte Produkt von Moduln}
% TODO: Inhalt ergänzen (Tex)
\end{DEF}

\begin{DEF}{A-T12-07-09}{Annulatorideal von R-Moduln}
% TODO: Inhalt ergänzen (Tex)
\end{DEF}

\begin{DEF}{A-T12-07-10}{Zyklischer R-Modul}
% TODO: Inhalt ergänzen (Tex)
\end{DEF}

\begin{EXA}{A-T12-07-11}{K-Vektorräume sind K-Moduln}
% TODO: Inhalt ergänzen (Tex)
\end{EXA}

\begin{EXA}{A-T12-07-12}{Abelsche Gruppen sind $\Z$-Moduln}
% TODO: Inhalt ergänzen (Tex)
\end{EXA}

\begin{EXA}{A-T12-07-13}{Moduln bzgl Polynomringe in Körpern sind ein K-Vektorraum mit einem K-linearen Endo}
% TODO: Inhalt ergänzen (Tex)
\end{EXA}

\begin{EXA}{A-T12-07-14}{Menge der Spaltentupel mit Elementen aus einem Ring ist mit komp. Add und diagonale R-Multip ein R-Moduln}
% TODO: Inhalt ergänzen (Tex)
\end{EXA}

\begin{EXA}{A-T12-07-15}{Für Körper sind K-Modulnhomomorphismen K-lineare Abbildungen}
% TODO: Inhalt ergänzen (Tex)
\end{EXA}

\begin{EXA}{A-T12-07-16}{Für Z sind Z-Modulnhomomorphismen Gruppenhomomorphismen}
% TODO: Inhalt ergänzen (Tex)
\end{EXA}

\begin{EXA}{A-T12-07-17}{Für Polynomringe in Körpern sind Modulnhomomorphismen K-lineare Abbildungen, die mit $X*$ Polynom kommutieren}
% TODO: Inhalt ergänzen (Tex)
\end{EXA}

\begin{EXA}{A-T12-07-18}{Freie Moduln vom Rang n als isomorphe R-Moduln zu $R^n$}
% TODO: Inhalt ergänzen (Tex)
\end{EXA}

\begin{EXA}{A-T12-07-19}{Für Körper sind Untermoduln Untervektorräume}
% TODO: Inhalt ergänzen (Tex)
\end{EXA}

\begin{EXA}{A-T12-07-20}{Für $\Z$ sind Untermoduln Untergruppen}
% TODO: Inhalt ergänzen (Tex)
\end{EXA}

\begin{EXA}{A-T12-07-21}{Für Polynomringe in Körpern sind Untermoduln Endo-Stabile Untervektorräume}
% TODO: Inhalt ergänzen (Tex)
\end{EXA}

\begin{EXA}{A-T12-07-22}{Untermoduln eines Ringes sind Ideale}
% TODO: Inhalt ergänzen (Tex)
\end{EXA}

\begin{EXA}{A-T12-07-23}{R-Modulhomomorphismus von Koeffizienten aus $R^n$ in M für fixiertes Elemente-Tupel in R-Moduln: Surjektiv gdw Endlich erzeugt}
% TODO: Inhalt ergänzen (Tex)
\end{EXA}

\begin{PROP}{A-T15-03-01}{Ringhomomorphismen bilden Einheiten auf Einheiten ab und induzieren G-Hom auf Einheitsgruppen}
% TODO: Inhalt ergänzen (Tex)
\end{PROP}

\begin{DEF}{A-T15-03-02}{Schiefkörper als Ring mit Einheitsgruppe $= R$ ohne $0$}
% TODO: Inhalt ergänzen (Tex)
\end{DEF}

\begin{DEF}{A-T15-03-03}{Körper als abelscher Schiefkörper}
% TODO: Inhalt ergänzen (Tex)
\end{DEF}

\begin{EXA}{A-T15-03-04}{Quaternionen als nichtkommutativer Schiefkörper}
% TODO: Inhalt ergänzen (Tex)
\end{EXA}

\begin{DEF}{A-T15-04-01}{Potenzreihenring mit Koeffizienten in Ring}
% TODO: Inhalt ergänzen (Tex)
\end{DEF}

\begin{DEF}{A-T15-04-02}{Polynomring mit Koeffizienten in Ring als Unterring von Potenzreihenring}
% TODO: Inhalt ergänzen (Tex)
\end{DEF}

\begin{CONC}{CM-T12-01-01}{2-Körper Problem mit konservativen Zentralkräften}
% TODO: Inhalt ergänzen (Tex)
\end{CONC}

\begin{CONC}{CM-T12-01-02}{Effektive Einkörper Problem}
% TODO: Inhalt ergänzen (Tex)
\end{CONC}

\begin{CONC}{CM-T12-02-01}{Aufbau Streuexperiment}
% TODO: Inhalt ergänzen (Tex)
\end{CONC}

\begin{DEF}{CM-T12-02-02}{Beschreibung vor Streuung:
-Geschwindigkeit und Energie der Teilchen
- Homo Teilchenstromdichte des Teilchenstrahls
- Stoßparameter}
% TODO: Inhalt ergänzen (Tex)
\end{DEF}

\begin{DEF}{CM-T12-02-03}{Beschreibung nach Streuung:
- Raumwinkel u Zählrate (Detektor)
- Zählrate als Teilchenstrom
- Differenzieller Wirkungsquerschnitt
- Totaler Wirkungsquerschnitt}
% TODO: Inhalt ergänzen (Tex)
\end{DEF}

\begin{CONC}{CM-T12-02-04}{Differenzieller und totaler Wirkungsquerschnitt}
% TODO: Inhalt ergänzen (Tex)
\end{CONC}

\begin{CONC}{CM-T12-02-05}{Gesamtzahl gestreuter Teilchen $\backslash$ Zeiteinh $=$ tot WQS $*$ Teilchenstromdichte}
% TODO: Inhalt ergänzen (Tex)
\end{CONC}

\begin{CONC}{CM-T12-02-06}{Totale WQS als effektive Querschnittfläche, die das Potential der Projektile bietet}
% TODO: Inhalt ergänzen (Tex)
\end{CONC}

\begin{EXA}{CM-T12-02-07}{Streunung an harter Kugel}
% TODO: Inhalt ergänzen (Tex)
\end{EXA}

\begin{REM}{CM-T12-02-08}{Definitionen unabhängig von Klassische / QM}
% TODO: Inhalt ergänzen (Tex)
\end{REM}

\begin{REM}{CM-T12-03-01}{Berechnung des WQS unter Annahmen:
1) Streuung im Rahmen der CM beschreibbar
2) Elastische Streuung (kein Ener.Austausch zw. Projektil u Target
3) Streuung am Zentralpotential (Zw. Proj und Targ wirkende Potential hängt nur von Betrag Abstand ab}
% TODO: Inhalt ergänzen (Tex)
\end{REM}

\begin{EXA}{CM-T12-03-02}{Trajektorie des Projektils für repulsives Potential}
% TODO: Inhalt ergänzen (Tex)
\end{EXA}

\begin{CONC}{CM-T12-03-03}{Zusammenhang zwischen Streuwinkel und Restwinkel $/phi_inf$}
% TODO: Inhalt ergänzen (Tex)
\end{CONC}

\begin{CONC}{CM-T12-03-04}{Streuwinkel und differentieller WQS hängen für ein Zentralpotential nur von von Streuparameter und Energie ab}
% TODO: Inhalt ergänzen (Tex)
\end{CONC}

\begin{PROP}{CM-T12-03-05}{Darstellung des Streuparameters als Funktion von Streuwinkel und Energie}
% TODO: Inhalt ergänzen (Tex)
\end{PROP}

\begin{PROP}{CM-T12-03-06}{Darstellung des Drehimpulses durch Streuparameter und Energie (Erhaltungssätze)}
% TODO: Inhalt ergänzen (Tex)
\end{PROP}

\begin{CONC}{CM-T12-03-07}{Bilanzgleichung für Streuung}
% TODO: Inhalt ergänzen (Tex)
\end{CONC}

\begin{PROP}{CM-T12-03-08}{Darstellung von differentielle WQS durch Streuparameter und Energie}
% TODO: Inhalt ergänzen (Tex)
\end{PROP}

\begin{EXA}{CM-T12-03-09}{Rutherfordscher Wirkungsquerschnitt}
% TODO: Inhalt ergänzen (Tex)
\end{EXA}

\begin{REM}{CM-T12-03-10}{Anmerkungen}
% TODO: Inhalt ergänzen (Tex)
\end{REM}

\begin{CONC}{CM-T12-04-02}{Zusammenhang zwischen Streuwinkel in effektivem 1KP und Streuwinkel des Projektils im 2KP}
% TODO: Inhalt ergänzen (Tex)
\end{CONC}

\begin{CONC}{CM-T12-04-03}{Messung des Wirkungsquerschnitts vom Laborsystem aus: Experimentelle Größe ist Streuwinkel des Projektils}
% TODO: Inhalt ergänzen (Tex)
\end{CONC}

\begin{CONC}{CM-T12-04-04}{Streuprozess im CM-System}
% TODO: Inhalt ergänzen (Tex)
\end{CONC}

\begin{PROP}{CM-T12-04-05}{Herleitung des Zusammenhangs zwischen Streuwinkel des Projektils im Laborsystem und Streuwinkel im effektiven 1KP}
% TODO: Inhalt ergänzen (Tex)
\end{PROP}

\begin{CONC}{CM-T12-04-06}{Formel für Beziehung zwischen Streuwinkel des Projektils in Laborsystem und Streuwinkel in eff. 1KP}
% TODO: Inhalt ergänzen (Tex)
\end{CONC}

\begin{CONC}{CM-T12-04-07}{Grenzfälle der Beziehung zwischen Streuwinkel}
% TODO: Inhalt ergänzen (Tex)
\end{CONC}

\begin{PROP}{CM-T12-04-08}{Umrechnung von differentiellem Wirkungsquerschnitt im CMS in Laborsystem}
% TODO: Inhalt ergänzen (Tex)
\end{PROP}

\begin{CONC}{CM-T12-04-09}{Formel für Beziehung zwsichen diff. WQS im CMS und Laborsystem}
% TODO: Inhalt ergänzen (Tex)
\end{CONC}

\begin{CONC}{CM-T12-04-10}{Grenzfälle der Beziehung zwischen diff. WQS}
% TODO: Inhalt ergänzen (Tex)
\end{CONC}

\begin{CONC}{CM-T12-05-01}{2-Körper Problem mit konservativen Zentralkräften}
% TODO: Inhalt ergänzen (Tex)
\end{CONC}

\begin{CONC}{CM-T12-05-02}{Reduktion of effektives 1-Körper Problem}
% TODO: Inhalt ergänzen (Tex)
\end{CONC}

\begin{DEF}{CM-T12-05-03}{Gesamtenergie in reduzierte 1-Körper Problem}
% TODO: Inhalt ergänzen (Tex)
\end{DEF}

\begin{DEF}{CM-T12-05-04}{Explizite Formel für Bahnkurve on Polarkoordianten}
% TODO: Inhalt ergänzen (Tex)
\end{DEF}

\begin{CONC}{CM-T12-05-05}{Qualitative Beschreibung der Bewegung durch Graph des effektiven Potentials}
% TODO: Inhalt ergänzen (Tex)
\end{CONC}

\begin{CONC}{CM-T12-05-06}{Streuung als ungebundene Bewegung im 2KP}
% TODO: Inhalt ergänzen (Tex)
\end{CONC}

\begin{REM}{CM-T12-07-01}{Atwood-Pendel}
% TODO: Inhalt ergänzen (Tex)
\end{REM}

\begin{REM}{CM-T12-07-02}{Totale Wirkungsquerschnitt für feste Kugel}
% TODO: Inhalt ergänzen (Tex)
\end{REM}

\begin{REM}{CM-T12-07-03}{Relation $L = \frac{s}{2\mu E}$ in 2-Körper Systemen}
% TODO: Inhalt ergänzen (Tex)
\end{REM}

\begin{REM}{CM-T12-07-04}{Allgemeine Herleitung des tot. WQS in reduzierten 1-Körper Systemen}
% TODO: Inhalt ergänzen (Tex)
\end{REM}

\begin{DEF}{EFT1-T12-02-01}{Lokale triviale Faserung mit typischen Fasern auf Mfk}
% TODO: Inhalt ergänzen (Tex)
\end{DEF}

\begin{DEF}{EFT1-T12-02-02}{Vektorraumbündel}
% TODO: Inhalt ergänzen (Tex)
\end{DEF}

\begin{EXA}{EFT1-T12-02-03}{Projektion von Kreuzprodukt ist eine lokal triviale Faserung}
% TODO: Inhalt ergänzen (Tex)
\end{EXA}

\begin{EXA}{EFT1-T12-02-04}{Tangentialbündel mit differenzierbarer Struktur ist Vektorraumbündel}
% TODO: Inhalt ergänzen (Tex)
\end{EXA}

\begin{EXA}{EFT1-T12-02-05}{Vektorraumbündel zu $S^1$}
% TODO: Inhalt ergänzen (Tex)
\end{EXA}

\begin{EXA}{EFT1-T12-02-06}{Lokale triviale Faserung über $S^1$}
% TODO: Inhalt ergänzen (Tex)
\end{EXA}

\begin{DEF}{EFT1-T12-02-07}{Lokale triviale Faserung als Tripel von Totalraum, Basisraum, Bündelprojektion mit typischen Fasern}
% TODO: Inhalt ergänzen (Tex)
\end{DEF}

\begin{DEF}{EFT1-T12-02-08}{Reale Fasern in lokal trivialen Faserungen}
% TODO: Inhalt ergänzen (Tex)
\end{DEF}

\begin{DEF}{EFT1-T12-02-09}{Bündelkarten für offene Teilmengen der Basis}
% TODO: Inhalt ergänzen (Tex)
\end{DEF}

\begin{DEF}{EFT1-T12-02-10}{Bündelatlas für lokale triviale Faserungen}
% TODO: Inhalt ergänzen (Tex)
\end{DEF}

\begin{DEF}{EFT1-T12-02-11}{Faserkarte am Punkt x im Basisraum}
% TODO: Inhalt ergänzen (Tex)
\end{DEF}

\begin{DEF}{EFT1-T12-02-12}{Bündelkartenwechsel zwischen Bündelkarten}
% TODO: Inhalt ergänzen (Tex)
\end{DEF}

\begin{DEF}{EFT1-T12-02-13}{G-Faserbündel mit Liegruppen als Strukturgruppen}
% TODO: Inhalt ergänzen (Tex)
\end{DEF}

\begin{DEF}{EFT1-T12-02-14}{Prinzipalbüdel / Hauptfaserbündel}
% TODO: Inhalt ergänzen (Tex)
\end{DEF}

\begin{REM}{EFT1-T12-02-15}{Beziehung zwischen Vektorraumbündeln und GL-Faserbündeln}
% TODO: Inhalt ergänzen (Tex)
\end{REM}

\begin{DEF}{EFT1-T12-02-16}{(Differenzierbare) (Lokale) Schnitte in lokal trivialen Faserungen}
% TODO: Inhalt ergänzen (Tex)
\end{DEF}

\begin{DEF}{EFT1-T12-02-17}{Raum der differenzierbaren lokalen Schnitte}
% TODO: Inhalt ergänzen (Tex)
\end{DEF}

\begin{EXA}{EFT1-T12-02-18}{Raum der diff. lokalen Schnitte in Kreuzprodukten}
% TODO: Inhalt ergänzen (Tex)
\end{EXA}

\begin{EXA}{EFT1-T12-02-19}{Raum der diff. Lokalen Schnitte im Tangentialbündel}
% TODO: Inhalt ergänzen (Tex)
\end{EXA}

\begin{EXA}{EFT1-T12-02-20}{Jedes Vektorraumbündel hat einen lokalen Schnitt $x$ auf $O_x$ in $E_x$}
% TODO: Inhalt ergänzen (Tex)
\end{EXA}

\begin{EXA}{EFT1-T12-02-21}{Im Tangentialbündel existiert kein diff. Schnitt, der nirgends verschwindet}
% TODO: Inhalt ergänzen (Tex)
\end{EXA}

\begin{EXA}{EFT1-T12-02-22}{$S^1$ auf $S^1$, $z$ auf $z^2$ gibt es keinen Schnitt}
% TODO: Inhalt ergänzen (Tex)
\end{EXA}

\begin{REM}{EFT1-T12-02-23}{Raum der diff Schnitte in Vektorraumbündeln ist der Vektorraum von glatten Abbildungen auf M}
% TODO: Inhalt ergänzen (Tex)
\end{REM}

\begin{REM}{EFT1-T12-02-24}{Für Bündelkarten in Vektorraumbündeln existieren k lokale Schnitte, die an jeder Stelle eine Basis der realen Faser bilden}
% TODO: Inhalt ergänzen (Tex)
\end{REM}

\begin{REM}{EFT1-T12-02-25}{k lokale Schnitte, die bei Punkt eine Basis der Faser bilden, induzieren eine Bündelkarte}
% TODO: Inhalt ergänzen (Tex)
\end{REM}

\begin{REM}{EFT1-T12-02-26}{Bündelkarten in G-Prinzipalbündeln induzieren lokale Schnitte}
% TODO: Inhalt ergänzen (Tex)
\end{REM}

\begin{REM}{EFT1-T12-02-27}{Präbündel mit Strukturgruppe G zu Liegruppe G, Mfk, (disj) Vereinigung von punktweise Mfk und Projektion}
% TODO: Inhalt ergänzen (Tex)
\end{REM}

\begin{PROP}{EFT1-T12-02-28}{Für Präbündel $(E,\pi,M)$ existiert auf E genau einem Topologie und differenzierbare Struktur, sodass $(E,\pi,M)$ ein Faserbündel mit Strukturgruppe G wird und Präbündelkarten Bündelkarten werden}
% TODO: Inhalt ergänzen (Tex)
\end{PROP}

\begin{EXA}{EFT1-T12-02-29}{Bündelstruktur von Tangentialbündel als Ergebnis der Konstruktion von Präbündeln}
% TODO: Inhalt ergänzen (Tex)
\end{EXA}

\begin{EXA}{EFT1-T12-02-30}{Präbündel zum GL-Prinzipalbündel}
% TODO: Inhalt ergänzen (Tex)
\end{EXA}

\begin{EXA}{EFT1-T12-02-31}{Präbündel zum $O(n)$-Prinzipalbündel für Riemannische Mfk}
% TODO: Inhalt ergänzen (Tex)
\end{EXA}

\begin{KORO}{EFT1-T12-02-32}{Direkte Summe von Vektorraumbündeln ergeben Prävektorraumbündel}
% TODO: Inhalt ergänzen (Tex)
\end{KORO}

\begin{EXA}{EFT1-T12-02-33}{Hom-Raum für Homomorphismen zwischen Vektorraumbündeln sind Vektorraumbündel}
% TODO: Inhalt ergänzen (Tex)
\end{EXA}

\begin{EXA}{EFT1-T12-02-34}{Mult}
% TODO: Inhalt ergänzen (Tex)
\end{EXA}

\begin{EXA}{EFT1-T12-02-35}{Sym}
% TODO: Inhalt ergänzen (Tex)
\end{EXA}

\begin{EXA}{EFT1-T12-02-36}{Alt}
% TODO: Inhalt ergänzen (Tex)
\end{EXA}

\begin{DEF}{EFT1-T12-02-37}{Bündelmetrik auf Totalraum ist ein Schnitt in $Sym^2(E)$, sodass g pw. positiv definit}
% TODO: Inhalt ergänzen (Tex)
\end{DEF}

\begin{EXA}{EFT1-T12-02-38}{Riemannische Metrik als Bündelmetrik im Tangentialbündel}
% TODO: Inhalt ergänzen (Tex)
\end{EXA}

\begin{EXA}{EFT1-T12-02-39}{$\Gamma (Alt^k(TM))$}
% TODO: Inhalt ergänzen (Tex)
\end{EXA}

\begin{DEF}{EFT1-T12-02-40}{Vektorraumbündel vom endlichen Typ}
% TODO: Inhalt ergänzen (Tex)
\end{DEF}

\begin{EXA}{EFT1-T12-02-41}{Tangentialbündel von $S^n$ ist von endlichem Typ}
% TODO: Inhalt ergänzen (Tex)
\end{EXA}

\begin{DEF}{EFT1-T12-02-43}{Bündelisomorphismus}
% TODO: Inhalt ergänzen (Tex)
\end{DEF}

\begin{DEF}{EFT1-T12-02-44}{Trivialisierung von Totalraum}
% TODO: Inhalt ergänzen (Tex)
\end{DEF}

\begin{DEF}{EFT1-T12-02-45}{Vektorraumbündelabbildung über diff. Abbildungen zwischen Vektorraumbündeln}
% TODO: Inhalt ergänzen (Tex)
\end{DEF}

\begin{DEF}{EFT1-T12-02-46}{Vektorraumbündelisomorphismus}
% TODO: Inhalt ergänzen (Tex)
\end{DEF}

\begin{EXA}{EFT1-T12-02-47}{Differential von glatten Abbildungen zw. Tangentialbündel von Mfk ist eine Vektorraumbündelabbildung über glatte Abbildung $f$}
% TODO: Inhalt ergänzen (Tex)
\end{EXA}

\begin{DEF}{EFT1-T12-02-48}{Induzierte Bündel durch Abbildungen}
% TODO: Inhalt ergänzen (Tex)
\end{DEF}

\begin{PROP}{EFT1-T12-02-49}{Schnitte in induzierten Bündeln längs $f$}
% TODO: Inhalt ergänzen (Tex)
\end{PROP}

\begin{EXA}{EFT1-T12-02-50}{Menge der Vektorfelder längs Kurven}
% TODO: Inhalt ergänzen (Tex)
\end{EXA}

\begin{EXA}{EFT1-T12-02-51}{Vektorraumbündel bzgl Grassmann-Mfk}
% TODO: Inhalt ergänzen (Tex)
\end{EXA}

\begin{REM}{EFT1-T12-02-52}{Bündelabbildungen bzgl induzierte Bündel}
% TODO: Inhalt ergänzen (Tex)
\end{REM}

\begin{KORO}{EFT1-T12-02-53}{Homotope Abbildungen in Faserbündel induzieren isomorphe Bündel}
% TODO: Inhalt ergänzen (Tex)
\end{KORO}

\begin{DEF}{EFT1-T12-02-54}{Induzierte Bündel bei Einbettungen von UnterMfk}
% TODO: Inhalt ergänzen (Tex)
\end{DEF}

\begin{DEF}{EFT1-T12-02-55}{Untervektorraumbündel}
% TODO: Inhalt ergänzen (Tex)
\end{DEF}

\begin{REM}{EFT1-T12-02-56}{Untervektorraumbündel sind Vektorraumbündel}
% TODO: Inhalt ergänzen (Tex)
\end{REM}

\begin{REM}{EFT1-T12-02-57}{Quotienten-Räume bzgl Untervektorraumbündel sind Vektorraumbündel}
% TODO: Inhalt ergänzen (Tex)
\end{REM}

\begin{REM}{EFT1-T12-02-58}{Untervektorraumbündel bzgl Bündelmetrik}
% TODO: Inhalt ergänzen (Tex)
\end{REM}

\begin{REM}{EFT1-T12-02-59}{Tangentialbündel von UnterMfk sind Untervektorraumbündel}
% TODO: Inhalt ergänzen (Tex)
\end{REM}

\begin{REM}{EFT1-T12-02-60}{Normalenbündel von UnterMfk}
% TODO: Inhalt ergänzen (Tex)
\end{REM}

\begin{PROP}{EFT1-T12-02-61}{Rang-Satz für Vektorraumhomomorphismen: Konstanter Rang impliziert ker und im sind Untervektorraumbündel}
% TODO: Inhalt ergänzen (Tex)
\end{PROP}

\begin{KORO}{EFT1-T12-02-62}{Charakterisierung von Vektorraumbündeln von endlichem Typ}
% TODO: Inhalt ergänzen (Tex)
\end{KORO}

\begin{DEF}{EFT1-T12-02-63}{Reduktionen von Faserbündeln mit Strukturgruppe bzgl abgeschlossener Untergruppe}
% TODO: Inhalt ergänzen (Tex)
\end{DEF}

\begin{EXA}{EFT1-T12-02-64}{Charakterisierung von orientierten Mfk}
% TODO: Inhalt ergänzen (Tex)
\end{EXA}

\begin{PROP}{EFT1-T12-02-65}{Ehresmannscher Faserungssatz: Totalräume mit eigentlich regulären Abbildungen in zusammenhängenden Basisraum implizieren eine lokale triviale Faserung}
% TODO: Inhalt ergänzen (Tex)
\end{PROP}

\begin{DEF}{EFT1-T12-02-66}{TEST 4}
% TODO: Inhalt ergänzen (Tex)
\end{DEF}

\begin{DEF}{EFT1-T12-02-67}{TEST 5}
% TODO: Inhalt ergänzen (Tex)
\end{DEF}

\begin{DEF}{EFT1-T12-02-68}{TEST 6}
% TODO: Inhalt ergänzen (Tex)
\end{DEF}

\begin{DEF}{GPDE-T12-02-01}{Allgemeine Differentialoperator}
% TODO: Inhalt ergänzen (Tex)
\end{DEF}

\begin{DEF}{GPDE-T12-02-02}{Affin linear in k-ter Komponente}
% TODO: Inhalt ergänzen (Tex)
\end{DEF}

\begin{DEF}{GPDE-T12-02-03}{Lineare Operator k-ter Ordnung}
% TODO: Inhalt ergänzen (Tex)
\end{DEF}

\begin{DEF}{GPDE-T12-02-04}{Semilineare Operator k-ter Ordnung}
% TODO: Inhalt ergänzen (Tex)
\end{DEF}

\begin{DEF}{GPDE-T12-02-05}{Quasilineare Operator k-ter Ordnung}
% TODO: Inhalt ergänzen (Tex)
\end{DEF}

\begin{DEF}{GPDE-T12-02-06}{Voll nichtlineare Operator k-ter Ordnung}
% TODO: Inhalt ergänzen (Tex)
\end{DEF}

\begin{DEF}{GPDE-T12-02-07}{Lineare Operator 2-ter Ordnung}
% TODO: Inhalt ergänzen (Tex)
\end{DEF}

\begin{DEF}{GPDE-T12-02-08}{Elliptischer linearer Operator 3-ter Ordnung}
% TODO: Inhalt ergänzen (Tex)
\end{DEF}

\begin{DEF}{GPDE-T12-02-09}{Parabolischer lineaer Operator 2-ter Ordnung}
% TODO: Inhalt ergänzen (Tex)
\end{DEF}

\begin{DEF}{GPDE-T12-02-10}{Strikt linearer Operator 2-ter Ordnung}
% TODO: Inhalt ergänzen (Tex)
\end{DEF}

\begin{DEF}{GPDE-T12-02-11}{Uniform linearer Operator 2-ter Ordnung}
% TODO: Inhalt ergänzen (Tex)
\end{DEF}

\begin{DEF}{GPDE-T12-02-12}{Partieller Differentialoperator 2. Ordung}
% TODO: Inhalt ergänzen (Tex)
\end{DEF}

\begin{DEF}{GPDE-T12-02-13}{Elliptischer partieller Differentialoperator 2. Ordnung}
% TODO: Inhalt ergänzen (Tex)
\end{DEF}

\begin{DEF}{GPDE-T12-02-14}{Strikt elliptischer Differentialoperator 2. Ordnung}
% TODO: Inhalt ergänzen (Tex)
\end{DEF}

\begin{DEF}{GPDE-T12-02-15}{Uniform elliptischer Differentialoperator 2. Ordnung}
% TODO: Inhalt ergänzen (Tex)
\end{DEF}

\begin{EXA}{GPDE-T12-02-16}{Lineare elliptische Differentialoperatoren}
% TODO: Inhalt ergänzen (Tex)
\end{EXA}

\begin{EXA}{GPDE-T12-02-17}{Korrsp. Operator der Monge-Ampère Gleichung ist elliptisch für strikt konvexe Funktionen}
% TODO: Inhalt ergänzen (Tex)
\end{EXA}

\begin{EXA}{GPDE-T12-02-18}{Korrespondierender Operator zur Mittlere Krümmungs Gleichung ist uniform elliptisch}
% TODO: Inhalt ergänzen (Tex)
\end{EXA}

\begin{DEF}{GPDE-T12-02-19}{Parabolischer partieller Differentialoperator 2. Ordnung}
% TODO: Inhalt ergänzen (Tex)
\end{DEF}

\begin{DEF}{GPDE-T12-02-20}{Strikt parabolischer Differentialoperator 2. Ordnung}
% TODO: Inhalt ergänzen (Tex)
\end{DEF}

\begin{DEF}{GPDE-T12-02-21}{Uniform parabolischer Differentialoperator 2. Ordnung}
% TODO: Inhalt ergänzen (Tex)
\end{DEF}

\begin{EXA}{GPDE-T12-02-22}{Korresp. Operator zur Mittleren Krümmungsfluss Gleichung ist uniform parabolisch}
% TODO: Inhalt ergänzen (Tex)
\end{EXA}

\begin{CONC}{GPDE-T12-02-23}{Linearisierung von Differentialoperatoren}
% TODO: Inhalt ergänzen (Tex)
\end{CONC}

\begin{CONC}{GPDE-T12-02-24}{Definition des allg. Partiellen Differentialoperators 2. Ordnung}
% TODO: Inhalt ergänzen (Tex)
\end{CONC}

\begin{DEF}{GPDE-T12-03-01}{Lp-Raum}
% TODO: Inhalt ergänzen (Tex)
\end{DEF}

\begin{DEF}{GPDE-T12-03-02}{Lokal Hölder Stetig mit Exponent}
% TODO: Inhalt ergänzen (Tex)
\end{DEF}

\begin{DEF}{GPDE-T12-03-03}{Hölder Stetig mit Exponent}
% TODO: Inhalt ergänzen (Tex)
\end{DEF}

\begin{DEF}{GPDE-T12-03-04}{Lipschitz Stetig}
% TODO: Inhalt ergänzen (Tex)
\end{DEF}

\begin{DEF}{GPDE-T12-03-05}{Ck Differenzierbar mit (lokal) Hölder Stetigen k-ten Ableitungen}
% TODO: Inhalt ergänzen (Tex)
\end{DEF}

\begin{REM}{GPDE-T12-06-01}{Was ist ein schw. MP?}
% TODO: Inhalt ergänzen (Tex)
\end{REM}

\begin{DEF}{GPDE-T12-06-02}{Parabolischer Rand}
% TODO: Inhalt ergänzen (Tex)
\end{DEF}

\begin{THEO}{GPDE-T12-06-03}{Parabolisches schwaches Maximum Prinzip}
% TODO: Inhalt ergänzen (Tex)
\end{THEO}

\begin{KORO}{GPDE-T12-06-04}{Eindeutigkeit von Lösungen von parabolischen Differentialoperatoren}
% TODO: Inhalt ergänzen (Tex)
\end{KORO}

\begin{THEO}{GPDE-T12-06-05}{Elliptisches schwaches Maximumsprinzip}
% TODO: Inhalt ergänzen (Tex)
\end{THEO}

\begin{KORO}{GPDE-T12-06-06}{Eindeutigkeit von Lösungen von elliptischen Differentialoperatoren}
% TODO: Inhalt ergänzen (Tex)
\end{KORO}

\begin{REM}{GPDE-T12-07-01}{Motivation Starke Maximum Prinzipien}
% TODO: Inhalt ergänzen (Tex)
\end{REM}

\begin{LEM}{GPDE-T12-07-02}{Propagation von Positivität}
% TODO: Inhalt ergänzen (Tex)
\end{LEM}

\begin{THEO}{GPDE-T12-07-03}{Parabolisches starkes Maximum Prinzip}
% TODO: Inhalt ergänzen (Tex)
\end{THEO}

\begin{THEO}{GPDE-T12-07-04}{Elliptisches starkes Maximum Prinzip}
% TODO: Inhalt ergänzen (Tex)
\end{THEO}

\begin{LEM}{GPDE-T12-08-01}{Parabolisches Hopf-Lemma für Rand-Punkte}
% TODO: Inhalt ergänzen (Tex)
\end{LEM}

\begin{LEM}{GPDE-T12-08-02}{Elliptisches Hopf-Lemma für Rand-Punkte}
% TODO: Inhalt ergänzen (Tex)
\end{LEM}

\begin{DEF}{GPDE-T12-08-03}{Interior Ball Kondition an offene Mengen}
% TODO: Inhalt ergänzen (Tex)
\end{DEF}

\begin{KORO}{GPDE-T12-08-04}{Eindeutigkeit für Neumann Probleme für elliptische Operatoren auf Mengen mit Interior Ball Kondition}
% TODO: Inhalt ergänzen (Tex)
\end{KORO}

\begin{THEO}{GPDE-T12-09-01}{Elliptisches Vergleichsprinzip}
% TODO: Inhalt ergänzen (Tex)
\end{THEO}

\begin{THEO}{GPDE-T12-09-02}{Parabolisches Vergleichsprinzip}
% TODO: Inhalt ergänzen (Tex)
\end{THEO}

\begin{DEF}{HA-T12-01-01}{Kategorie}
% TODO: Inhalt ergänzen (Tex)
\end{DEF}

\begin{EXA}{HA-T12-01-02}{Exa Kategorien}
% TODO: Inhalt ergänzen (Tex)
\end{EXA}

\begin{DEF}{HA-T12-01-03}{Unterkategorie}
% TODO: Inhalt ergänzen (Tex)
\end{DEF}

\begin{DEF}{HA-T12-01-04}{Volle Unterkategorie}
% TODO: Inhalt ergänzen (Tex)
\end{DEF}

\begin{EXA}{HA-T12-01-05}{Exa Volle Unterkategorie}
% TODO: Inhalt ergänzen (Tex)
\end{EXA}

\begin{DEF}{HA-T12-01-06}{Funktor}
% TODO: Inhalt ergänzen (Tex)
\end{DEF}

\begin{EXA}{HA-T12-01-07}{Hom-Funktor}
% TODO: Inhalt ergänzen (Tex)
\end{EXA}

\begin{DEF}{HA-T12-01-08}{Sequenz in Kategorie}
% TODO: Inhalt ergänzen (Tex)
\end{DEF}

\begin{DEF}{HA-T12-01-09}{Diagramm in Katagorie}
% TODO: Inhalt ergänzen (Tex)
\end{DEF}

\begin{DEF}{HA-T12-01-10}{Weg in Kategorie}
% TODO: Inhalt ergänzen (Tex)
\end{DEF}

\begin{DEF}{HA-T12-01-11}{Gelabelter Weg}
% TODO: Inhalt ergänzen (Tex)
\end{DEF}

\begin{DEF}{HA-T12-01-12}{Einfacher Weg}
% TODO: Inhalt ergänzen (Tex)
\end{DEF}

\begin{DEF}{HA-T12-01-13}{Kommutatives Diagramm}
% TODO: Inhalt ergänzen (Tex)
\end{DEF}

\begin{DEF}{HA-T12-01-14}{Kontravariante Funktoren}
% TODO: Inhalt ergänzen (Tex)
\end{DEF}

\begin{DEF}{HA-T12-01-15}{Isopmorphie in Kategorie}
% TODO: Inhalt ergänzen (Tex)
\end{DEF}

\begin{DEF}{HA-T12-01-16}{Natürliche Transformation}
% TODO: Inhalt ergänzen (Tex)
\end{DEF}

\begin{DEF}{HA-T12-01-17}{Natürlicher Isomorphismus}
% TODO: Inhalt ergänzen (Tex)
\end{DEF}

\begin{DEF}{HA-T12-01-18}{Komposition von Natürlichen Transformationen}
% TODO: Inhalt ergänzen (Tex)
\end{DEF}

\begin{DEF}{HA-T12-01-19}{(Kleine) Funktorenkategorie}
% TODO: Inhalt ergänzen (Tex)
\end{DEF}

\begin{EXA}{HA-T12-01-20}{Exa Kleine Funktorenkategorie: Diag / Seq}
% TODO: Inhalt ergänzen (Tex)
\end{EXA}

\begin{DEF}{HA-T12-01-21}{Komplex und Ketten-Abbildung}
% TODO: Inhalt ergänzen (Tex)
\end{DEF}

\begin{STUD}{HA-T12-01-22}{3 Grundbegriffe: Kat, Funk, naat Trafo}
% TODO: Inhalt ergänzen (Tex)
\end{STUD}

\begin{STUD}{HA-T12-01-23}{Von (kleinen) Funktorkategorien zur Kategorie der Komplexe in der Kategroei der Abelschen Gruppen}
% TODO: Inhalt ergänzen (Tex)
\end{STUD}

\begin{DEF}{HA-T12-02-01}{Links/Rechts Moduln}
% TODO: Inhalt ergänzen (Tex)
\end{DEF}

\begin{DEF}{HA-T12-02-02}{Abelsche Gruppe}
% TODO: Inhalt ergänzen (Tex)
\end{DEF}

\begin{DEF}{HA-T12-02-04}{Kern / Im / CoKern für R-Hom}
% TODO: Inhalt ergänzen (Tex)
\end{DEF}

\begin{THEO}{HA-T12-02-05}{ISO-Sätze}
% TODO: Inhalt ergänzen (Tex)
\end{THEO}

\begin{DEF}{HA-T12-02-06}{Freie R-Moduln}
% TODO: Inhalt ergänzen (Tex)
\end{DEF}

\begin{DEF}{HA-T12-02-07}{Freie Abelsche Gruppe}
% TODO: Inhalt ergänzen (Tex)
\end{DEF}

\begin{DEF}{HA-T12-04-01}{Additive Kategorie}
% TODO: Inhalt ergänzen (Tex)
\end{DEF}

\begin{DEF}{HA-T12-04-02}{Additiver Funktor von additiven Kategorien}
% TODO: Inhalt ergänzen (Tex)
\end{DEF}

\begin{DEF}{HA-T12-04-03}{Direkte Summe in additiven Kategorien}
% TODO: Inhalt ergänzen (Tex)
\end{DEF}

\begin{DEF}{HA-T12-04-04}{Monomorphismen in Kategorien}
% TODO: Inhalt ergänzen (Tex)
\end{DEF}

\begin{DEF}{HA-T12-04-05}{Epimorphismen in Kategorien}
% TODO: Inhalt ergänzen (Tex)
\end{DEF}

\begin{DEF}{HA-T12-04-06}{Monics / Epics in additiven Kategorien}
% TODO: Inhalt ergänzen (Tex)
\end{DEF}

\begin{DEF}{HA-T12-04-07}{Ker / Coker in additiven Kategorien}
% TODO: Inhalt ergänzen (Tex)
\end{DEF}

\begin{PROP}{HA-T12-04-08}{Beziehung Monic / Epic und ker / cokern in additiven Kategorien}
% TODO: Inhalt ergänzen (Tex)
\end{PROP}

\begin{DEF}{HA-T12-04-09}{Subgadget von Objekten in additiven Katgeorien}
% TODO: Inhalt ergänzen (Tex)
\end{DEF}

\begin{DEF}{HA-T12-04-10}{Quotienten-Objekt in additiven Kategorien}
% TODO: Inhalt ergänzen (Tex)
\end{DEF}

\begin{DEF}{HA-T12-04-11}{Abelsche Kategorie}
% TODO: Inhalt ergänzen (Tex)
\end{DEF}

\begin{EXA}{HA-T12-04-12}{Exa abelsche Kategorie: (Volle Unterkategorien von) Abelsche Gruppen}
% TODO: Inhalt ergänzen (Tex)
\end{EXA}

\begin{DEF}{HA-T12-04-13}{Exakte Kategorie}
% TODO: Inhalt ergänzen (Tex)
\end{DEF}

\begin{EXA}{HA-T12-04-14}{Exa Exakte Kategorie}
% TODO: Inhalt ergänzen (Tex)
\end{EXA}

\begin{REM}{HA-T12-04-15}{Exaktheit in abelschen Kategorien durch Subobjekt in Gadgete}
% TODO: Inhalt ergänzen (Tex)
\end{REM}

\begin{DEF}{HA-T12-04-16}{Abelsche Unterkategorie}
% TODO: Inhalt ergänzen (Tex)
\end{DEF}

\begin{PROP}{HA-T12-04-17}{Funktorkategorie zu abelschen Kategorie ist abelsch}
% TODO: Inhalt ergänzen (Tex)
\end{PROP}

\begin{DEF}{HA-T12-04-18}{Projektive Objekte in abelschen Kategorien}
% TODO: Inhalt ergänzen (Tex)
\end{DEF}

\begin{DEF}{HA-T12-04-19}{Injektive Objekte in abelschen Kategorien}
% TODO: Inhalt ergänzen (Tex)
\end{DEF}

\begin{STUD}{HA-T12-04-20}{Herleitung abelscher Kategorien und additiver Funktoren als allg. Rahmen für Komplexe in abelschen Kategorien}
% TODO: Inhalt ergänzen (Tex)
\end{STUD}

\begin{REM}{HA-T12-05-01}{Exakte Sequenzen sind Komplexe}
% TODO: Inhalt ergänzen (Tex)
\end{REM}

\begin{REM}{HA-T12-05-02}{Kurze Exakte Sequenzen zu Komplexen erweitern}
% TODO: Inhalt ergänzen (Tex)
\end{REM}

\begin{DEF}{HA-T12-05-03}{Sequenzen von Objekte}
% TODO: Inhalt ergänzen (Tex)
\end{DEF}

\begin{DEF}{HA-T12-05-04}{Positive / Negative Komplexe}
% TODO: Inhalt ergänzen (Tex)
\end{DEF}

\begin{DEF}{HA-T12-05-05}{Ketten / Zyklen / Ränder in Komplexen}
% TODO: Inhalt ergänzen (Tex)
\end{DEF}

\begin{DEF}{HA-T12-05-06}{n-te Homologie in Komplexen}
% TODO: Inhalt ergänzen (Tex)
\end{DEF}

\begin{REM}{HA-T12-05-07}{Homologie als Abweichung von Exaktheit eines Komplexes}
% TODO: Inhalt ergänzen (Tex)
\end{REM}

\begin{EXA}{HA-T12-05-08}{Fundamentale Exakte Sequenzen für Komplexe: Zyklen Ränder und Homologie}
% TODO: Inhalt ergänzen (Tex)
\end{EXA}

\begin{PROP}{HA-T12-05-09}{n-te Homologie ist additiver Funktor}
% TODO: Inhalt ergänzen (Tex)
\end{PROP}

\begin{REM}{HA-T12-05-10}{PROOF: n-te Homologie ist additiver Funktor}
% TODO: Inhalt ergänzen (Tex)
\end{REM}

\begin{THEO}{HA-T12-05-11}{Zu kurzen exakte Sequenz $(K,KAbb)$ in abel. Kategorie der Komplexe existiert ein Zusammenhangs-Homomorphismus}
% TODO: Inhalt ergänzen (Tex)
\end{THEO}

\begin{REM}{HA-T12-05-12}{PROOF: Zu kurzen exakte Sequenz $(K,KAbb)$ in abel. Kategorie der Komplexe existiert ein Zusammenhangs-Homomorphismus}
% TODO: Inhalt ergänzen (Tex)
\end{REM}

\begin{THEO}{HA-T12-05-13}{Kurze Exakte Sequenz in Kategorie der Komplexe induziert lange exakte Homologie-Sequenz}
% TODO: Inhalt ergänzen (Tex)
\end{THEO}

\begin{REM}{HA-T12-05-14}{PROOF: Kurze Exakte Sequenz in Kategorie der Komplexe induziert lange exakte Homologie-Sequenz}
% TODO: Inhalt ergänzen (Tex)
\end{REM}

\begin{THEO}{HA-T12-05-15}{Zusammenhangs-Homomorphismus zu kurzen exakten Sequenzen in Kategorie der Komplexe ist natürlich}
% TODO: Inhalt ergänzen (Tex)
\end{THEO}

\begin{REM}{HA-T12-05-16}{PROOF: Zusammenhangs-Homomorphismus zu kurzen exakten Sequenzen in Kategorie der Komplexe ist natürlich}
% TODO: Inhalt ergänzen (Tex)
\end{REM}

\begin{DEF}{HA-T12-05-17}{Arrow Kategorie}
% TODO: Inhalt ergänzen (Tex)
\end{DEF}

\begin{REM}{HA-T12-05-18}{Interpretation des Zusammenhangs-Homomorphismus}
% TODO: Inhalt ergänzen (Tex)
\end{REM}

\begin{DEF}{HA-T12-05-19}{Grad einer Abbildung zwischen Komplexen}
% TODO: Inhalt ergänzen (Tex)
\end{DEF}

\begin{EXA}{HA-T12-05-20}{Exa Grad einer Abbildung zwischen Komplexen}
% TODO: Inhalt ergänzen (Tex)
\end{EXA}

\begin{DEF}{HA-T12-05-21}{Homotope Ketten Abbildungen (Null-Homotopie)}
% TODO: Inhalt ergänzen (Tex)
\end{DEF}

\begin{PROP}{HA-T12-05-22}{Homotope Ketten Abbildungen induzieren gleiche Homologie Abbildungen}
% TODO: Inhalt ergänzen (Tex)
\end{PROP}

\begin{REM}{HA-T12-05-23}{PROOF: Homotope Ketten Abbildungen induzieren gleiche Homologie Abbildungen}
% TODO: Inhalt ergänzen (Tex)
\end{REM}

\begin{DEF}{HA-T12-05-24}{Kontrahierbare Komplexe}
% TODO: Inhalt ergänzen (Tex)
\end{DEF}

\begin{PROP}{HA-T12-05-25}{Kontrahierbare Komplexe sind azyklisch}
% TODO: Inhalt ergänzen (Tex)
\end{PROP}

\begin{STUD}{HA-T12-05-26}{Einführung der Homologie-Funktoren}
% TODO: Inhalt ergänzen (Tex)
\end{STUD}

\begin{STUD}{HA-T12-05-27}{(Natürlicher) Zusammenhangs-Homomorphismus}
% TODO: Inhalt ergänzen (Tex)
\end{STUD}

\begin{STUD}{HA-T12-05-28}{Interpretation des Zusammenhangs-Isomorphismus via Arrow Kategorie}
% TODO: Inhalt ergänzen (Tex)
\end{STUD}

\begin{STUD}{HA-T12-05-29}{Was ist die Singuläre Homologie Theorie}
% TODO: Inhalt ergänzen (Tex)
\end{STUD}

\begin{DEF}{HA-T12-06-01}{Komplex in abelschen Kategorien}
% TODO: Inhalt ergänzen (Tex)
\end{DEF}

\begin{DEF}{HA-T12-06-02}{Ketten Abbildung zwischen Komplexen in abelschen Kategorien}
% TODO: Inhalt ergänzen (Tex)
\end{DEF}

\begin{DEF}{HA-T12-06-03}{Kategorie der Komplexe in abelschen Kategorien}
% TODO: Inhalt ergänzen (Tex)
\end{DEF}

\begin{DEF}{HA-T12-06-04}{Unterkomplex in abelschen Kategorien}
% TODO: Inhalt ergänzen (Tex)
\end{DEF}

\begin{PROP}{HA-T12-06-05}{Kategorie der Komplexe abelsch, falls Kategorie abelsch}
% TODO: Inhalt ergänzen (Tex)
\end{PROP}

\begin{REM}{HA-T12-06-06}{PROOF: Kategorie der Komplexe abelsch, falls Kategorie abelsch}
% TODO: Inhalt ergänzen (Tex)
\end{REM}

\begin{DEF}{HA-T12-06-07}{Isomorphie in Kategorie der Komplexe}
% TODO: Inhalt ergänzen (Tex)
\end{DEF}

\begin{DEF}{HA-T12-06-08}{Direkte Summe von Komplexen}
% TODO: Inhalt ergänzen (Tex)
\end{DEF}

\begin{DEF}{HA-T12-06-09}{Exaktheit von Sequenze von Komplexen und Ketten Abbildungen}
% TODO: Inhalt ergänzen (Tex)
\end{DEF}

\begin{DEF}{HA-T12-06-10}{Kurze Exakte Sequenzen von Komplexen und Ketten Abbildungen}
% TODO: Inhalt ergänzen (Tex)
\end{DEF}

\begin{DEF}{HA-T12-06-11}{Quotienten Komplex}
% TODO: Inhalt ergänzen (Tex)
\end{DEF}

\begin{PROP}{HA-T12-06-12}{Kettenabbildung in Quotienten Komplex durch natürliche Abbildung}
% TODO: Inhalt ergänzen (Tex)
\end{PROP}

\begin{STUD}{HA-T12-06-13}{Basics der allg. Komplexe in abelschen Kategorien und die Kategorie der Komplexe}
% TODO: Inhalt ergänzen (Tex)
\end{STUD}

\begin{LEM}{HA-T12-07-01}{Basics für Exaktheit von Sequenzen}
% TODO: Inhalt ergänzen (Tex)
\end{LEM}

\begin{LEM}{HA-T12-07-02}{Kurze Exakte Sequenzen Basics}
% TODO: Inhalt ergänzen (Tex)
\end{LEM}

\begin{LEM}{HA-T12-07-03}{Links/Rechts Vervollständigung von $03-03$ Komm Exa}
% TODO: Inhalt ergänzen (Tex)
\end{LEM}

\begin{LEM}{HA-T12-07-04}{5-Lemma}
% TODO: Inhalt ergänzen (Tex)
\end{LEM}

\begin{LEM}{HA-T12-07-05}{$030$–$030$ vert. ISO: oben exa $\Leftrightarrow$ unten exa}
% TODO: Inhalt ergänzen (Tex)
\end{LEM}

\begin{LEM}{HA-T12-07-06}{$3\times 3$ Lemma}
% TODO: Inhalt ergänzen (Tex)
\end{LEM}

\begin{LEM}{HA-T12-07-07}{Schlagstock Lemma}
% TODO: Inhalt ergänzen (Tex)
\end{LEM}

\begin{LEM}{HA-T12-07-08}{Schlangen Lemma}
% TODO: Inhalt ergänzen (Tex)
\end{LEM}

\begin{DEF}{HA-T12-10-01}{Splitting Basics}
% TODO: Inhalt ergänzen (Tex)
\end{DEF}

\begin{LEM}{HA-T12-10-02}{Splitting Cases}
% TODO: Inhalt ergänzen (Tex)
\end{LEM}

\begin{DEF}{HA-T12-12-01}{Eilenberg-Stennrod Axiom}
% TODO: Inhalt ergänzen (Tex)
\end{DEF}

\begin{CONC}{QM-T12-02-01}{Plancksche Wirkungsquantum für Energie-Kreisfrequenz und Impuls-Wellenvektor}
% TODO: Inhalt ergänzen (Tex)
\end{CONC}

\begin{DEF}{QM-T12-03-01}{Ebene (kompl) Welle}
% TODO: Inhalt ergänzen (Tex)
\end{DEF}

\begin{DEF}{QM-T12-03-02}{Wellenpaket}
% TODO: Inhalt ergänzen (Tex)
\end{DEF}

\begin{EXA}{QM-T12-03-03}{Überlagerung zweier Wellen}
% TODO: Inhalt ergänzen (Tex)
\end{EXA}

\begin{DEF}{QM-T12-03-04}{Gruppengeschwindigkeit}
% TODO: Inhalt ergänzen (Tex)
\end{DEF}

\begin{CONC}{QM-T12-03-05}{Gruppengeschwindigkeit = Mechanische Geschwindigkeit der zugeordneten Teilchen}
% TODO: Inhalt ergänzen (Tex)
\end{CONC}

\begin{DEF}{QM-T12-03-06}{Phasengeschwindigkeit einer Welle}
% TODO: Inhalt ergänzen (Tex)
\end{DEF}

\begin{DEF}{QM-T12-03-07}{Allgemeine Wellenpaket}
% TODO: Inhalt ergänzen (Tex)
\end{DEF}

\begin{EXA}{QM-T12-03-08}{Gaußsche Wellenpaket in 1D}
% TODO: Inhalt ergänzen (Tex)
\end{EXA}

\begin{DEF}{QM-T12-03-09}{Breite des Wellenpakets im Ort-Raum}
% TODO: Inhalt ergänzen (Tex)
\end{DEF}

\begin{DEF}{QM-T12-03-10}{Breite des Gauß'schen Wellenpakets im k-Raum}
% TODO: Inhalt ergänzen (Tex)
\end{DEF}

\begin{CONC}{QM-T12-03-11}{Unschärfe-Relation für 1D Gaußsche Wellenpaket zw. Impuls und Ort}
% TODO: Inhalt ergänzen (Tex)
\end{CONC}

\begin{CONC}{QM-T12-03-12}{Born-Interpretation}
% TODO: Inhalt ergänzen (Tex)
\end{CONC}

\begin{DEF}{QM-T12-05-01}{Ebene Wellen mit Plankschem Wirkungsquantum}
% TODO: Inhalt ergänzen (Tex)
\end{DEF}

\begin{DEF}{QM-T12-05-02}{Ebene Wellen mit Plankschem Wirkungsquantum zu nicht-relativistischem Teilchen}
% TODO: Inhalt ergänzen (Tex)
\end{DEF}

\begin{CONC}{QM-T12-05-03}{Schrödinger Gleichung für nicht-relativistisches Teilchen}
% TODO: Inhalt ergänzen (Tex)
\end{CONC}

\begin{DEF}{QM-T12-05-04}{Schrödinger Gleichung}
% TODO: Inhalt ergänzen (Tex)
\end{DEF}

\begin{REM}{QM-T12-05-05}{Linearität und Superposition der Schrödinger Gleichung}
% TODO: Inhalt ergänzen (Tex)
\end{REM}

\begin{CONC}{QM-T12-05-06}{Herleitung der freien Schrödinger Gleichung durch Korrspondenzprinzip}
% TODO: Inhalt ergänzen (Tex)
\end{CONC}

\begin{CONC}{QM-T12-06-01}{Energie eines klassischen Teilchen in Potential}
% TODO: Inhalt ergänzen (Tex)
\end{CONC}

\begin{CONC}{QM-T12-06-02}{Verallgemeinerte Schrödinger-Gleichung mit Potential}
% TODO: Inhalt ergänzen (Tex)
\end{CONC}

\begin{REM}{QM-T12-06-03}{Plausibilitätsbetrachtung für allg. Version der Schrödingergleichung}
% TODO: Inhalt ergänzen (Tex)
\end{REM}

\begin{DEF}{QM-T12-07-01}{Matrix-Operator}
% TODO: Inhalt ergänzen (Tex)
\end{DEF}

\begin{EXA}{QM-T12-07-02}{EXA Matrix-Operatoren}
% TODO: Inhalt ergänzen (Tex)
\end{EXA}

\begin{DEF}{QM-T12-07-03}{Linearer Operator}
% TODO: Inhalt ergänzen (Tex)
\end{DEF}

\begin{REM}{QM-T12-07-04}{Komplexe Konjugation ist antilinear}
% TODO: Inhalt ergänzen (Tex)
\end{REM}

\begin{DEF}{QM-T12-07-05}{Kommutator}
% TODO: Inhalt ergänzen (Tex)
\end{DEF}

\begin{DEF}{QM-T12-07-06}{Skalarprodukt}
% TODO: Inhalt ergänzen (Tex)
\end{DEF}

\begin{DEF}{QM-T12-08-01}{Ortsoperator als lin. Multiplikationsoperator}
% TODO: Inhalt ergänzen (Tex)
\end{DEF}

\begin{DEF}{QM-T12-08-02}{Impulsoperator als lin. Differentialoperator}
% TODO: Inhalt ergänzen (Tex)
\end{DEF}

\begin{DEF}{QM-T12-08-03}{Schrödinger Operator}
% TODO: Inhalt ergänzen (Tex)
\end{DEF}

\begin{REM}{QM-T12-08-04}{Linearität des Schrödinger Operator}
% TODO: Inhalt ergänzen (Tex)
\end{REM}

\begin{REM}{QM-T12-08-05}{Darstellung des Schrödinger Operators durch Orts und Impulsoperator}
% TODO: Inhalt ergänzen (Tex)
\end{REM}

\begin{REM}{QM-T12-08-06}{Vertauschungsrelation zwischen Orts und Impulsoperator}
% TODO: Inhalt ergänzen (Tex)
\end{REM}

\begin{REM}{QM-T12-08-07}{Schrödinger Operator als Hamilton Operator}
% TODO: Inhalt ergänzen (Tex)
\end{REM}

\begin{CONC}{QM-T12-08-08}{Korrespondenzprinzip zw. Klassischer Mechanik und Quantenmechanik bzgl Ort und Impuls}
% TODO: Inhalt ergänzen (Tex)
\end{CONC}

\begin{DEF}{QM-T12-08-09}{Poisson Klammer}
% TODO: Inhalt ergänzen (Tex)
\end{DEF}

\begin{DEF}{QM-T12-10-01}{Wahrscheinlichkeitsdichte für die Anwesenheit eines Teilchens in einem Gebiet zu einer Wellenfunktion}
% TODO: Inhalt ergänzen (Tex)
\end{DEF}

\begin{REM}{QM-T12-10-02}{Normierung der Wahrscheinlichkeitsdichte}
% TODO: Inhalt ergänzen (Tex)
\end{REM}

\begin{EXA}{QM-T12-10-03}{Normierung von Gauß-Wellenpaket in 3D zu $t=0$}
% TODO: Inhalt ergänzen (Tex)
\end{EXA}

\begin{REM}{QM-T12-11-01}{Normierung invariant unter zeitlicher Entwicklung}
% TODO: Inhalt ergänzen (Tex)
\end{REM}

\begin{DEF}{QM-T12-11-02}{W-keits-Stromdichte zu einer Wellenfunktion}
% TODO: Inhalt ergänzen (Tex)
\end{DEF}

\begin{CONC}{QM-T12-11-03}{Kontinuitätsgleichung für W-keits-Stromdichte}
% TODO: Inhalt ergänzen (Tex)
\end{CONC}

\begin{REM}{QM-T12-11-04}{Reelle Wellenfunktionen können keinen Strom transportieren}
% TODO: Inhalt ergänzen (Tex)
\end{REM}

\begin{CONC}{QM-T12-11-05}{Kontinuitätsgleichung impliziert Normierungs-Erhaltung}
% TODO: Inhalt ergänzen (Tex)
\end{CONC}

\begin{EXA}{QM-T12-11-06}{Gaußsche Wellenpaket in 3D: W-keitsdichte wird mit Gruppengeschwindigkeit transportiert}
% TODO: Inhalt ergänzen (Tex)
\end{EXA}

\begin{REM}{QM-T12-11-07}{Reelle Faktoren tragen nicht zur Stromdichte bei 
Stromdichte $=$ Geschwindigkeit $*$ Teilchendichte gilt allg}
% TODO: Inhalt ergänzen (Tex)
\end{REM}

\begin{EXA}{QM-T12-12-01}{W-keitsdichte und Stromdichte für ebene Welle: Erfüllt Kontinuitätsgleichung, aber nicht normierbar}
% TODO: Inhalt ergänzen (Tex)
\end{EXA}

\begin{DEF}{QM-T12-12-02}{Fourier-Integral: Fouriertrafo und Inverse Fouriertrafo von Lösungen der (allg) Schrödingergleichung in Ortsvariable}
% TODO: Inhalt ergänzen (Tex)
\end{DEF}

\begin{REM}{QM-T12-12-03}{Fourier-Trafo von Lösung der Schrödinger Gleichung in Normierungsbedingung}
% TODO: Inhalt ergänzen (Tex)
\end{REM}

\begin{CONC}{QM-T12-12-04}{Interpre: Betrag der Inversen Fourier-Trafo als W-keitsdichte im Impulsraum}
% TODO: Inhalt ergänzen (Tex)
\end{CONC}

\begin{DEF}{QM-T12-13-01}{Erwartungswert eine Größe}
% TODO: Inhalt ergänzen (Tex)
\end{DEF}

\begin{CONC}{QM-T12-13-02}{Erwartungswert der Ortskoordinate}
% TODO: Inhalt ergänzen (Tex)
\end{CONC}

\begin{CONC}{QM-T12-13-03}{Erwartungswert des Impulsoperators im Impulsraum}
% TODO: Inhalt ergänzen (Tex)
\end{CONC}

\begin{CONC}{QM-T12-13-04}{Erwartungswert des Impulsoperators im Ortsraum}
% TODO: Inhalt ergänzen (Tex)
\end{CONC}

\begin{REM}{QM-T12-13-05}{Basisdarstellungen im Funktionenraum}
% TODO: Inhalt ergänzen (Tex)
\end{REM}

\begin{DEF}{QM-T12-13-06}{Mittlere Schwankungsquadrat / Varianz einer Größe}
% TODO: Inhalt ergänzen (Tex)
\end{DEF}

\begin{CONC}{QM-T12-13-07}{Heisenberg'sche Unschärfe-Relation}
% TODO: Inhalt ergänzen (Tex)
\end{CONC}

\begin{CONC}{QM-T12-14-01}{Doppelspaltexperiment}
% TODO: Inhalt ergänzen (Tex)
\end{CONC}

\begin{CONC}{QM-T12-14-02}{Wellen-Bild und Abstände der Inferenz-Maxima}
% TODO: Inhalt ergänzen (Tex)
\end{CONC}

\begin{CONC}{QM-T12-14-03}{Ein-Teilchen Inferenz}
% TODO: Inhalt ergänzen (Tex)
\end{CONC}

\begin{DEF}{QM-T12-16-01}{Hermitisch konjugierter Operator}
% TODO: Inhalt ergänzen (Tex)
\end{DEF}

\begin{PROP}{QM-T12-16-02}{$(A*psi)^*=psi^*A^{HK}$}
% TODO: Inhalt ergänzen (Tex)
\end{PROP}

\begin{CONC}{QM-T12-16-03}{Anwendung von Hermitische Konjugation auf Schrödinger Gleichung}
% TODO: Inhalt ergänzen (Tex)
\end{CONC}

\begin{DEF}{QM-T12-16-04}{Selbst-adjungierte Operatoren}
% TODO: Inhalt ergänzen (Tex)
\end{DEF}

\begin{CONC}{QM-T12-16-05}{Hermitische Konjugierte Schrödinger Gleichung}
% TODO: Inhalt ergänzen (Tex)
\end{CONC}

\begin{CONC}{QM-T12-17-01}{Zeitliche Entwicklung des Erwartungswertes}
% TODO: Inhalt ergänzen (Tex)
\end{CONC}

\begin{CONC}{QM-T12-17-02}{Ehrenfest-Theorem, falls Operator A zusätzlich von Zeit abhängt}
% TODO: Inhalt ergänzen (Tex)
\end{CONC}

\begin{CONC}{QM-T12-18-01}{Anwendung von Ehrenfest-Theorem auf Impuls- und Orts-Operator für Teilchen in Zeit-unabhängigem Potential}
% TODO: Inhalt ergänzen (Tex)
\end{CONC}

\begin{CONC}{QM-T12-18-02}{Erwartungswert von Operatoren gehorcht den klassischen Bewegungsgleichungen (Verträglich mit Korrespondenzprinzip)}
% TODO: Inhalt ergänzen (Tex)
\end{CONC}

\begin{CONC}{QM-T12-22-01}{Seperation der Variablen für Schrödinger Gleichung mit zeitunabhängigem Potential}
% TODO: Inhalt ergänzen (Tex)
\end{CONC}

\begin{CONC}{QM-T12-22-02}{Stationäre Schrödinger Gleichung als Eigenwertgleichung}
% TODO: Inhalt ergänzen (Tex)
\end{CONC}

\begin{DEF}{QM-T12-22-03}{Eigenwerte und Eigenfunktionen}
% TODO: Inhalt ergänzen (Tex)
\end{DEF}

\begin{CONC}{QM-T12-22-04}{Lineare Superposition von Lösungen der Schrödinger Gleichung mit zeitunabh. Potential}
% TODO: Inhalt ergänzen (Tex)
\end{CONC}

\begin{DEF}{QM-T12-22-05}{Eigenwert-Spektrum}
% TODO: Inhalt ergänzen (Tex)
\end{DEF}

\begin{CONC}{QM-T12-22-06}{Überlagerungen von orthonormierten Eigenfunktionen sind Lösungen}
% TODO: Inhalt ergänzen (Tex)
\end{CONC}

\begin{EXA}{QM-T12-22-07}{Eigenfunktionen des Impulsoperators}
% TODO: Inhalt ergänzen (Tex)
\end{EXA}

\begin{CONC}{QM-T12-23-01}{Eindimensionale, zeitunabhängige Potential (Heaviside) und zugehörige zeitunabhäängige Schrödinger Gleichung}
% TODO: Inhalt ergänzen (Tex)
\end{CONC}

\begin{REM}{QM-T12-23-02}{Einfluss von Diskontinuitäten auf Wellenfunktion}
% TODO: Inhalt ergänzen (Tex)
\end{REM}

\begin{REM}{QM-T12-23-03}{Randbedingungen}
% TODO: Inhalt ergänzen (Tex)
\end{REM}

\begin{REM}{QM-T12-24-01}{Für konstante Potentiale wird die stationäre Schrödingergleichung durch ebene Wellen gelöst}
% TODO: Inhalt ergänzen (Tex)
\end{REM}

\begin{CONC}{QM-T12-24-02}{Case: $E>V$: Eigenenergie des Zustandes $>$ Potential}
% TODO: Inhalt ergänzen (Tex)
\end{CONC}

\begin{CONC}{QM-T12-24-03}{Superposition von einlaufende und reflektierte Welle für $x\leq 0$}
% TODO: Inhalt ergänzen (Tex)
\end{CONC}

\begin{CONC}{QM-T12-24-04}{Teilchenfluss (W-keits-Stromdichte) der Superposition}
% TODO: Inhalt ergänzen (Tex)
\end{CONC}

\begin{CONC}{QM-T12-24-05}{Transmitierte Welle für $x\geq 0$}
% TODO: Inhalt ergänzen (Tex)
\end{CONC}

\begin{CONC}{QM-T12-24-06}{Stetigkeitsbedingungen an Lösung}
% TODO: Inhalt ergänzen (Tex)
\end{CONC}

\begin{CONC}{QM-T12-24-07}{Teilchenstromerhaltung: W-keits-Teilchenstromdichte ist stetig bei $x=0$}
% TODO: Inhalt ergänzen (Tex)
\end{CONC}

\begin{CONC}{QM-T12-24-08}{Unterschied zur CM: Bei $E>V$ wird an der Potentialstufe ein Bruchteil des Elektrons reflektiert}
% TODO: Inhalt ergänzen (Tex)
\end{CONC}

\begin{REM}{QM-T12-25-01}{Case: $E<V$: Eigenenergie des Zustandes $<$ Potential}
% TODO: Inhalt ergänzen (Tex)
\end{REM}

\begin{CONC}{QM-T12-25-02}{Allg. Lösung für $x\leq 0$}
% TODO: Inhalt ergänzen (Tex)
\end{CONC}

\begin{CONC}{QM-T12-25-03}{Allg. Lösung für $x\geq 0$}
% TODO: Inhalt ergänzen (Tex)
\end{CONC}

\begin{CONC}{QM-T12-25-04}{Normierbarkeit von u:: u bei $0$ beschränkt:: $B_2=0$}
% TODO: Inhalt ergänzen (Tex)
\end{CONC}

\begin{CONC}{QM-T12-25-05}{Stetigkeitsbedingungen an Lösung für $E<V$ bei $x=0$}
% TODO: Inhalt ergänzen (Tex)
\end{CONC}

\begin{CONC}{QM-T12-25-06}{W-keits-Stromdichte konstant 0:: Stehende Wellen transportieren keine Teilchen}
% TODO: Inhalt ergänzen (Tex)
\end{CONC}

\begin{CONC}{QM-T12-25-07}{Unterschied zur CM: Bei $E<V$ dringt ein Bruchteil des Teilchens auch in das verbotene Gebiet $x>0$, klassisch exponentieller Abfall mit Eindingstiefe}
% TODO: Inhalt ergänzen (Tex)
\end{CONC}

\begin{REM}{QM-T12-31-01}{Deltadistributionen}
% TODO: Inhalt ergänzen (Tex)
\end{REM}

\begin{REM}{QM-T12-31-02}{Gaus-Approximation der Deltadistribution}
% TODO: Inhalt ergänzen (Tex)
\end{REM}

\begin{REM}{QM-T12-31-03}{Lorentz-Approximation der Deltadistribution}
% TODO: Inhalt ergänzen (Tex)
\end{REM}

\begin{REM}{QM-T12-31-04}{Fouriertransformation der Lorentz-Funktion}
% TODO: Inhalt ergänzen (Tex)
\end{REM}

\begin{REM}{QM-T12-31-05}{Quantenmechanischer Drehimpuls
- Komponenten durch Korrespondenz
- Kommutator Relationen zwischen Komponenten
- Relationen für Kommutator
- $L^2$,$L_i$ kommutieren
Maximale Menge von kommutierenden (Drehimpuls) Operatoren}
% TODO: Inhalt ergänzen (Tex)
\end{REM}

\begin{REM}{QM-T12-31-06}{Hamiltonian mit komplexem Potential
- Kontinuitätsgleichung für komplexe Potential
- Evolution-Gleichung für totale Wahrscheinlichkeit
- Darstellung von totaler W-keit für W=w reell
- Totale W-keit für $W=0$
- Totale W-keit für $W=w\neq 0$; Interpretation}
% TODO: Inhalt ergänzen (Tex)
\end{REM}

\begin{CONC}{QMS-T12-02-01}{Bezeihugn zw. Ket-Zustands in Hilbertraum und Wellenfunktion}
% TODO: Inhalt ergänzen (Tex)
\end{CONC}

\begin{DEF}{QMS-T12-02-02}{Raum Translation}
% TODO: Inhalt ergänzen (Tex)
\end{DEF}

\begin{DEF}{QMS-T12-02-03}{Symmetrie Transformation}
% TODO: Inhalt ergänzen (Tex)
\end{DEF}

\begin{DEF}{QMS-T12-02-04}{Darstellung der Raum Translationen durch Raum Translation Operator}
% TODO: Inhalt ergänzen (Tex)
\end{DEF}

\begin{CONC}{QMS-T12-02-05}{Raum Translation symmetrisch, dann Wahrscheinlichkeit konstant}
% TODO: Inhalt ergänzen (Tex)
\end{CONC}

\begin{PROP}{QMS-T12-02-06}{Gleichheit der Norm von Wellenfunktion und räumlich transformierter Wellenfunktion, lässt vermuten, dass RT-Operator unitär}
% TODO: Inhalt ergänzen (Tex)
\end{PROP}

\begin{PROP}{QMS-T12-02-07}{Exponential Darstellung des RT-Operator durch Taylor-Entwicklung von räumlich translierter Wellenfunktion}
% TODO: Inhalt ergänzen (Tex)
\end{PROP}

\begin{PROP}{QMS-T12-02-08}{RT-Operator (Exp-Darstellung) ist unitär}
% TODO: Inhalt ergänzen (Tex)
\end{PROP}

\begin{CONC}{QMS-T12-02-09}{Hamilton-Operator invariant unter räumlicher Translation}
% TODO: Inhalt ergänzen (Tex)
\end{CONC}

\begin{CONC}{QMS-T12-02-10}{Lösung von Schrödinger Gleichung invariant unter räumlicher Translation}
% TODO: Inhalt ergänzen (Tex)
\end{CONC}

\begin{CONC}{QMS-T12-02-11}{RT-Operator und Hamilton-Operator kommutieren impliziert Symmetrie von RT-Operator}
% TODO: Inhalt ergänzen (Tex)
\end{CONC}

\begin{DEF}{RG-T12-02-01}{n-dimensionale Untermannigfaltigkeit des euklidischen Raumes}
% TODO: Inhalt ergänzen (Tex)
\end{DEF}

\begin{EXA}{RG-T12-02-02}{n-Sphäre}
% TODO: Inhalt ergänzen (Tex)
\end{EXA}

\begin{EXA}{RG-T12-02-03}{Hyperboloid}
% TODO: Inhalt ergänzen (Tex)
\end{EXA}

\begin{EXA}{RG-T12-02-04}{n-Torus}
% TODO: Inhalt ergänzen (Tex)
\end{EXA}

\begin{EXA}{RG-T12-02-05}{$SO(n)$}
% TODO: Inhalt ergänzen (Tex)
\end{EXA}

\begin{PROP}{RG-T12-02-06}{Charakterisierungen von Untermannigfaltigkeiten im euklidischen Raum}
% TODO: Inhalt ergänzen (Tex)
\end{PROP}

\begin{REM}{RG-T12-02-07}{Anmerkungen}
% TODO: Inhalt ergänzen (Tex)
\end{REM}

\begin{DEF}{RG-T12-03-01}{Atlas auf topologischen Hausdorff-Räumen}
% TODO: Inhalt ergänzen (Tex)
\end{DEF}

\begin{DEF}{RG-T12-03-02}{Äquivalente Atlanten}
% TODO: Inhalt ergänzen (Tex)
\end{DEF}

\begin{REM}{RG-T12-03-03}{Beispiel für nicht äquivalente Atlanten}
% TODO: Inhalt ergänzen (Tex)
\end{REM}

\begin{DEF}{RG-T12-03-04}{Glatte Mannigfaltigkeit}
% TODO: Inhalt ergänzen (Tex)
\end{DEF}

\begin{DEF}{RG-T12-03-05}{Orientierte Mannigfaltigkeiten}
% TODO: Inhalt ergänzen (Tex)
\end{DEF}

\begin{DEF}{RG-T12-03-06}{Untermannigfaltigkeit einer Mannigfaltigkeit}
% TODO: Inhalt ergänzen (Tex)
\end{DEF}

\begin{EXA}{RG-T12-03-07}{n-Torus als Mfk}
% TODO: Inhalt ergänzen (Tex)
\end{EXA}

\begin{EXA}{RG-T12-03-08}{n-Sphäre als Mfk}
% TODO: Inhalt ergänzen (Tex)
\end{EXA}

\begin{EXA}{RG-T12-03-09}{Hyperboloid als Mfk}
% TODO: Inhalt ergänzen (Tex)
\end{EXA}

\begin{EXA}{RG-T12-03-10}{Reelle projektiver Raum als Mfk}
% TODO: Inhalt ergänzen (Tex)
\end{EXA}

\begin{EXA}{RG-T12-03-11}{Komplexe projektive Raum als Mfk}
% TODO: Inhalt ergänzen (Tex)
\end{EXA}

\begin{EXA}{RG-T12-03-12}{Möbiusband als Mfk}
% TODO: Inhalt ergänzen (Tex)
\end{EXA}

\begin{REM}{RG-T12-03-13}{Quotienten-Räume als Mfk als Motivation für verallg. Mfk-Begriff}
% TODO: Inhalt ergänzen (Tex)
\end{REM}

\begin{DEF}{RG-T12-03-14}{TEST}
% TODO: Inhalt ergänzen (Tex)
\end{DEF}

\begin{DEF}{RG-T12-03-15}{TEST 2}
% TODO: Inhalt ergänzen (Tex)
\end{DEF}

\begin{DEF}{RG-T12-03-16}{TEST 3}
% TODO: Inhalt ergänzen (Tex)
\end{DEF}

\begin{DEF}{RG-T12-04-01}{Glatte Abbildung zwischen Mfk}
% TODO: Inhalt ergänzen (Tex)
\end{DEF}

\begin{DEF}{RG-T12-04-02}{Immersion / Submersion von Mfk}
% TODO: Inhalt ergänzen (Tex)
\end{DEF}

\begin{DEF}{RG-T12-04-03}{Einbettung von Mfk}
% TODO: Inhalt ergänzen (Tex)
\end{DEF}

\begin{DEF}{RG-T12-04-04}{Diffeomorphismus von Mfk}
% TODO: Inhalt ergänzen (Tex)
\end{DEF}

\begin{DEF}{RG-T12-04-05}{Tangentialvektor: Äquivalenzklassen von Kurven}
% TODO: Inhalt ergänzen (Tex)
\end{DEF}

\begin{DEF}{RG-T12-04-06}{Tangentenvektoren: Keime}
% TODO: Inhalt ergänzen (Tex)
\end{DEF}

\begin{DEF}{RG-T12-04-07}{Tangentenvektoren: Paare von Koordinatensysteme um p und Vektor}
% TODO: Inhalt ergänzen (Tex)
\end{DEF}

\begin{REM}{RG-T12-04-08}{Konstruktion des Tangentialraums}
% TODO: Inhalt ergänzen (Tex)
\end{REM}

\begin{DEF}{RG-T12-04-09}{Tangentialbündel}
% TODO: Inhalt ergänzen (Tex)
\end{DEF}

\begin{PROP}{RG-T12-04-10}{Tangentialbündel ist 2n-dimensional Mfk}
% TODO: Inhalt ergänzen (Tex)
\end{PROP}

\begin{DEF}{RG-T12-04-11}{Vektorfeld als glatter Schnitt in Tangentialbündel}
% TODO: Inhalt ergänzen (Tex)
\end{DEF}

\begin{REM}{RG-T12-04-12}{Darstellung von Vektorfeldern durch partielle Abbleitungen (Tangentenvektoren)}
% TODO: Inhalt ergänzen (Tex)
\end{REM}

\begin{DEF}{RG-T12-04-13}{Vektorfeld als Abbildung von glatten Funktionen auf Mfk}
% TODO: Inhalt ergänzen (Tex)
\end{DEF}

\begin{DEF}{RG-T12-04-14}{Lieklammer von Vektorfeldern (ergibt Vektorfelder)}
% TODO: Inhalt ergänzen (Tex)
\end{DEF}

\begin{REM}{RG-T12-04-15}{Lieklammer:
Jacobi-Identität
Schiefsymmetrisch
Nicht linear über R}
% TODO: Inhalt ergänzen (Tex)
\end{REM}

\begin{DEF}{RG-T12-04-16}{Differential von glatten Abbildungen}
% TODO: Inhalt ergänzen (Tex)
\end{DEF}

\begin{DEF}{RG-T12-05-01}{Finsler-Metrik auf Mfk}
% TODO: Inhalt ergänzen (Tex)
\end{DEF}

\begin{DEF}{RG-T12-05-02}{Länge von Kurven auf Mfk}
% TODO: Inhalt ergänzen (Tex)
\end{DEF}

\begin{DEF}{RG-T12-05-03}{Riemannische Metrik auf Mfk}
% TODO: Inhalt ergänzen (Tex)
\end{DEF}

\begin{REM}{RG-T12-05-04}{Pseudo-Riemannische Metrik auf Mfk}
% TODO: Inhalt ergänzen (Tex)
\end{REM}

\begin{REM}{RG-T12-05-05}{Lokale Beschreibung von Riemannischer Metrik}
% TODO: Inhalt ergänzen (Tex)
\end{REM}

\begin{DEF}{RG-T12-05-06}{Abzählbare Mfk im Unendlichen}
% TODO: Inhalt ergänzen (Tex)
\end{DEF}

\begin{REM}{RG-T12-05-07}{Relevanz der Abzählbarkeit im Unendlichen
1) Existenz von Verfeinerungen (lokal endlich) für offene Überdeckungen
2) Zerlegung der Eins}
% TODO: Inhalt ergänzen (Tex)
\end{REM}

\begin{PROP}{RG-T12-05-08}{Jede Mfk besitzt eine Riemannische Metrik}
% TODO: Inhalt ergänzen (Tex)
\end{PROP}

\begin{EXA}{RG-T12-05-09}{$\R^2$ in Polarkoordinaten}
% TODO: Inhalt ergänzen (Tex)
\end{EXA}

\begin{DEF}{RG-T12-05-10}{(Lokale) Isometrie von RMfk}
% TODO: Inhalt ergänzen (Tex)
\end{DEF}

\begin{DEF}{RG-T12-05-11}{Isometrische Einbettung von RMfk}
% TODO: Inhalt ergänzen (Tex)
\end{DEF}

\begin{EXA}{RG-T12-05-12}{Untermannigfaltigkeit mit induzierter Metrik}
% TODO: Inhalt ergänzen (Tex)
\end{EXA}

\begin{REM}{RG-T12-05-13}{Kompakte Mfk lassen sich in Euklidischen Raum einbetten}
% TODO: Inhalt ergänzen (Tex)
\end{REM}

\begin{REM}{RG-T12-05-14}{Unterscheidung zw. Innerer und äußerer Geometrie: Eigenschaften der Mfk vs der Einbettung}
% TODO: Inhalt ergänzen (Tex)
\end{REM}

\begin{EXA}{RG-T12-05-15}{Rotationsfläche}
% TODO: Inhalt ergänzen (Tex)
\end{EXA}

\begin{EXA}{RG-T12-05-16}{Hyperbolischer Raum}
% TODO: Inhalt ergänzen (Tex)
\end{EXA}

\begin{EXA}{RG-T12-05-17}{Poincaremodell des hyperbolischen Raumes}
% TODO: Inhalt ergänzen (Tex)
\end{EXA}

\begin{DEF}{RG-T12-05-18}{Riemannisches Produkt}
% TODO: Inhalt ergänzen (Tex)
\end{DEF}

\begin{EXA}{RG-T12-05-19}{RxSn}
% TODO: Inhalt ergänzen (Tex)
\end{EXA}

\begin{EXA}{RG-T12-05-20}{Flacher Torus}
% TODO: Inhalt ergänzen (Tex)
\end{EXA}

\begin{PROP}{RG-T12-05-21}{Charakterisierung der Isometrien von flachen Tori}
% TODO: Inhalt ergänzen (Tex)
\end{PROP}

\begin{EXA}{RG-T12-05-22}{Kleinsche Flasche}
% TODO: Inhalt ergänzen (Tex)
\end{EXA}

\begin{REM}{RG-T12-07-01}{Ableitung von Vektorfeldern längs Vektorfeldern}
% TODO: Inhalt ergänzen (Tex)
\end{REM}

\begin{DEF}{RG-T12-07-02}{Zusammenhang auf Mfk}
% TODO: Inhalt ergänzen (Tex)
\end{DEF}

\begin{THEO}{RG-T12-07-03}{Fundamentaltheorem der Riemannischen Geometrie}
% TODO: Inhalt ergänzen (Tex)
\end{THEO}

\begin{DEF}{RG-T12-07-04}{Koszulgleichung für Zusammenhänge}
% TODO: Inhalt ergänzen (Tex)
\end{DEF}

\begin{DEF}{RG-T12-07-05}{Chirstoffelsymbole als Korrekturterme in lokalen Koordinaten}
% TODO: Inhalt ergänzen (Tex)
\end{DEF}

\begin{REM}{RG-T12-07-06}{Levi-Civita Zusammenhang $D_XY_p$ hängt nur von $X_p$ ab}
% TODO: Inhalt ergänzen (Tex)
\end{REM}

\begin{LEM}{RG-T12-07-07}{Formel für Christoffelsymbole aus Koszulgleichung}
% TODO: Inhalt ergänzen (Tex)
\end{LEM}

\begin{EXA}{RG-T12-07-08}{Christoffelsymbole für Rn für euklidische Metrik}
% TODO: Inhalt ergänzen (Tex)
\end{EXA}

\begin{EXA}{RG-T12-07-09}{Christoffelsymbole für $\R^2\backslash 0$ und lokale Darstellung der Metrik zur Polarkoordianten}
% TODO: Inhalt ergänzen (Tex)
\end{EXA}

\begin{PROP}{RG-T12-07-10}{Induzierter LC-Zusammenhang auf Untermannigfaltigkeiten von RMfk}
% TODO: Inhalt ergänzen (Tex)
\end{PROP}

\begin{DEF}{RG-T12-08-01}{Vektorfelder längs Kurven}
% TODO: Inhalt ergänzen (Tex)
\end{DEF}

\begin{PROP}{RG-T12-08-02}{Kovariante Ableitungs-Operator längs Kurven induziert durch LC-Zusammenhang der RMfk (EE)}
% TODO: Inhalt ergänzen (Tex)
\end{PROP}

\begin{PROP}{RG-T12-08-03}{Kovariante Ableitung der Riemannischen Metrik längs Kurven}
% TODO: Inhalt ergänzen (Tex)
\end{PROP}

\begin{DEF}{RG-T12-09-01}{Parallele Vektorfelder längs Kurven}
% TODO: Inhalt ergänzen (Tex)
\end{DEF}

\begin{PROP}{RG-T12-09-02}{Eind. Existenz von parallelen Vektorfelder für Anfangswert (Punkt,Tangentialvektor)}
% TODO: Inhalt ergänzen (Tex)
\end{PROP}

\begin{DEF}{RG-T12-09-03}{Parallelverschiebung von Tangentialvektoren bzgl parallelen Vektorfeldern längs Kurven}
% TODO: Inhalt ergänzen (Tex)
\end{DEF}

\begin{REM}{RG-T12-09-04}{Abhängigkeit der Parallelverschiebung von Kurve}
% TODO: Inhalt ergänzen (Tex)
\end{REM}

\begin{PROP}{RG-T12-09-05}{Parallelverschiebung ist Isometrie zwischen Tangentialräumen}
% TODO: Inhalt ergänzen (Tex)
\end{PROP}

\begin{EXA}{RG-T12-09-06}{Parallelverschiebung im euklidischen Raum}
% TODO: Inhalt ergänzen (Tex)
\end{EXA}

\begin{EXA}{RG-T12-09-07}{Parallelverschiebung auf $S^n$}
% TODO: Inhalt ergänzen (Tex)
\end{EXA}

\begin{REM}{RG-T12-16-01}{n-Torus ist glatte Mfk}
% TODO: Inhalt ergänzen (Tex)
\end{REM}

\begin{EXA}{RG-T12-16-02}{R mit zwei $0$ ist nicht Hausdorff}
% TODO: Inhalt ergänzen (Tex)
\end{EXA}

\begin{EXA}{RG-T12-16-03}{Lie-Klammer berechnen von Vektorfeldern}
% TODO: Inhalt ergänzen (Tex)
\end{EXA}

\begin{REM}{RG-T12-17-01}{Tangentialbündel ist glatte Mfk}
% TODO: Inhalt ergänzen (Tex)
\end{REM}

\begin{REM}{RG-T12-17-02}{Differential und Untermannigfaltigkeit von regulären Werten}
% TODO: Inhalt ergänzen (Tex)
\end{REM}

\begin{REM}{RG-T12-17-03}{Lorentz-Skalarprodukt ist eine Metrik}
% TODO: Inhalt ergänzen (Tex)
\end{REM}

\begin{REM}{RG-T12-17-04}{Pullback-Metrik von Hyperbolischem Raum mit Lorentz-Metrik}
% TODO: Inhalt ergänzen (Tex)
\end{REM}

\begin{DEF}{TOP1-T12-02-01}{Standard-p-Simpelx}
% TODO: Inhalt ergänzen (Tex)
\end{DEF}

\begin{DEF}{TOP1-T12-02-02}{Singulärer-p-Simplex}
% TODO: Inhalt ergänzen (Tex)
\end{DEF}

\begin{DEF}{TOP1-T12-02-03}{Singuläre-p-Kettengruppe}
% TODO: Inhalt ergänzen (Tex)
\end{DEF}

\begin{DEF}{TOP1-T12-02-04}{Rand-Operator auf Singulären-p-Kettengruppen}
% TODO: Inhalt ergänzen (Tex)
\end{DEF}

\begin{DEF}{TOP1-T12-02-05}{Definition der Singulären Homologie Abbildung und Singuläre Komplexe}
% TODO: Inhalt ergänzen (Tex)
\end{DEF}

\begin{REM}{TOP1-T12-02-06}{Inklusion' der Singulären Komplexe}
% TODO: Inhalt ergänzen (Tex)
\end{REM}

\begin{REM}{TOP1-T12-02-07}{Problem: Ist die definierte Singuläre Homologie Abbildung eine Instanz einer Homologie Theorie?}
% TODO: Inhalt ergänzen (Tex)
\end{REM}

\begin{DEF}{TOP1-T12-04-01}{Affiner Singulärer Simplex}
% TODO: Inhalt ergänzen (Tex)
\end{DEF}

\begin{DEF}{TOP1-T12-04-02}{Unterkategorie von Singulärer Komplex der Standard Simplex erzeugt durch affine singuläre Simplize}
% TODO: Inhalt ergänzen (Tex)
\end{DEF}

\begin{DEF}{TOP1-T12-04-03}{Kegel Operator auf affine singuläre Simplize}
% TODO: Inhalt ergänzen (Tex)
\end{DEF}

\begin{LEM}{TOP1-T12-04-04}{Rand von Kegel eines affinen singulären Simplex}
% TODO: Inhalt ergänzen (Tex)
\end{LEM}

\begin{DEF}{TOP1-T12-04-05}{Barycenter-Operator (Subdivison Operator) auf Unterkomplex der affinen singulären Simplize}
% TODO: Inhalt ergänzen (Tex)
\end{DEF}

\begin{PROP}{TOP1-T12-04-06}{Barycenter-Operator ist Kettenabbildung zwischen Subkomplexen der affinen singulären Simplize}
% TODO: Inhalt ergänzen (Tex)
\end{PROP}

\begin{DEF}{TOP1-T12-04-07}{Homotopie-Operator von Subkomp p nach Subkomp $p+1$}
% TODO: Inhalt ergänzen (Tex)
\end{DEF}

\begin{PROP}{TOP1-T12-04-08}{Barycenter-Operator ist homotop zu Identität-K-Abb bzgl Homotopie-Operator}
% TODO: Inhalt ergänzen (Tex)
\end{PROP}

\begin{DEF}{TOP1-T12-04-09}{Barycenter-Operator auf singulären Komplexe}
% TODO: Inhalt ergänzen (Tex)
\end{DEF}

\begin{DEF}{TOP1-T12-04-10}{Homotopie-Operator auf singulären Komplexen}
% TODO: Inhalt ergänzen (Tex)
\end{DEF}

\begin{THEO}{TOP1-T12-04-11}{a) Operatoren sind Natürlich 
b) BC-Operator ist Ketten-Abbildung und homotop zu Identität bzgl H-Operator 
c) Verträglich mit Def auf Subkompelx der affinen sing. Simplize
d) Abbildungen von sing Simplex im Bild des sing Simplex}
% TODO: Inhalt ergänzen (Tex)
\end{THEO}

\begin{PROP}{TOP1-T12-04-12}{Iterative Anwendung von BC-Operator ist kettenhomotop zur Identität bzgl Abbildung abhängig von H-Operator}
% TODO: Inhalt ergänzen (Tex)
\end{PROP}

\begin{LEM}{TOP1-T12-04-13}{Durchmesser Abschätzung für Simplize in Ketten der Barycenter Unterteilung von affinen Simplize}
% TODO: Inhalt ergänzen (Tex)
\end{LEM}

\begin{KORO}{TOP1-T12-04-14}{Durchmesser Abschätzung für affine Simplize in k-fach BC-unterteilten Identitäten von Standard-Simplizes}
% TODO: Inhalt ergänzen (Tex)
\end{KORO}

\begin{PROP}{TOP1-T12-04-15}{k-fach BC-unterteilter singulärer Simplex ist U-klein}
% TODO: Inhalt ergänzen (Tex)
\end{PROP}

\begin{DEF}{TOP1-T12-04-16}{Singulärer Subkomplexe erzeugt durch U-kleine singuläre Simplize und Homologiegruppen über offener Überdeckung}
% TODO: Inhalt ergänzen (Tex)
\end{DEF}

\begin{PROP}{TOP1-T12-04-17}{Homologie-Äquivalenz von normalen Homologiegruppen und Homologiegruppen über offenen Überdeckungen}
% TODO: Inhalt ergänzen (Tex)
\end{PROP}

\begin{DEF}{TOP1-T12-04-18}{Singulärer Quotientenkomplex über offener Überdeckung}
% TODO: Inhalt ergänzen (Tex)
\end{DEF}

\begin{THEO}{TOP1-T12-04-19}{Ausschneidungssatz: Zu $B\subset A\subset X$ induziert die Inklusion $(X-B,A-B)$ in $(X,A)$ eine Isomorphismus zw relativen Homologiegruppen $(X,A)$ über $\{A,X-B\}$ und relative Homologiegruppe $(X,A)$}
% TODO: Inhalt ergänzen (Tex)
\end{THEO}

\begin{THEO}{TOP1-T12-04-20}{Ausschneidungsaxiom}
% TODO: Inhalt ergänzen (Tex)
\end{THEO}

\begin{STUD}{TOP1-T12-04-21}{Eigenschaften des Barycenter / Homotopie Operators}
% TODO: Inhalt ergänzen (Tex)
\end{STUD}

\begin{DEF}{TOP1-T12-09-01}{Relative Homologie (QuotientenKomplexe)}
% TODO: Inhalt ergänzen (Tex)
\end{DEF}

\begin{THEO}{TOP1-T12-09-02}{Existenz von exakter sing. Homologie-Sequenz für $(X,A)$}
% TODO: Inhalt ergänzen (Tex)
\end{THEO}

\begin{THEO}{TOP1-T12-09-03}{Kurze Exakte Sequenz von relative Singulären Komplexen induziert lange exakte relative Homologie-Sequenz}
% TODO: Inhalt ergänzen (Tex)
\end{THEO}

\begin{KORO}{TOP1-T12-09-04}{Splitting der Singulären Komplexe von $(X,A)$ nach $(X)$}
% TODO: Inhalt ergänzen (Tex)
\end{KORO}

\begin{DEF}{TOP1-T12-09-05}{Relative Homologie (Relative Zyklen u Ränder)}
% TODO: Inhalt ergänzen (Tex)
\end{DEF}

\begin{DEF}{TOP1-T12-09-06}{Relative Zyklen und Ränder}
% TODO: Inhalt ergänzen (Tex)
\end{DEF}

\begin{PROP}{TOP1-T12-09-07}{Definition der relativen Homologie sind isomorph}
% TODO: Inhalt ergänzen (Tex)
\end{PROP}

\begin{DEF}{TOP1-T12-09-08}{Aumentierter singulärer Komplex}
% TODO: Inhalt ergänzen (Tex)
\end{DEF}

\begin{DEF}{TOP1-T12-09-09}{Reduzierte Homologiegruppe}
% TODO: Inhalt ergänzen (Tex)
\end{DEF}

\begin{PROP}{TOP1-T12-09-10}{Reduzierte Homologiegruppen sind ISO zu punktierte Homologiegruppe für $n\geq 0$}
% TODO: Inhalt ergänzen (Tex)
\end{PROP}

\pagebreak
\printbibliography
\end{document}