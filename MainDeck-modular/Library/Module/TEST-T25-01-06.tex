% Auto‑generiert aus Library.tex
% UnitID: TEST-T25-01-06
% Titel : TESTBeispiel8

\begin{EXA}{TEST-T25-01-06}{TESTBeispiel8}
a) $p r_{1}: U \times F \rightarrow U$ ist eine lokal triviale Faserung.\\
b) $T M:=\bigcup_{x \in M} T_{x} M \rightarrow M$ mit der üblichen differenzierbaren Struktur ist ein $\operatorname{dim} M$-dimensionales Vektorraumbündel. Denn ist ( $U, h$ ) eine Karte für $M$ und $\left(\partial_{1}^{(h)}, \ldots, \partial_{n}^{(h)}\right)$ Koordinatenbasis auf $U$, so ist

$$
\bigcup_{x \in U} T_{x} U \rightarrow U \times \mathbb{R}^{n}, \sim \sum_{i=1}^{n} a_{i}(x) \partial_{i}^{(h)}(x) \mapsto\left(x, a_{1}(x), \ldots, a_{n}(x)\right) .
$$

eine Bündelkarte.\\
c) Sei $U:=[0,1] / 0 \sim 1 \cong S^{1}$.\\
$E:=[0,1] \times \mathbb{R} /(0, t) \sim(1,-t) \neq S^{1} \times \mathbb{R}$. Dann ist $\pi: E \rightarrow U,[(x, t)] \mapsto[x]$ ein Vektorraumbündel:

Ist $x \neq[0]$, so wähle $U=(0,1) \subset M$. Dann ist

$$
\pi^{-1}(U)=\{(x, t) \mid x \in(0,1), t \in \mathbb{R}\} \underset{\varphi}{\cong} U \times \mathbb{R}
$$

Ist $x=[0]$, so wähle $U=M \backslash\left\{\frac{1}{2}\right\}$ und

$$
\varphi: \pi^{-1}(U) \rightarrow U \times \mathbb{R},\left\{[(x, t)] \left\lvert\, x \neq \frac{1}{2}\right., t \in \mathbb{R}\right\} \mapsto \begin{cases}([x], t), & 0 \leq x<\frac{1}{2} \\ ([x],-t) & \frac{1}{2}<x \leq 1\end{cases}
$$

d) $S^{1} \rightarrow S^{1}, z \mapsto z^{2}$ ist eine lokal triviale Faserung mit $F=\mathbb{Z}_{2}$. (Übungsaufgabe: Was ist $\varphi$ ?)
\end{EXA}
