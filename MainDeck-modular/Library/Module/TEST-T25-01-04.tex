% Auto‑generiert aus Library.tex
% UnitID: TEST-T25-01-04
% Titel : TESTBeispiel7

\begin{EXA}{TEST-T25-01-04}{TESTBeispiel7}
a) Sei $(E, \pi, M)$ eine lokal triviale Faserung wie in 1.1. Dann heißt $E$ Totalraum, $M$ Basis, $\pi$ Bündelprojektion und $F$ typische Faser.\\
Für jedes $x \in M$ heißt $E_{x}=\pi^{-1}(x)$ reale Faser an der Stelle $x$.\\
Für $U \subset M$ offen heißt $\varphi: E \mid U \rightarrow U \times F$ Bündelkarte und

$$
\left\{\left(U_{\lambda}, \varphi_{\lambda}\right) \mid\left(U_{\lambda}, \varphi_{\lambda}\right) \text { Bündelkarte }, \bigcup_{\lambda \in \Lambda} U_{\lambda}=M\right\}
$$

heißt Bündelatlas.\\
Die Abbildung $\varphi_{x}: E_{x} \rightarrow F, \varphi_{x}:=p r_{2} \circ \varphi \mid E_{x}$ heißt Faserkarte.\\
Sind $(U, \varphi)$ und $(V, \psi)$ Bündelkarten, so heißt die Abbildung

$$
\omega: U \cap V \rightarrow \operatorname{Diffeo}(F), x \mapsto \psi_{x} \circ \varphi_{x}^{-1}
$$

der Bündelkartenwechsel zwischen $\varphi$ und $\psi$.
\end{EXA}
