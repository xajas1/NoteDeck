\documentclass[10pt, letterpaper]{article}

% Inhaltsverzeichnis für Pakettypen (nur für Übersicht im Header, wird nicht im Dokument angezeigt)
% 1. Seitenlayout und Ränder
% 2. Sprache und Zeichensatz
% 3. Mathematik und Theorem-Umgebungen
% 4. Eigene Makros
% 5. Diagramme und Grafiken
% 6. Tabellen und Aufzählungen
% 7. Inhaltsverzeichnis
% 8. Abschnittsüberschriften
% 9. Abstrakt-Umgebung
% 10. Todos/Notizen
% 11. Rahmen/Box-Umgebungen
% 12. Python-Integration
% 13. Literaturverwaltung
% 14. Hyperlinks
% 15. Absatzeinstellungen
% 16. Umgebungen
% 17  Graphik
% 18  Extra
% 00. Titel und Autor

% --- 1. Seitenlayout und Ränder ---
\usepackage[margin=3cm]{geometry}

% --- 2. Sprache und Zeichensatz ---
\usepackage[english]{babel}
\usepackage[T1]{fontenc}
\usepackage[utf8]{inputenc}

% --- 3. Mathematik und Theorem-Umgebungen ---
\usepackage{amsmath, amssymb, amsthm}
\usepackage{mathrsfs}
\DeclareMathOperator{\WF}{WF}

% --- 4. Eigene Makros ---
\usepackage{xcolor}
\newcommand{\SKP}{\langle\cdot,\cdot\rangle}
\newcommand{\R}{\mathbb{R}}
\newcommand{\N}{\mathbb{N}}
\newcommand{\Q}{\mathbb{Q}}
\newcommand{\Z}{\mathbb{Z}}
\newcommand{\C}{\mathbb{C}}
\newcommand{\entwurf}[1]{\textcolor{red}{#1}}

% --- 5. Diagramme und Grafiken ---
\usepackage{graphicx}
\usepackage{tikz}
\usetikzlibrary{decorations.pathreplacing, arrows.meta, positioning}
\usepackage{tikz-cd}

% --- 6. Tabellen und Aufzählungen ---
\usepackage{enumitem}
\setlist[itemize]{left=0.5cm}

\newenvironment{romanenum}[1][]
  {%
    \ifx&#1&
    \else
      \textbf{#1}\quad
    \fi
    \begin{enumerate}[label=\roman*)]
  }
  {%
    \end{enumerate}%
  }

% --- 7. Inhaltsverzeichnis ---
\usepackage{tocloft}
\renewcommand{\cftsecfont}{\footnotesize}
\renewcommand{\cftsubsecfont}{\footnotesize}
\renewcommand{\cftsubsubsecfont}{\footnotesize}
\renewcommand{\cftsecpagefont}{\footnotesize}
\renewcommand{\cftsubsecpagefont}{\footnotesize}
\renewcommand{\cftsubsubsecpagefont}{\footnotesize}
\usepackage{etoc}

% --- 8. Abschnittsüberschriften ---
\usepackage{titlesec}
\titleformat{\section}{\normalfont\large\bfseries}{\thesection}{1em}{}
\titleformat{\subsection}{\normalfont\normalsize\bfseries}{\thesubsection}{0.5em}{}
\titleformat{\subsubsection}{\normalfont\normalsize\bfseries}{\thesubsubsection}{0.5em}{}
\setcounter{secnumdepth}{4}

% --- 9. Abstrakt-Umgebung ---
\usepackage{changepage}
\renewenvironment{abstract}
  {
    \begin{adjustwidth}{1.5cm}{1.5cm}
    \small
    \textsc{Abstract. –}%
  }
  {
    \end{adjustwidth}
  }

% --- 10. Todos/Notizen ---
\usepackage{todonotes}

% --- 11. Rahmen/Box-Umgebungen ---
\usepackage{mdframed}
\usepackage{tcolorbox}
\colorlet{shadecolor}{gray!25}

\newenvironment{customTheorem}
  {\vspace{10pt}%
   \begin{mdframed}[
     backgroundcolor=gray!20,
     linewidth=0pt,
     innertopmargin=10pt,
     innerbottommargin=10pt,
     skipabove=\dimexpr\topsep+\ht\strutbox\relax,
     skipbelow=\topsep,
   ]}
  {\end{mdframed}
   \vspace{10pt}%
  }

% --- 12. Python-Integration ---
% (Deaktiviert in dieser Version, aktiviere bei Bedarf)
% \usepackage{pythontex}
% \usepackage[makestderr]{pythontex}

% --- 13. Literaturverwaltung ---
\usepackage{csquotes}
\usepackage[backend=biber, style=alphabetic, citestyle=alphabetic]{biblatex}
\addbibresource{bibliography.bib}

% --- 14. Hyperlinks ---
\usepackage{hyperref}
\hypersetup{
  colorlinks   = true,
  urlcolor     = blue,
  linkcolor    = blue,
  citecolor    = blue,
  frenchlinks  = true
}

% --- 15. Absatzeinstellungen ---
\usepackage[parfill]{parskip}
\sloppy

% --- 16. Umgebungen ---
\usepackage{thmtools}
\usepackage{tikz-cd}

\newcommand{\CustomHeading}[3]{%
  \par\medskip\noindent%
  \textbf{#1 #2} \textnormal{(#3)}.\enskip%
}

\newenvironment{DEF}[2]{\begin{unitbox}\CustomHeading{Definition}{#1}{#2}}{\end{unitbox}}
\newenvironment{PROP}[2]{\begin{unitbox}\CustomHeading{Proposition}{#1}{#2}}{\end{unitbox}}
\newenvironment{THEO}[2]{\begin{unitbox}\CustomHeading{Theorem}{#1}{#2}}{\end{unitbox}}
\newenvironment{LEM}[2]{\begin{unitbox}\CustomHeading{Lemma}{#1}{#2}}{\end{unitbox}}
\newenvironment{KORO}[2]{\begin{unitbox}\CustomHeading{Corollar}{#1}{#2}}{\end{unitbox}}
\newenvironment{REM}[2]{\begin{unitbox}\CustomHeading{Remark}{#1}{#2}}{\end{unitbox}}
\newenvironment{EXA}[2]{\begin{unitbox}\CustomHeading{Example}{#1}{#2}}{\end{unitbox}}
\newenvironment{STUD}[2]{\begin{unitbox}\CustomHeading{Study}{#1}{#2}}{\end{unitbox}}
\newenvironment{CONC}[2]{\begin{unitbox}\CustomHeading{Concept}{#1}{#2}}{\end{unitbox}}
\newenvironment{OTH}[2]{\begin{unitbox}\CustomHeading{Other}{#1}{#2}}{\end{unitbox}}
\newenvironment{EXE}[2]{\begin{unitbox}\CustomHeading{Exercise}{#1}{#2}}{\end{unitbox}}
\newenvironment{MOT}[2]{\begin{unitbox}\CustomHeading{Motivation}{#1}{#2}}{\end{unitbox}}
\newenvironment{PROOF}[2]{\begin{unitbox}\CustomHeading{Proof}{#1}{#2}}{\end{unitbox}}

% --- Unit Umgebung für Source-Inhalte ---
\usepackage{mdframed}
\newmdenv[
  linewidth=1pt,
  topline=false,
  bottomline=false,
  rightline=false,
  leftmargin=0cm,
  rightmargin=0cm,
  skipabove=10pt,
  skipbelow=10pt,
  innertopmargin=0.5\baselineskip,
  innerbottommargin=0.5\baselineskip,
  backgroundcolor=gray!10,
  linecolor=gray
]{unitbox}

\newenvironment{unit}[1]
  {\begin{unitbox}\textbf{Unit #1}\par\smallskip}
  {\end{unitbox}}

% --- 17. Graphik ---
\usepackage{graphicx}
\graphicspath{ {./images/} }
\usepackage[export]{adjustbox}

% --- 18. Extras ---
\usepackage{stmaryrd}
\usepackage{bbold}  % falls du athbb{1} nutzen willst

% --- 00. Titel und Autor ---
\title{Mein Titel}
\author{Tim Jaschik}
\date{\today}

\begin{document}

\maketitle
\rule{\textwidth}{0.5pt}
\begin{abstract}
Kurze Beschreibung …
\end{abstract}
\rule{\textwidth}{0.5pt}
\vspace{0.5cm}

\tableofcontents

\pagebreak



IV.11. Der Hurewicz Homomorphismus. 

Wir identifizieren $\Delta^1 \cong I$, wobei $\left(t_0, t_1\right) \in \Delta^1$ dem Element $t_1 \in I$ zugeordnet wird, dh. die Ecken $e_0, e_1 \in \Delta^1$ entsprechen $e_0 \leftrightarrow 0$ und $e_1 \leftrightarrow 1$. Mit Hilfe dieser Identifizierung können wir Wege $\sigma: I \rightarrow X$ mit 1-Simplizes $\tilde{\sigma}: \Delta^1 \rightarrow X$ identifizieren, $\tilde{\sigma}\left(t_0, t_1\right)=\sigma\left(t_1\right)$.


IV.11.1. Lemma. 

Es gilt:
\begin{enumerate}[label=(\roman*)]
    \item Ist $x \in X$, dann existiert $\tau \in C_2(X)$ mit $\tilde{c}_x = \partial \tau$.${}^{35}$
    
    \item Ist $\sigma: I \rightarrow X$ eine Schleife, dann gilt $\partial \tilde{\sigma} = 0$.
    
    \item Sind $\sigma_0 \simeq \sigma_1: I \rightarrow X$ homotop relativ Endpunkten, dann existiert $\tau \in C_2(X)$ mit $\tilde{\sigma}_1 = \tilde{\sigma}_0 + \partial \tau$.
    
    \item Sind $\sigma_0, \sigma_1: I \rightarrow X$ mit $\sigma_0(1) = \sigma_1(0)$, dann existiert $\tau \in C_2(X)$ mit $\left( \sigma_0 \sigma_1 \right)^{\sim} = \tilde{\sigma}_0 + \tilde{\sigma}_1 + \partial \tau$.
    
    \item Ist $\sigma: I \rightarrow X$, dann existiert $\tau \in C_2(X)$ mit $\bar{\sigma}^{\sim} = -\tilde{\sigma} + \partial \tau$.${}^{36}$
    
    \item Ist $f: X \rightarrow Y$ stetig und $\sigma: I \rightarrow X$, dann gilt $f \circ \tilde{\sigma} = (f \circ \sigma)^{\sim}$.
  \end{enumerate}
  

Beweis. 

Ad (i): Für den konstanten 2-Simplex $\tau: \Delta^2 \rightarrow X, \tau\left(t_0, t_1, t_2\right):=x$, erhalten wir $\partial \tau=\tilde{c}_x-\tilde{c}_x+\tilde{c}_x=\tilde{c}_x$. 

Ad (ii): Für eine Schleife $\sigma: I \rightarrow X$ gilt $\partial \tilde{\sigma}=$ $\sigma(1)-\sigma(0)=0 \in C_0(X)$. 

Ad (iii): Sei also $H: I \times I \rightarrow X$ eine Homotopie relativ Endpunkten von $\sigma_0$ nach $\sigma_1$. Definiere $x_0:=\sigma_0(0)=\sigma_1(0), x_1:=\sigma_0(1)=\sigma_1(1)$, $\rho: I \rightarrow X, \rho(t):=H_t(t)$, sowie $\tau_0, \tau_1: \Delta^2 \rightarrow X, \tau_0\left(t_0, t_1, t_2\right):=H_{t_2}\left(t_1+t_2\right)$, $\tau_1\left(t_0, t_1, t_2\right):=H_{t_1+t_2}\left(t_2\right)$. Dann gilt $\partial \tau_0=\tilde{c}_{x_1}-\tilde{\rho}+\tilde{\sigma}_0$ und $\partial \tau_1=\tilde{\sigma}_1-\tilde{\rho}+\tilde{c}_{x_0}$. Nach (i) existieren $\tau_2, \tau_3 \in C_2(X)$ mit $\partial \tau_2=\tilde{c}_{x_0}$ und $\partial \tau_3=\tilde{c}_{x_1}$. Wir erhalten daher
$$
\tilde{\sigma}_1-\tilde{\sigma}_0=\partial\left(\tau_1-\tau_0-\tau_2+\tau_3\right)
$$
die Behauptung folgt daher mit $\tau:=\tau_1-\tau_0-\tau_2+\tau_3$. 

Ad (iv): Definieren wir $\tau: \Delta^2 \rightarrow X, \tau\left(t_0, t_1, t_2\right):=\left(\sigma_0 \sigma_1\right)\left(t_1 / 2+t_2\right)$, dann folgt $\partial \tau=\tilde{\sigma}_1-\left(\sigma_0 \sigma_1\right)^{\sim}+\tilde{\sigma}_0$. $\operatorname{Ad}(\mathrm{v})$ : Setze $x_0:=\sigma(0)$. Nach (iv) existiert $\tau_1 \in C_2(X)$ mit $(\sigma \bar{\sigma})^{\sim}=\tilde{\sigma}+\bar{\sigma}^{\sim}-\partial \tau$. Da $\sigma \bar{\sigma} \simeq c_{x_0}$ erhalten wir aus (iii) ein $\tau_2 \in C_2(X)$ mit $(\sigma \bar{\sigma})^{\sim}=\tilde{c}_{x_0}+\partial \tau_2$. Nach (i) existiert $\tau_3 \in C_2(X)$ mit $\partial \tau_3=\tilde{c}_{x_0}$. Zusammen erhalten wir
$$
\tilde{\sigma}+\bar{\sigma}^{\sim}=\partial\left(\tau_1+\tau_2+\tau_3\right) .
$$

Behauptung (vi) ist trivial, $(f \circ \tilde{\sigma})\left(t_0, t_1\right)=f\left(\tilde{\sigma}\left(t_0, t_1\right)\right)=f\left(\sigma\left(t_1\right)\right)=(f \circ \sigma)\left(t_1\right)=$ $(f \circ \sigma)^{\sim}\left(t_0, t_1\right)$, für $\left(t_0, t_1\right) \in \Delta^1$.
Nach Lemma IV.11.1(ii) und (iii) ist
$$
h_1=h_1^{\left(X, x_0\right)}: \pi_1\left(X, x_0\right) \rightarrow H_1(X), \quad h_1([\sigma]):=[\tilde{\sigma}] .
$$
eine wohldefinierte Abbildung, sie wird der (erste) Hurewicz-Homomorphismus genannt. Dabei bezeichnet $[\sigma] \in \pi_1\left(X, x_0\right)$ die Homotopieklasse der Schleife $\sigma$ : $I \rightarrow X$ bei $x_0$, und $[\tilde{\sigma}] \in H_1(X)$ die von dem ensprechenden 1-Simplex $\tilde{\sigma}$ : $\Delta^1 \rightarrow X$ repräsenterte Homologieklasse. In Proposition IV.11.2 unten werden wir zeigen, dass dies tatsächlich ein Gruppenhomomorphismus ist.



IV.11.2. Proposition (Hurewicz-Homomorphismus). 

Ist ( $X, x_0$ ) ein punktierter Raum, dann definiert (IV.43) einen Gruppenhomomorphismus. Dieser Homomorphismus ist natürlich, dh. das linke Diagramm
% Linkes Diagramm
\[
\begin{tikzcd}
\pi_1(X, x_0) \arrow[r, "h_1^{(X,x_0)}"] \arrow[d, "f_*"'] & H_1(X) \arrow[d, "f_*"] \\
\pi_1(Y, y_0) \arrow[r, "h_1^{(Y,y_0)}"'] & H_1(Y)
\end{tikzcd}
\]

% Rechtes Diagramm (korrigiert)
\[
\begin{tikzcd}
  & H_1(X) \arrow[dl, "h_1^{(X,x_0)}"'] \arrow[dr, "h_1^{(X,x_1)}"] \\
  \pi_1(X, x_0) \arrow[rr, "\beta_h"', "\cong"{above}] & & \pi_1(X, x_1)
\end{tikzcd}
\]


kommutiert für jede Abbildung punktierter Räume $f:\left(X, x_0\right) \rightarrow\left(Y, y_0\right)$. Für jeden Weg $h: I \rightarrow X$ von $h(0)=x_0$ nach $h(1)=x_1$ ist darüber hinaus das rechte Diagramm oben kommutative, siehe Proposition I.1.18.

Beweis. Sind $\sigma_1, \sigma_2: I \rightarrow X$ zwei Schleifen bei $x_0$, dann folgt aus Lemma IV.11.1(iv)
$$
h_1\left(\left[\sigma_1\right]\left[\sigma_2\right]\right)=h_1\left(\left[\sigma_1 \sigma_2\right]=\left[\left(\sigma_1 \sigma_2\right)^{\sim}\right]=\left[\tilde{\sigma}_1\right]+\left[\tilde{\sigma}_2\right]=h_1\left(\left[\sigma_1\right]\right)+h_1\left(\left[\sigma_2\right]\right),\right.
$$
also ist (IV.43) ein Gruppenhomomorphismus. Ist $f:\left(X, x_0\right) \rightarrow\left(Y, y_0\right)$ eine Abbildung punktierter Räume und $\sigma: I \rightarrow X$ eine Schleife bei $x_0$, dann folgt aus Lemma IV.11.1(vi)
$$
\begin{aligned}
h_1^{\left(Y, y_0\right)}\left(f_*([\sigma])\right)=h_1^{\left(Y, y_0\right)}([f \circ \sigma])=\left[(f \circ \sigma)^{\sim}\right] & \\
& =[f \circ \tilde{\sigma}]=f_*([\tilde{\sigma}])=f_*\left(h_1^{\left(X, x_0\right)}([\sigma])\right)
\end{aligned}
$$
Dies zeigt die Natürlichkeit von $h_1$. Ist nun $\sigma: I \rightarrow X$ eine Schleife bei $x_1$, dann folgt
$$
\begin{aligned}
h_1^{\left(X, x_0\right)}\left(\beta_h([\sigma])\right)=h_1^{\left(X, x_0\right)}([h \sigma \bar{h}]) & =\left[(h \sigma \bar{h})^{\sim}\right] \\
= & {\left[\tilde{h}+\tilde{\sigma}+\bar{h}^{\sim}\right]=[\tilde{h}+\tilde{\sigma}-\tilde{h}]=[\tilde{\sigma}]=h_1^{\left(X, x_0\right)}([\sigma]) }
\end{aligned}
$$
wobei wir Lemma IV.11.1(iv) und (v) verwendet haben.



IV.11.3. Satz (Hurewicz-Isomorphismus). 

Es sei ( $X, x_0$ ) ein wegzusammenhängender punktierter Raum. Dann ist der Hurewicz-Homomorphismus (IV.43) surjektiv und sein Kern stimmt mit der Kommutatoruntergruppe von $\pi_1\left(X, x_0\right)$ überein. Er induziert daher einen Isomorphismus $\pi_1\left(X, x_0\right)_{\mathrm{ab}} \cong H_1(X)$.


Beweis. 

Da $H_1(X)$ abelsch ist, induziert (IV.43) einen Homomorphismus
$$
h_1: \pi_1\left(X, x_0\right)_{\mathrm{ab}} \rightarrow H_1(X)
$$
es genügt zu zeigen, dass (IV.44) ein Isomorphismus ist. Da $X$ wegzusammenhängend ist, können wir zu jedem Punkt $x \in X$ einen Weg $\rho_x: I \rightarrow X$ von $\rho_x(0)=x_0$ nach $\rho_x(1)=x$ wählen. Ist nun $\tilde{\sigma}: \Delta^1 \rightarrow X$ ein 1-Simplex und $\sigma: I \rightarrow X$ der entsprechende Weg, dann ist $\left(\rho_{\sigma(0)} \sigma\right) \bar{\rho}_{\sigma(1)}$ eine Schleife bei $x_0$ und definiert daher ein Element in $\left[\rho_{\sigma(0)} \sigma \bar{\rho}_{\sigma(1)}\right] \in \pi_1\left(X, x_0\right)$. Da $\pi_1\left(X, x_0\right)_{\text {ab }}$ abelsch ist können wir einen Homomorphismus auf Erzeugern $\tilde{\sigma}: \Delta^1 \rightarrow X$ wie folgt definieren:

$$
\phi: C_1(X) \rightarrow \pi_1\left(X, x_0\right)_{\mathrm{ab}}, \quad \phi(\tilde{\sigma}):=\left[\rho_{\sigma(0)} \sigma \bar{\rho}_{\sigma(1)}\right] .
$$


Wir zeigen zunächst

$$
\phi \circ \partial=1: C_2(X) \rightarrow \pi_1\left(X, x_0\right)_{\mathrm{ab}}
$$

dh. $\phi$ definiert einen Homomorphismus

$$
\phi: H_1(X) \rightarrow \pi_1\left(X, x_0\right)_{\mathrm{ab}}, \quad \phi([c]):=\phi(c) .
$$


Für $\tau: \Delta^2 \rightarrow X$ ist also $\phi(\partial \tau)=1$ zu zeigen. ${ }^{37}$ Setzen wir $\tilde{\sigma}_i:=\tau \circ \delta_2^i: \Delta^1 \rightarrow X$, $i=0,1,2$, dann gilt offensichtlich $\partial \tau=\tilde{\sigma}_0-\tilde{\sigma}_1+\tilde{\sigma}_2$. Da $\phi$ ein Homomorphismus ist, erhalten wir:

$$
\begin{aligned}
\phi(\partial \tau) & =\phi\left(\tilde{\sigma}_0\right) \phi\left(\tilde{\sigma}_1\right)^{-1} \phi\left(\tilde{\sigma}_2\right) \\
& =\left[\rho_{\sigma_0(0)} \sigma_0 \bar{\rho}_{\sigma_0(1)}\right]\left[\rho_{\sigma_1(0)} \sigma_1 \bar{\rho}_{\sigma_1(1)}\right]^{-1}\left[\rho_{\sigma_2(0)} \sigma_2 \bar{\rho}_{\sigma_2(1)}\right] \\
& =\left[\rho_{\sigma_0(0)} \sigma_0 \bar{\rho}_{\sigma_0(1)} \rho_{\sigma_1(1)} \bar{\sigma}_1 \bar{\rho}_{\sigma_1(0)} \rho_{\sigma_2(0)} \sigma_2 \bar{\rho}_{\sigma_2(1)}\right] \\
& =\left[\rho_{\sigma_0(0)} \sigma_0 \bar{\sigma}_1 \sigma_2 \bar{\rho}_{\sigma_2(1)}\right] \\
& =\left[\rho_{\sigma_0(0)} \bar{\rho}_{\sigma_2(1)}\right]=\left[c_{x_0}\right]=1
\end{aligned}
$$


Dabei haben wir verwendet, dass $\sigma_0 \bar{\sigma}_1 \sigma_2, \bar{\rho}_{\sigma_0(1)} \rho_{\sigma_1(1)}, \bar{\rho}_{\sigma_1(0)} \rho_{\sigma_2(0)}$ und $\rho_{\sigma_0(0)} \bar{\rho}_{\sigma_2(1)}$ nullhomotope Schleifen sind. Damit ist (IV.45) gezeigt. Es genügt nun zu zeigen, dass (IV.46) invers zu (IV.44) ist. Zunächst gilt

$$
\phi \circ h_1=\operatorname{id}_{\pi_1\left(X, x_0\right)_{\mathrm{ab}}}
$$

denn für jede Schleife $\sigma: I \rightarrow X$ bei $x_0$ gilt

$$
\phi\left(h_1([\sigma])\right)=\phi([\tilde{\sigma}])=\phi(\tilde{\sigma})=\left[\rho_{x_0} \sigma \bar{\rho}_{x_0}\right]=\left[\rho_{x_0}\right][\tilde{\sigma}]\left[\rho_{x_0}\right]^{-1}=[\sigma]
$$


Es bleibt daher nur noch

$$
h_1 \circ \phi=\operatorname{id}_{H_1(X)}
$$

zu zeigen. Um dies einzusehen definieren wir einen Homomorphismus auf Erzeugern $x \in X$ durch

$$
g: C_0(X) \rightarrow C_1(X), \quad g(x):=\tilde{\rho}_x
$$


Für jeden 1-Simplex $\tilde{\sigma}: \Delta^1 \rightarrow X$ gilt dann

$$
\begin{aligned}
h_1(\phi(\tilde{\sigma}))= & h_1\left(\left[\rho_{\sigma(0)} \sigma \bar{\rho}_{\sigma(1)}\right]\right)=\left[\left(\rho_{\sigma(0)} \sigma \bar{\rho}_{\sigma(1)}\right)^{\sim}\right] \\
& =\left[\tilde{\rho}_{\sigma(0)}+\tilde{\sigma}-\tilde{\rho}_{\sigma(1)}\right]=[\tilde{\sigma}-g(\partial \tilde{\sigma})] .
\end{aligned}
$$


Dabei haben wir Lemma IV.11.1(iv) und (v) verwendet. Es folgt sofort $h_1(\phi(c))=$ $[c-g(\partial c)]$ für alle $c \in C_1(X)$, also $h_1(\phi(c))=[c]$, für alle Zyklen $c \in Z_1(X)$. Damit ist (IV.47) gezeigt und der Beweis vollständig.








\pagebreak
\printbibliography
\end{document}