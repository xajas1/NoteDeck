\documentclass[10pt]{article}
\usepackage[ngerman]{babel}
\usepackage[utf8]{inputenc}
\usepackage[T1]{fontenc}
\usepackage{amsmath}
\usepackage{amsfonts}
\usepackage{amssymb}
\usepackage[version=4]{mhchem}
\usepackage{stmaryrd}

%New command to display footnote whose markers will always be hidden
\let\svthefootnote\thefootnote
\newcommand\blfootnotetext[1]{%
  \let\thefootnote\relax\footnote{#1}%
  \addtocounter{footnote}{-1}%
  \let\thefootnote\svthefootnote%
}

%Overriding the \footnotetext command to hide the marker if its value is `0`
\let\svfootnotetext\footnotetext
\renewcommand\footnotetext[2][?]{%
  \if\relax#1\relax%
    \ifnum\value{footnote}=0\blfootnotetext{#2}\else\svfootnotetext{#2}\fi%
  \else%
    \if?#1\ifnum\value{footnote}=0\blfootnotetext{#2}\else\svfootnotetext{#2}\fi%
    \else\svfootnotetext[#1]{#2}\fi%
  \fi
}

\begin{document}
\section*{Chapter 6}
\section*{Der Spin der Elektronen}
\subsection*{Quantenmechanische Beschreibung}
Die Existenz des Spins bedeutet, daß Elektronen, außer der Ortskoordinate $\vec{x}$, oder der Impulskoordinate $\vec{p}$, einen weiteren Freiheitsgrad besitzen: Spin nach "oben" und Spin nach "unten", jeweils bezüglich einer vorgegebenen Richtung (Quantisierungsachse, z.B. die $x_{3}$-Richtung). Und dieses obwohl Elektronen Punktteilchen sind.

\section*{Spinore}
Man verdoppelt die Wellenfunktion $\psi(\vec{x}, t)$ zu einem Spinor $\tilde{\psi}(\vec{x}, t)$,

$$
\psi(\vec{x}, t) \rightarrow \quad \tilde{\psi}(\vec{x}, t)=\binom{\psi_{+}(\vec{x}, t)}{\psi_{-}(\vec{x}, t)}
$$

wobei $\psi_{+}(\vec{x}, t)$ ein Elektron mit Spin "oben" und $\psi_{-}(\vec{x}, t)$ ein Elektron mit Spin "unten" beschreibt.

\section*{Pauli-Matrizen}
Der Spin-Operator $\overrightarrow{\mathbf{S}}$ ist nach Anbschnitt ?? durch die Pauli-Matrizen $\sigma_{i}$ gegeben,

$$
\begin{aligned}
\overrightarrow{\mathbf{S}} & =\frac{\hbar}{2}\left(\sigma_{1}, \sigma_{2}, \sigma_{3}\right) \\
\sigma_{1} & =\left(\begin{array}{cc}
0 & 1 \\
1 & 0
\end{array}\right), \quad \sigma_{2}=\left(\begin{array}{cc}
0 & -i \\
i & 0
\end{array}\right), \quad \sigma_{3}=\left(\begin{array}{cc}
1 & 0 \\
0 & -1
\end{array}\right)
\end{aligned}
$$

Bis auf den Faktor $\hbar / 2$ folgen die Pauli Matrizen den Kommuationsrelationen von Drehimpulsoperatoren,

$$
\left[\sigma_{k}, \sigma_{l}\right]=2 i \epsilon_{k l m} \sigma_{m}, \quad \sigma_{j}^{2}=1, \quad j=1,2,3
$$

\section*{Basiswahl}
Bezeichen wir mit $\tilde{\psi}_{ \pm}$die Zustände mit Spin rauf/runter,

$$
\tilde{\psi}_{+}=\binom{1}{0}, \quad \tilde{\psi}_{-}=\binom{0}{1}
$$

so gilt erwartungsgemäß:

$$
\mathbf{S}_{3} \tilde{\psi}_{+}=\frac{\hbar}{2}\left(\begin{array}{cc}
1 & 0 \\
0 & -1
\end{array}\right)\binom{1}{0}=\frac{\hbar}{2}\binom{1}{0}, \quad \mathbf{S}_{3} \tilde{\psi}_{-}=-\frac{\hbar}{2}\binom{0}{1}
$$

Für festes $\vec{x}$ und $t$ spannen $\tilde{\psi}_{+}$und $\tilde{\psi}_{-}$einen 2-dimionalen Vektorraum auf, dessen Elemente als Spinoren bezeichnet werden.

\section*{Wellenfunktionen}
Ein allgemeines Element des Vektorraums hat komplexen Koeffizienten $c_{+}$und $c_{-}$, die orts- und zeitabhängig sind:

$$
\tilde{\psi}(\vec{x}, t)=c_{+}(\vec{x}, t) \tilde{\psi}_{+}+c_{-}(\vec{x}, t) \tilde{\psi}_{-}=\binom{c_{+}(\vec{x}, t)}{c_{-}(\vec{x}, t)}
$$

Die Entwicklungskoeffizienten $c_{ \pm}(\vec{x}, t)$ entsprechen also Wellenfunktionen $\psi_{ \pm}(\vec{x}, t)$, von denen es pro Elektron nun zwei gibt. Die Norm von $\tilde{\psi}(\vec{x}, t)$ ist durch

$$
(\tilde{\psi}, \tilde{\psi})=\left|c_{+}\right|^{2}+\left|c_{-}\right|^{2}
$$

gegeben. Wegen der physikalischen Interpretation muss $(\tilde{\psi}, \tilde{\psi})=1$ sein. $\left|c_{ \pm}\right|^{2}$ ist damit die Wahrscheinlichkeit dafür, daß ein Elektron im Zustand $\psi$ den Spin parallel/antiparallel zur $x_{3}$-Achse ausgerichtet hat, mit $\left|c_{+}\right|^{2}+\left|c_{-}\right|^{2}=1$.

\section*{Erwartungswerte}
Im folgenden wird die Ortsabhängigkeit von $\tilde{\psi}$ ignoriert und zunächst nur Spineigenschaften betrachtet. Für die Erwartungswerte $\left\langle\mathbf{S}_{j}\right\rangle$ der Komponenten $\mathbf{S}_{j}$ im Zustand $\tilde{\psi}$ ergibt sich

$$
<\mathbf{S}_{1}>=\frac{\hbar}{2}\left(c_{+}^{*}, c_{-}^{*}\right)\left(\begin{array}{ll}
0 & 1 \\
1 & 0
\end{array}\right)\binom{c_{+}}{c_{-}}=\frac{\hbar}{2}\left(c_{+}^{*} c_{-}+c_{-}^{*} c_{+}\right),
$$

und analog

$$
\begin{aligned}
& <\mathbf{S}_{2}>=-\frac{i \hbar}{2}\left(c_{+}^{*} c_{-}-c_{-}^{*} c_{+}\right) \\
& <\mathbf{S}_{3}>=\frac{\hbar}{2}\left(\left|c_{+}\right|^{2}-\left|c_{-}\right|^{2}\right)
\end{aligned}
$$

Als Observable sind die Erwartungswerte reel.

\subsection*{Drehungen von Spins}
Im Abschnitt ?? wurde gezeigt, daß Wellenfunktionen via

$$
\psi(R \vec{x})=e^{-i \overrightarrow{\mathbf{L}} \cdot \vec{\varphi} / \hbar} \psi(\vec{x})
$$

gedreht werden. Dieses gilt für ganzzahligen Drehimpuls $j=0,1,2 \ldots$ Für $j=1 / 2$ ist der Drehimpulsoperatoren $\vec{L}$ durch den Spin-Operator $\vec{S}$ zu ersetzen,

$$
R \tilde{\psi}=e^{-i \overrightarrow{\mathbf{S}} \cdot \vec{\varphi} / \hbar} \tilde{\psi}
$$

wobei $\tilde{\psi}$ ein Spinor ist.

\section*{Drehung um die z-Achse}
Als Beispiel betrachten wir eine Drehung um die 3-Achse, d.h. $\vec{\varphi}=(0,0, \varphi)$ :

$$
\begin{aligned}
e^{-i \mathbf{S}_{3} \varphi / \hbar} & =\sum_{n=0}^{\infty} \frac{(-i \varphi / 2)^{n}}{n!}\left(\begin{array}{cc}
1 & 0 \\
0 & -1
\end{array}\right)^{n} \\
& =\sum_{l=0}^{\infty} \frac{(-i \varphi / 2)^{(2 l)}}{(2 l)!}\left(\begin{array}{ll}
1 & 0 \\
0 & 1
\end{array}\right)+\sum_{l=0}^{\infty} \frac{(-i \varphi / 2)^{(2 l+1)}}{(2 l+1)!}\left(\begin{array}{cc}
1 & 0 \\
0 & -1
\end{array}\right) \\
& =\cos (\varphi / 2)\left(\begin{array}{ll}
1 & 0 \\
0 & 1
\end{array}\right)-i \sin (\varphi / 2)\left(\begin{array}{cc}
1 & 0 \\
0 & -1
\end{array}\right)=\left(\begin{array}{cc}
e^{-i \varphi / 2} & 0 \\
0 & e^{i \varphi / 2}
\end{array}\right)
\end{aligned}
$$

Spin nach oben und unten erhalten also entgegengesetzte Phasen.

$$
\text { Bei einer Drehung um } 2 \pi \text { erhalten Spinoren die Phase }(-1) \text {. }
$$

Um einen Spin in den Ausgangszustand überzuführen bedarf es also einer Drehung um $4 \pi$.

\subsection*{Das magnetische Moment des Elektrons}
\section*{Gyromagnetischer Faktor}
Ein Elektron mit Spin ist ein rotierendes (geladenes) Teilchen. Aus der Elektrodynamik wissen wir, daß ein Ringstrom mit Drehimpuls $\vec{L}=\vec{S}$ ein magnetisches Moment der Grösse

$$
\begin{aligned}
\vec{\mu} & =-\frac{e_{0} g}{2 m_{e}} \overrightarrow{\mathbf{S}} & & \text { mit } \\
g & \approx 2 & & \text { (in guter Näherung) }
\end{aligned}
$$

erzeugt. Der $g$-Faktor heißt gyromagnetischer Faktor. Für klassische Ringströme gilt $g=1$, siehe auch Abschnitt ??.

\section*{Elektron im Magnetfeld}
In einem äußeren Feld $\vec{B}$ hat ein klassisches magnetisches Moment $\vec{\mu}$ die Energie $-\vec{\mu} \cdot \vec{B}$. Nach dem Korrespondenzprinzip führt dies zum Hamilton-Operator

$$
\mathbf{H}=\frac{e_{0} g \hbar}{4 m_{e}} \vec{\sigma} \cdot \vec{B}
$$

Für einen zeitabhängigen Spinor $\tilde{\psi}(t)=\binom{c_{+}(\vec{x}, t)}{c_{-}(\vec{x}, t)}$ erhalten wir die zeithabhängige Schrödinger-Gleichung

$$
i \hbar \frac{d}{d t} \tilde{\psi}(t)=\frac{e_{0} g \hbar}{4 m_{e}}(\vec{\sigma} \cdot \vec{B}) \tilde{\psi}(t)
$$

\section*{Lamor-Frequenz}
Wir betrachten ein konstantes Magnetfeld $\vec{B}=\left(0,0, B_{0}\right)$ und lösen die SchrödingerGleichung mittels des Ansatzes $\tilde{\psi}(t)=e^{-i \omega t}\binom{c_{+}}{c_{-}}$, mit $c_{ \pm}=$const., also

$$
\hbar \omega\binom{c_{+}}{c_{-}}=\frac{e g \hbar B_{0}}{4 m_{e}}\left(\begin{array}{cc}
1 & 0 \\
0 & -1
\end{array}\right)\binom{c_{+}}{c_{-}} .
$$

Die beiden Eigenfrequenzen $\omega_{ \pm}$sind

$$
\omega_{+}=\frac{e_{0} g B_{0}}{4 m_{e}} \equiv \omega_{L} \quad \text { und } \quad \omega_{-}=-\omega_{L}
$$

mit Eigenvektoren sind $(1,0)$ und $(0,1)$. Hier is $\omega_{L}$ die Lamor-Frequenz. ${ }^{1}$ Für einen allgemeinen Anfangszustand $\tilde{\psi}(t=0)=(a, b)$ findet man daher

$$
\tilde{\psi}(t)=\binom{a e^{-i \omega_{L} t}}{b e^{i \omega_{L} t}}, \quad \omega_{L}=\frac{e_{0} g B}{4 m_{e}}, \quad|a|^{2}+|b|^{2}=1
$$

\section*{Präzession}
Als Beispiel betrachten wir einen Anfangszustand, in welchem der Spin entlang der 1Achse ausgericht ist, also senkrecht zum angelegten Magnetfeld.

Als Erstes müssen wir den Eigenvektor $(a, b)$ zu $\mathbf{S}_{1}$ (und zum Eigenwert $\hbar / 2$ ) finden:

$$
\frac{1}{2} \hbar\left(\begin{array}{ll}
0 & 1 \\
1 & 0
\end{array}\right)\binom{a}{b}=\frac{1}{2} \hbar\binom{a}{b}, \quad\binom{a}{b}=\frac{1}{\sqrt{2}}\binom{1}{1} .
$$

\footnotetext{${ }^{1}$ Die hier definierte Larmor-Frequenze gilt für Spin-1/2. Klassisch benutzt man $q g B /(2 m)$. Unter Einberechnung der $g$-Faktoren erhält man sehr ähnliche Werte.
}Wir berechnen nun den Zeit-abhängigen Erwartungswert,

$$
\begin{aligned}
\left(\tilde{\psi}(t), \mathbf{S}_{1} \tilde{\psi}(t)\right) & =<\mathbf{S}_{1}>(t) \\
& =\frac{\hbar}{2} \frac{1}{\sqrt{2}}\left(e^{i \omega_{L} t}, e^{-i \omega_{L} t}\right)\left(\begin{array}{cc}
0 & 1 \\
1 & 0
\end{array}\right)\binom{e^{-i \omega_{L} t}}{e^{i \omega_{L} t}} \frac{1}{\sqrt{2}} \\
& =\frac{\hbar}{2} \cos 2 \omega_{L} t
\end{aligned}
$$

Analog ergibt sich

$$
\begin{aligned}
<\mathbf{S}_{1}>(t) & =\frac{\hbar}{2} \cos 2 \omega_{L} t \\
<\mathbf{S}_{2}>(t) & =\frac{\hbar}{2} \sin 2 \omega_{L} t \\
<\mathbf{S}_{3}>(t) & =0
\end{aligned}
$$

d.h. der Spin "präzediert" mit der doppelten Lamor-Frequenz um die Richtung von $\vec{B}$.

\subsection*{Paramagnetische Resonanz}
Der $g$-Faktor in der Beziehung

$$
\vec{\mu}=\frac{q g}{2 m} \overrightarrow{\mathbf{S}}, \quad \overrightarrow{\mathbf{S}}=\frac{1}{2} \hbar \vec{\sigma}, \quad q: \text { Ladung }
$$

ist im Festkörper keine universelle Konstante, sondern hängt von der chemischen Umgebung ab (via der Spin-Bahn Kopplung, siehe Abschnitt ??). Eine Methode die Größe des $g$-Faktors experimentell zu bestimmen ist die paramagnetische Resonanz.

\section*{RF-Felder}
Analog zu der Induktion von atomaren Übergängen zwischen verschiedenen Niveaus durch Einstrahlen von Licht, kann man Übergänge zwischen den beiden Energieniveaus $\pm \hbar \omega_{L}=$ $\hbar g|q| B_{0} /(4 m)$ eines Spins in einem konstanten Magnetfeld $B_{0}$ induzieren. Dafür benötigt man ein zusätzlich oszillierendes Magnetfeld $B$, typischerweise im Radiofrequenz (RF) Bereich.

Es sei also

$$
\vec{B}=\left(B \cos \omega t, B \sin \omega t, B_{0}\right)
$$

Mit

$$
\vec{\sigma} \cdot \vec{B}=\left(\begin{array}{cc}
B_{0}, & B(\cos \omega t-i \sin \omega t) \\
B(\cos \omega t+i \sin \omega t), & -B_{0}
\end{array}\right)
$$

lautet die Schrödinger-Gleichung (mit $q=-e_{0}$ )

$$
i \hbar \frac{d \tilde{\psi}}{d t}=\frac{\hbar q g}{4 m}\left(\begin{array}{cc}
B_{0}, & B e^{-i \omega t} \\
B e^{i \omega t}, & -B_{0}
\end{array}\right) \tilde{\psi}(t)
$$

\section*{Variation der Konstanten}
Die Energie ist nicht erhalten, da $\mathbf{H}=\mathbf{H}(t)=\hbar \omega_{g} \vec{\sigma} \cdot \vec{B}(t)$ explizit von der Zeit abhängt, analog zu erzwungenen Schwingungen in der Mechanik.

Der Ansatz (Variation der Konstanten)

$$
\tilde{\psi}(t)=\binom{a(t) e^{-i \omega_{L} t}}{b(t) e^{i \omega_{L} t}}, \quad \frac{d \tilde{\psi}}{d t}=\binom{\left(\dot{a}-i \omega_{L} a\right) e^{-i \omega_{L} t}}{\left(\dot{b}+i \omega_{L} b\right) e^{i \omega_{L} t}}
$$

führt zu den Gleichungen

$$
\begin{aligned}
\dot{a} & =-i \omega_{g} e^{i\left(2 \omega_{L}-\omega\right) t} b(t) \\
\dot{b} & =-i \omega_{g} e^{-i\left(2 \omega_{L}-\omega\right) t} a(t) \\
\omega_{g} & =\left(g e_{0} B\right) /(4 m)
\end{aligned}
$$

Differenzieren der 1. Gleichung nach $t$ und Einsetzen der zweiten ergibt

$$
\ddot{a}-i\left(2 \omega_{L}-\omega\right) \dot{a}+\omega_{g}^{2} a=0
$$

Der Ansatz $a(t)=A e^{i \lambda t}$ führt zu einer quadratischen Gleichung für $\lambda$, mit den Lösungen

$$
\lambda_{1,2}=\omega_{L}-\frac{1}{2} \omega \pm \sqrt{\left(\omega_{L}-\frac{1}{2} \omega\right)^{2}+\omega_{g}^{2}}
$$

Die Bewegung des Systems, d.h. des Spinors $\tilde{\psi}(t)$ wird also durch die äussere Frequenz $\omega$ moduliert.

\section*{Induzierte Übergänge}
Wir preparieren das System zur Zeit $t=0$ in den Spin-oben Zustand: $a(0)=1$ und $b(0)=0$, was auch $\dot{a}(0)=0$ bedeutet. Für allg. Zeiten gilt

$$
\begin{aligned}
a(t) & =\left[\cos (\widehat{\omega} t)-i \frac{\omega_{L}-\omega / 2}{\widehat{\omega}} \sin \widehat{\omega} t\right] e^{i\left(\omega_{L}-\omega / 2\right) t} \\
b(t) & =-i \frac{\omega_{g}}{\widehat{\omega}} \sin \widehat{\omega} t e^{-i\left(\omega_{L}-\omega / 2\right) t} \\
\widehat{\omega} & =\sqrt{\left(\omega_{L}-\frac{1}{2} \omega\right)^{2}+\omega_{g}^{2}}
\end{aligned}
$$

Ist $T$ die Zeitspanne, während der das RF-Magnetfeld eingeschaltet ist, so ist am Ende dieser Zeitspanne der Bruchteil $|b(T)|^{2}$ der Spins "umgeklappt":

$$
|b(T)|^{2}=\frac{\omega_{g}^{2}}{\left(\omega_{L}-\frac{\omega}{2}\right)^{2}+\omega_{g}^{2}} \sin ^{2}\left(T \sqrt{\left(\omega_{L}-\frac{\omega}{2}\right)^{2}+\omega_{g}^{2}}\right)
$$

man hat mittels des RF-Feldes Übergänge zwischen den beiden Zeeman-Niveaus $\pm \omega_{L}$ erzeugt.

\section*{Resonanz}
Resonanz liegt vor, falls die Frequenz des RF-Feldes $\omega=2 \omega_{L}$ ist, also genau der Energiedifferenz der beiden Energieniveaus entspricht. Bei Resonanz ist es möglich, alle Spins umzudrehen, $|b(T)|=1$. Man wähle hierfür eine Einschaltzeit $T=\pi /\left(2 \omega_{g}\right)$, da dann $\sin ^{2}=1$ ( $\pi$-Puls).

Die obige "paramagnetische" Resonanzmethode hat viele Anwendungen in der Atomund Kernphysik, der Festkörperphsik (NMR, $\mu$ SR) sowie auch in der Medizin.


\end{document}