\begin{DEF}{EFT1.EFT1-1-02-1}{Lokale triviale Faserung mit typischen Fasern auf Mfk}
Seien $E, M$ und $F$ differenzierbare Mannigfaltikeiten und $\pi: E \rightarrow M$ eine differenzierbare Abbildung. Dann heißt $(E, \pi, M)$ eine lokal triviale Faserung mit typischer Faser $F$, wenn es zu jedem $x \in M$ eine offene Umgebung $U$ gibt und einen Diffeomorphismus $\varphi: \pi^{-1}(U):=E \mid U \rightarrow$ $U \times F$, sodass

    $$
    \begin{array}{lll}
    E \mid U & \xrightarrow{\varphi} & U \times F \\
    \pi \searrow & & \swarrow p r_1 \\
    & U &
    \end{array}
    $$
    
    kommutiert. Man spricht auch von der lokal trivialen Faserung $E \rightarrow M$ oder $E$.
\end{DEF}
