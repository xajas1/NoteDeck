\documentclass[10pt, letterpaper]{article}

% Inhaltsverzeichnis für Pakettypen (nur für Übersicht im Header, wird nicht im Dokument angezeigt)
% 1. Seitenlayout und Ränder
% 2. Sprache und Zeichensatz
% 3. Mathematik und Theorem-Umgebungen
% 4. Eigene Makros
% 5. Diagramme und Grafiken
% 6. Tabellen und Aufzählungen
% 7. Inhaltsverzeichnis
% 8. Abschnittsüberschriften
% 9. Abstrakt-Umgebung
% 10. Todos/Notizen
% 11. Rahmen/Box-Umgebungen
% 12. Python-Integration
% 13. Literaturverwaltung
% 14. Hyperlinks
% 15. Absatzeinstellungen
% 16. Umgebungen
% 17. Titel und Autor

% --- 1. Seitenlayout und Ränder ---
\usepackage[margin=3cm]{geometry}

% --- 2. Sprache und Zeichensatz ---
\usepackage[english]{babel}
\usepackage[T1]{fontenc}
\usepackage[utf8]{inputenc}

% --- 3. Mathematik und Theorem-Umgebungen ---
\usepackage{amsmath, amssymb, amsthm}
\usepackage{mathrsfs}
\DeclareMathOperator{\WF}{WF}

% --- 4. Eigene Makros ---
\usepackage{xcolor}
\newcommand{\SKP}{\langle\cdot,\cdot\rangle}
\newcommand{\R}{\mathbb{R}}
\newcommand{\N}{\mathbb{N}}
\newcommand{\Q}{\mathbb{Q}}
\newcommand{\Z}{\mathbb{Z}}
\newcommand{\C}{\mathbb{C}}
\newcommand{\entwurf}[1]{\textcolor{red}{#1}}

% --- 5. Diagramme und Grafiken ---
\usepackage{graphicx}
\usepackage{tikz}
\usetikzlibrary{decorations.pathreplacing, arrows.meta, positioning}
\usepackage{tikz-cd}

% --- 6. Tabellen und Aufzählungen ---
\usepackage{enumitem}
\setlist[itemize]{left=0.5cm}

\newenvironment{romanenum}[1][]
  {%
    \ifx&#1&
    \else
      \textbf{#1}\quad
    \fi
    \begin{enumerate}[label=\roman*)]
  }
  {%
    \end{enumerate}%
  }

% --- 7. Inhaltsverzeichnis ---
\usepackage{tocloft}
\renewcommand{\cftsecfont}{\footnotesize}
\renewcommand{\cftsubsecfont}{\footnotesize}
\renewcommand{\cftsubsubsecfont}{\footnotesize}
\renewcommand{\cftsecpagefont}{\footnotesize}
\renewcommand{\cftsubsecpagefont}{\footnotesize}
\renewcommand{\cftsubsubsecpagefont}{\footnotesize}
\usepackage{etoc}

% --- 8. Abschnittsüberschriften ---
\usepackage{titlesec}
\titleformat{\section}{\normalfont\large\bfseries}{\thesection}{1em}{}
\titleformat{\subsection}{\normalfont\normalsize\bfseries}{\thesubsection}{0.5em}{}
\titleformat{\subsubsection}{\normalfont\normalsize\bfseries}{\thesubsubsection}{0.5em}{}
\setcounter{secnumdepth}{4}

% --- 9. Abstrakt-Umgebung ---
\usepackage{changepage}
\renewenvironment{abstract}
  {
    \begin{adjustwidth}{1.5cm}{1.5cm}
    \small
    \textsc{Abstract. –}%
  }
  {
    \end{adjustwidth}
  }

% --- 10. Todos/Notizen ---
\usepackage{todonotes}

% --- 11. Rahmen/Box-Umgebungen ---
\usepackage{mdframed}
\usepackage{tcolorbox}
\colorlet{shadecolor}{gray!25}

\newenvironment{customTheorem}
  {\vspace{10pt}%
   \begin{mdframed}[
     backgroundcolor=gray!20,
     linewidth=0pt,
     innertopmargin=10pt,
     innerbottommargin=10pt,
     skipabove=\dimexpr\topsep+\ht\strutbox\relax,
     skipbelow=\topsep,
   ]}
  {\end{mdframed}
   \vspace{10pt}%
  }

% --- 12. Python-Integration ---
% (Deaktiviert in dieser Version, aktiviere bei Bedarf)
% \usepackage{pythontex}
% \usepackage[makestderr]{pythontex}

% --- 13. Literaturverwaltung ---
\usepackage{csquotes}
\usepackage[backend=biber, style=alphabetic, citestyle=alphabetic]{biblatex}
\addbibresource{bibliography.bib}

% --- 14. Hyperlinks ---
\usepackage{hyperref}
\hypersetup{
  colorlinks   = true,
  urlcolor     = blue,
  linkcolor    = blue,
  citecolor    = blue,
  frenchlinks  = true
}

% --- 15. Absatzeinstellungen ---
\usepackage[parfill]{parskip}
\sloppy

% --- 16. Umgebungen ---
\usepackage{thmtools}

\newcommand{\CustomHeading}[3]{%
  \par\medskip\noindent%
  \textbf{#1 #2} \textnormal{(#3)}.\enskip%
}

\newenvironment{DEF}[2]{\CustomHeading{Definition}{#1}{#2}}{}
\newenvironment{PROP}[2]{\CustomHeading{Proposition}{#1}{#2}}{}
\newenvironment{THEO}[2]{\CustomHeading{Theorem}{#1}{#2}}{}
\newenvironment{LEM}[2]{\CustomHeading{Lemma}{#1}{#2}}{}
\newenvironment{KORO}[2]{\CustomHeading{Corollar}{#1}{#2}}{}
\newenvironment{REM}[2]{\CustomHeading{Remark}{#1}{#2}}{}
\newenvironment{EXA}[2]{\CustomHeading{Example}{#1}{#2}}{}
\newenvironment{STUD}[2]{\CustomHeading{Study}{#1}{#2}}{}
\newenvironment{CONC}[2]{\CustomHeading{Concept}{#1}{#2}}{}

\newenvironment{PROOF}
  {\begin{proof}}%
{\end{proof}}

% --- 17. Titel und Autor ---
\title{Algebra}
\author{Tim Jaschik}
\date{\today}

\begin{document}

\maketitle
\rule{\textwidth}{0.5pt}
\begin{abstract}
…
\end{abstract}
\rule{\textwidth}{0.5pt}
\vspace{0.5cm}

\tableofcontents

\pagebreak






\section{Ringe}




\section{Ringe Basics}

\subsection{Definition}

\begin{DEF}{A-1-03-16}{Ringhomomorphismus}

\end{DEF}

\begin{DEF}{A-1-03-2}{Ring ohne Eins}

\end{DEF}

\begin{DEF}{A-1-03-21}{R-Linearkombination in Ringen}

\end{DEF}

\begin{DEF}{A-1-03-22}{Unterring eines Ringes}

\end{DEF}

\begin{DEF}{A-1-03-24}{Einheiten in Ringen}

\end{DEF}

\begin{DEF}{A-1-03-3}{Kommutativer Ring}

\end{DEF}

\begin{DEF}{A-1-03-30}{Schiefkörper als Ring mit Einheitsgruppe R ohne $0$}

\end{DEF}

\begin{DEF}{A-1-03-31}{Körper als abelscher Schiefkörper}

\end{DEF}

















\subsection{Example}

\begin{EXA}{A-1-03-18}{Pullback-Ringhomomorphismus}

\end{EXA}

\begin{EXA}{A-1-03-19}{Einschränkung als Pullback der Inklusion}

\end{EXA}

\begin{EXA}{A-1-03-20}{Auswertungshomomorphismus für Punkt-Inklusion}

\end{EXA}

\begin{EXA}{A-1-03-23}{Bild von Ringhomomorphismen ist ein Unterring}

\end{EXA}

\begin{EXA}{A-1-03-26}{Einheitengruppe von ganzen Zahlen}

\end{EXA}

\begin{EXA}{A-1-03-27}{Einheitengruppe von Gruppenringe}

\end{EXA}

\begin{EXA}{A-1-03-28}{Einheiten von Matrizenringe mit Koeffizienten in Körper}

\end{EXA}

\begin{EXA}{A-1-03-32}{Quaternionen als nichtkommutativer Schiefkörper}

\end{EXA}

\begin{EXA}{A-1-03-4}{Körper sind Ringe}

\end{EXA}

\begin{EXA}{A-1-03-5}{$(\mathbb{Z},+,*)$ kommutaiver Ring}

\end{EXA}

\begin{EXA}{A-1-03-6}{Ring der Funktionen}

\end{EXA}

\begin{EXA}{A-1-03-7}{Matrizenringe über Körper}

\end{EXA}

\begin{EXA}{A-1-03-8}{$(End_k(V),+,\circ)$ Ring}

\end{EXA}

\begin{EXA}{A-1-03-9}{Matrizenring über Ring}

\end{EXA}

































\subsection{Proposition}



\begin{PROP}{A-1-03-25}{Einheitsgruppe: Menge der Einheiten in Ringen sind Gruppe bzgl. Multiplikation in R}

\end{PROP}

\begin{PROP}{A-1-03-29}{Ringhomomorphismen bilden Einheiten auf Einheiten ab und induzieren G-Hom auf Einheitsgruppen}

\end{PROP}



\subsection{Lemma}



\begin{LEM}{A-1-03-14}{Rechenregeln für Ringe mit Eins}

\end{LEM}

\begin{LEM}{A-1-03-15}{Wenn Ring mit $0=1$, dann Nullring}

\end{LEM}



\subsection{Remark}



\begin{REM}{A-1-03-13}{Eins eines Ringes mit Eins ist eindeutig}

\end{REM}

\begin{REM}{A-1-03-17}{Ringhomomorphismen induzieren Gruppenhomomorphismen zwischen abelschen Gruppen}

\end{REM}







\begin{EXA}{A-1-03-10}{Nullring}

\end{EXA}

\begin{EXA}{A-1-03-11}{Produktring}

\end{EXA}

\begin{EXA}{A-1-03-12}{Gruppenring mit Koeffizienten aus Körper}

\end{EXA}









\begin{DEF}{A-1-03-1}{Ring mit Eins}

\end{DEF}






\section{Potenzreihenringe und Polynomringe}

\subsection{Definition}

\begin{DEF}{A-1-04-15}{Symmetrisches Polynom}

\end{DEF}

\begin{DEF}{A-1-04-2}{Polynomring mit Koeffizienten in Ring als Unterring von Potenzreihenring}

\end{DEF}







\subsection{Example}



\begin{EXA}{A-1-04-16}{Elementarsymmetrische Polynom in n-Variablen}

\end{EXA}

\begin{EXA}{A-1-04-8}{Auswertungshomomorphismus für Abbildung von Körper in Matrzenring }

\end{EXA}

\begin{EXA}{A-1-04-9}{Auswertungshomomorphismus für Abbildung von Körper in Abbildungsring der $End_V$ }

\end{EXA}







\subsection{Proposition}



\begin{PROP}{A-1-04-17}{Vieta-Formel}

\end{PROP}

\begin{PROP}{A-1-04-18}{Jedes symmetrische Polynom ist ein Polynom in den elementarsymmetrischen Polynomen}

\end{PROP}

\begin{PROP}{A-1-04-7}{Universelle Eigenschaft des Polynomringes: Auswertungs-Ringhomomorphismus}

\end{PROP}





\subsection{Lemma}

\begin{LEM}{A-1-04-14}{Gruppenhomomorphismus zwischen Symmetrische Gruppe und Gruppe der Ring-Automorphismen des Polynomringes in n-Variablen}

\end{LEM}



\subsection{Remark}



\begin{REM}{A-1-04-12}{Multiindex-Schreibweise}

\end{REM}

\begin{REM}{A-1-04-13}{Induzierter Ringautomorphismus auf Polynomring durch Permutation}

\end{REM}

\begin{REM}{A-1-04-3}{Eindeutige Darstellung in Polynomringen}

\end{REM}

\begin{REM}{A-1-04-4}{Polynomring als Unterring der R-Linearkombinationen}

\end{REM}

\begin{REM}{A-1-04-5}{Eigenschaften der Gradfunktion von Leitkoeffizienten}

\end{REM}

\begin{REM}{A-1-04-6}{Identifikation von R als Unterring von Polynomring mit Koeff in R}

\end{REM}















\begin{LEM}{A-1-04-11}{Eindeutige Darstellung in Polynomringen in n-Variablen}

\end{LEM}





\begin{DEF}{A-1-04-1}{Potenzreihenring mit Koeffizienten in Ring}

\end{DEF}

\begin{DEF}{A-1-04-10}{Polynomring in n-Variablen mit Koeffizienten aus Ring}

\end{DEF}







\section{Ideale und Quotienten}



\subsection{Definition}



\begin{DEF}{A-1-05-10}{Erzeugendensysteme von Ideale}

\end{DEF}

\begin{DEF}{A-1-05-17}{Quotienten für Ideale in Ringen mit Quotientenabbildung}

\end{DEF}

\begin{DEF}{A-1-05-2}{Ideal eines Ringes}

\end{DEF}

\begin{DEF}{A-1-05-7}{Von Teilmengen erzeugte Ideale}

\end{DEF}









\subsection{Example}



\begin{EXA}{A-1-05-11}{nZ}

\end{EXA}

\begin{EXA}{A-1-05-12}{$\{0\}$ und $\{1\}$ in jedem Ring sind Ideale}

\end{EXA}

\begin{EXA}{A-1-05-13}{$(2,X)$ im Polynomring $\mathbb{Z}(X)$}

\end{EXA}

\begin{EXA}{A-1-05-24}{Komplexen Zahlen isomorph zu Faktorring des Polynomringes in reellen Zahlen Mod $(X^2+1)$}

\end{EXA}

\begin{EXA}{A-1-05-4}{$Rx$ sind Ideale von $R$}

\end{EXA}













\subsection{Proposition}



\begin{PROP}{A-1-05-15}{Faktorring als Quotientenring bzgl. Ideale}

\end{PROP}

\begin{PROP}{A-1-05-18}{Faktorringe für Ideale mit kanonischer Projektion sind Quotienten}

\end{PROP}

\begin{PROP}{A-1-05-20}{Urbild von Idealen längs R-Homs ist Ideal}

\end{PROP}

\begin{PROP}{A-1-05-21}{Bilder von Idealen längs surjektiven R-Homs sind Ideale}

\end{PROP}

\begin{PROP}{A-1-05-22}{Homomorphisatz}

\end{PROP}

\begin{PROP}{A-1-05-25}{Erster Isomorphiesatz}

\end{PROP}

\begin{PROP}{A-1-05-26}{Zweiter Isomorphiesatz}

\end{PROP}

\begin{PROP}{A-1-05-5}{Kern eines R-Homs ist ein Ideal}

\end{PROP}

\begin{PROP}{A-1-05-6}{R-Hom ist injektiv gdw Kern $= 0$}

\end{PROP}



















\subsection{Corollar}



\begin{KORO}{A-1-05-16}{Jedes Ideal ist Kern eines geeigneten R-Homs}

\end{KORO}

\subsection{Lemma}



\begin{LEM}{A-1-05-14}{Vereinigung von aufsteigend inkludierten Idealen sind Ideale}

\end{LEM}

\begin{LEM}{A-1-05-3}{Charakterisierung von Idealen}

\end{LEM}

\begin{LEM}{A-1-05-9}{Schnitte von Idealen sind Ideale}

\end{LEM}





\subsection{Remark}



\begin{REM}{A-1-05-1}{Kern von Ringhomomorphismen nicht i.A. Unterring}

\end{REM}

\begin{REM}{A-1-05-19}{Quotientenabbildung ist surjektiv}

\end{REM}

\begin{REM}{A-1-05-23}{Struktur von Faktorring bestimmen durch raten eines Isomorphismus zw S und R/I und $I=ker(f)$}

\end{REM}

\begin{REM}{A-1-05-8}{Warum ist die Menge der erzeugten R-Linearkombinationen eine Ideal?}

\end{REM}








\section{Moduln}





\section{Moduln Grundlagen}

\subsection{Definition}

\subsection{Example}



\begin{EXA}{A-1-07-11}{K-Vektorräume sind K-Moduln}

\end{EXA}

\begin{EXA}{A-1-07-12}{Abelsche Gruppen sind Z-Moduln}

\end{EXA}

\begin{EXA}{A-1-07-13}{Moduln bzgl Polynomringe in Körpern sind ein K-Vektorraum mit einem K-linearen Endo}

\end{EXA}

\begin{EXA}{A-1-07-14}{Menge der Spaltentupel mit Elementen aus einem Ring ist mit komp. Add und diagonale R-Multip ein R-Moduln}

\end{EXA}

\begin{EXA}{A-1-07-15}{Für Körper sind K-Modulnhomomorphismen K-lineare Abbildungen}

\end{EXA}

\begin{EXA}{A-1-07-16}{Für Z sind Z-Modulnhomomorphismen Gruppenhomomorphismen}

\end{EXA}

\begin{EXA}{A-1-07-17}{Für Polynomringe in Körpern sind Modulnhomomorphismen K-lineare Abbildungen, die mit X* Polynom kommutieren}

\end{EXA}

\begin{EXA}{A-1-07-18}{Freie Moduln vom Rang n als isomorphe R-Moduln zu R^n}

\end{EXA}

\begin{EXA}{A-1-07-19}{Für Körper sind Untermoduln Untervektorräume}

\end{EXA}

\begin{EXA}{A-1-07-20}{Für Z sind Untermoduln Untergruppen}

\end{EXA}

\begin{EXA}{A-1-07-21}{Für Polynomringe in Körpern sind Untermoduln Endo-Stabile Untervektorräume}

\end{EXA}

\begin{EXA}{A-1-07-22}{Untermoduln eines Ringes sind Ideale}

\end{EXA}

\begin{EXA}{A-1-07-23}{R-Modulhomomorphismus von Koeffizienten aus R^n in M für fixiertes Elemente-Tupel in R-Moduln: Surjektiv gdw Endlich erzeugt}

\end{EXA}





























\begin{DEF}{A-1-07-1}{Modul zu einem Ring}

\end{DEF}

\begin{DEF}{A-1-07-10}{Zyklischer R-Modul}

\end{DEF}

\begin{DEF}{A-1-07-2}{R-Modulhomomorphismus}

\end{DEF}

\begin{DEF}{A-1-07-3}{Untermodul eines Moduls}

\end{DEF}

\begin{DEF}{A-1-07-4}{Durch Teilmengen eines R-Moduls erzeugte Untermoduln}

\end{DEF}

\begin{DEF}{A-1-07-5}{Kern und Bild eines R-Modulhomomorphismus}

\end{DEF}

\begin{DEF}{A-1-07-6}{Innere Direkte Summe von Untermoduln}

\end{DEF}

\begin{DEF}{A-1-07-7}{Direkte Summe von Moduln}

\end{DEF}

\begin{DEF}{A-1-07-8}{Direkte Produkt von Moduln}

\end{DEF}

\begin{DEF}{A-1-07-9}{Annulatorideal von R-Moduln}

\end{DEF}


























\pagebreak
\printbibliography
\end{document}