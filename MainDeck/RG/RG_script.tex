\documentclass[10pt, letterpaper]{article}

% Inhaltsverzeichnis für Pakettypen (nur für Übersicht im Header, wird nicht im Dokument angezeigt)
% 1. Seitenlayout und Ränder
% 2. Sprache und Zeichensatz
% 3. Mathematik und Theorem-Umgebungen
% 4. Eigene Makros
% 5. Diagramme und Grafiken
% 6. Tabellen und Aufzählungen
% 7. Inhaltsverzeichnis
% 8. Abschnittsüberschriften
% 9. Abstrakt-Umgebung
% 10. Todos/Notizen
% 11. Rahmen/Box-Umgebungen
% 12. Python-Integration
% 13. Literaturverwaltung
% 14. Hyperlinks
% 15. Absatzeinstellungen
% 16. Umgebungen
% 17. Titel und Autor

% --- 1. Seitenlayout und Ränder ---
\usepackage[margin=3cm]{geometry}

% --- 2. Sprache und Zeichensatz ---
\usepackage[english]{babel}
\usepackage[T1]{fontenc}
\usepackage[utf8]{inputenc}

% --- 3. Mathematik und Theorem-Umgebungen ---
\usepackage{amsmath, amssymb, amsthm}
\usepackage{mathrsfs}
\DeclareMathOperator{\WF}{WF}

% --- 4. Eigene Makros ---
\usepackage{xcolor}
\newcommand{\SKP}{\langle\cdot,\cdot\rangle}
\newcommand{\R}{\mathbb{R}}
\newcommand{\N}{\mathbb{N}}
\newcommand{\Q}{\mathbb{Q}}
\newcommand{\Z}{\mathbb{Z}}
\newcommand{\C}{\mathbb{C}}
\newcommand{\entwurf}[1]{\textcolor{red}{#1}}

% --- 5. Diagramme und Grafiken ---
\usepackage{graphicx}
\usepackage{tikz}
\usetikzlibrary{decorations.pathreplacing, arrows.meta, positioning}
\usepackage{tikz-cd}

% --- 6. Tabellen und Aufzählungen ---
\usepackage{enumitem}
\setlist[itemize]{left=0.5cm}

\newenvironment{romanenum}[1][]
  {%
    \ifx&#1&
    \else
      \textbf{#1}\quad
    \fi
    \begin{enumerate}[label=\roman*)]
  }
  {%
    \end{enumerate}%
  }

% --- 7. Inhaltsverzeichnis ---
\usepackage{tocloft}
\renewcommand{\cftsecfont}{\footnotesize}
\renewcommand{\cftsubsecfont}{\footnotesize}
\renewcommand{\cftsubsubsecfont}{\footnotesize}
\renewcommand{\cftsecpagefont}{\footnotesize}
\renewcommand{\cftsubsecpagefont}{\footnotesize}
\renewcommand{\cftsubsubsecpagefont}{\footnotesize}
\usepackage{etoc}

% --- 8. Abschnittsüberschriften ---
\usepackage{titlesec}
\titleformat{\section}{\normalfont\large\bfseries}{\thesection}{1em}{}
\titleformat{\subsection}{\normalfont\normalsize\bfseries}{\thesubsection}{0.5em}{}
\titleformat{\subsubsection}[runin]{\normalfont\normalsize\bfseries}{\thesubsubsection}{0.5em}{}
\setcounter{secnumdepth}{4}

% --- 9. Abstrakt-Umgebung ---
\usepackage{changepage}
\renewenvironment{abstract}
  {
    \begin{adjustwidth}{1.5cm}{1.5cm}
    \small
    \textsc{Abstract. –}%
  }
  {
    \end{adjustwidth}
  }

% --- 10. Todos/Notizen ---
\usepackage{todonotes}

% --- 11. Rahmen/Box-Umgebungen ---
\usepackage{mdframed}
\usepackage{tcolorbox}
\colorlet{shadecolor}{gray!25}

\newenvironment{customTheorem}
  {\vspace{10pt}%
   \begin{mdframed}[
     backgroundcolor=gray!20,
     linewidth=0pt,
     innertopmargin=10pt,
     innerbottommargin=10pt,
     skipabove=\dimexpr\topsep+\ht\strutbox\relax,
     skipbelow=\topsep,
   ]}
  {\end{mdframed}
   \vspace{10pt}%
  }

% --- 12. Python-Integration ---
% (Deaktiviert in dieser Version, aktiviere bei Bedarf)
% \usepackage{pythontex}
% \usepackage[makestderr]{pythontex}

% --- 13. Literaturverwaltung ---
\usepackage{csquotes}
\usepackage[backend=biber, style=alphabetic, citestyle=alphabetic]{biblatex}
\addbibresource{bibliography.bib}

% --- 14. Hyperlinks ---
\usepackage{hyperref}
\hypersetup{
  colorlinks   = true,
  urlcolor     = blue,
  linkcolor    = blue,
  citecolor    = blue,
  frenchlinks  = true
}

% --- 15. Absatzeinstellungen ---
\usepackage[parfill]{parskip}
\sloppy

% --- 16. Umgebungen ---
\usepackage{thmtools}

\newcommand{\CustomHeading}[3]{%
  \par\medskip\noindent%
  \textbf{#1 #2} \textnormal{(#3)}.\enskip%
}

\newenvironment{DEF}[2]{\CustomHeading{Definition}{#1}{#2}}{}
\newenvironment{PROP}[2]{\CustomHeading{Proposition}{#1}{#2}}{}
\newenvironment{THEO}[2]{\CustomHeading{Theorem}{#1}{#2}}{}
\newenvironment{LEM}[2]{\CustomHeading{Lemma}{#1}{#2}}{}
\newenvironment{KORO}[2]{\CustomHeading{Corollar}{#1}{#2}}{}
\newenvironment{REM}[2]{\CustomHeading{Remark}{#1}{#2}}{}
\newenvironment{EXA}[2]{\CustomHeading{Example}{#1}{#2}}{}
\newenvironment{STUD}[2]{\CustomHeading{Study}{#1}{#2}}{}
\newenvironment{CONC}[2]{\CustomHeading{Concept}{#1}{#2}}{}

\newenvironment{PROOF}
  {\begin{proof}}%
{\end{proof}}

% --- 17. Titel und Autor ---
\title{Riemannische Geometrie}
\author{Tim Jaschik}
\date{\today}

\begin{document}

\maketitle
\rule{\textwidth}{0.5pt}
\begin{abstract}
Kurze Beschreibung …
\end{abstract}
\rule{\textwidth}{0.5pt}
\vspace{0.5cm}

\tableofcontents

\pagebreak


\section{Untermannigfaltigkeiten im euklidischen Raum}

Test, ein weiterer Test, und noch einer oben drauf



\subsection{Definition}

\begin{DEF}{RG-1-02-1}{n-dimensionale Untermannigfaltigkeit des euklidischen Raumes}

\end{DEF}




\subsection{Example}

\begin{EXA}{RG-1-02-2}{n-Sphäre}

\end{EXA}

\begin{EXA}{RG-1-02-3}{Hyperboloid}

\end{EXA}

\begin{EXA}{RG-1-02-4}{n-Torus}

\end{EXA}

\begin{EXA}{RG-1-02-5}{SO(n)}

\end{EXA}













\subsection{Proposition}

\begin{PROP}{RG-1-02-6}{Charakterisierungen von Untermannigfaltigkeiten im euklidischen Raum}

\end{PROP}








\subsection{Corollar}







\subsection{Remark}

\begin{REM}{RG-1-02-7}{Anmerkungen}

\end{REM}




\section{Glatte Mannigfaltigkeiten}

\subsection{Definition}

\begin{DEF}{RG-1-03-2}{Äquivalente Atlanten}

\end{DEF}

\begin{DEF}{RG-1-03-4}{Glatte Mannigfaltigkeit}

\end{DEF}

\begin{DEF}{RG-1-03-5}{Orientierte Mannigfaltigkeiten}

\end{DEF}

\begin{DEF}{RG-1-03-6}{Untermannigfaltigkeit einer Mannigfaltigkeit}

\end{DEF}

\begin{DEF}{RG-1-03-14}{TEST}

\end{DEF}











\subsection{Example}

\begin{EXA}{RG-1-03-7}{n-Torus als Mfk}

\end{EXA}

\begin{EXA}{RG-1-03-8}{n-Sphäre als Mfk}

\end{EXA}

\begin{EXA}{RG-1-03-9}{Hyperboloid als Mfk}

\end{EXA}







\subsection{Remark}



\begin{REM}{RG-1-03-13}{Quotienten-Räume als Mfk als Motivation für verallg. Mfk-Begriff}

\end{REM}

\begin{REM}{RG-1-03-3}{Beispiel für nicht äquivalente Atlanten}

\end{REM}







\begin{EXA}{RG-1-03-10}{Reelle projektiver Raum als Mfk}

\end{EXA}

\begin{EXA}{RG-1-03-11}{Komplexe projektive Raum als Mfk}

\end{EXA}

\begin{EXA}{RG-1-03-12}{Möbiusband als Mfk}

\end{EXA}









\begin{DEF}{RG-1-03-1}{Atlas auf topologischen Hausdorff-Räumen}

\end{DEF}




\section{Differenzierbare Mannigfaltigkeiten}





\section{Untermannigfaltigkeiten im euklidischen Raum}





\section{Glatte Mannigfaltigkeiten}





\section{Glatte Abbildungen und Vektorfelder}

\subsection{Definition}

\begin{DEF}{RG-1-04-11}{Vektorfeld als glatter Schnitt in Tangentialbündel}

\end{DEF}

\begin{DEF}{RG-1-04-13}{Vektorfeld als Abbildung von glatten Funktionen auf Mfk}

\end{DEF}

\begin{DEF}{RG-1-04-14}{Lieklammer von Vektorfeldern (ergibt Vektorfelder)}

\end{DEF}

\begin{DEF}{RG-1-04-16}{Differential von glatten Abbildungen}

\end{DEF}

\begin{DEF}{RG-1-04-2}{Immersion / Submersion von Mfk}

\end{DEF}

\begin{DEF}{RG-1-04-3}{Einbettung von Mfk}

\end{DEF}

\begin{DEF}{RG-1-04-4}{Diffeomorphismus von Mfk}

\end{DEF}

\begin{DEF}{RG-1-04-5}{Tangentialvektor: Äquivalenzklassen von Kurven}

\end{DEF}

\begin{DEF}{RG-1-04-6}{Tangentenvektoren: Keime}

\end{DEF}

\begin{DEF}{RG-1-04-7}{Tangentenvektoren: Paare von Koordinatensysteme um p und Vektor}

\end{DEF}

\begin{DEF}{RG-1-04-9}{Tangentialbündel}

\end{DEF}























\subsection{Proposition}

\subsection{Remark}



\begin{REM}{RG-1-04-12}{Darstellung von Vektorfeldern durch partielle Abbleitungen (Tangentenvektoren)}

\end{REM}

\begin{REM}{RG-1-04-15}{Lieklammer:
Jacobi-Identität
Schiefsymmetrisch
Nicht linear über R}

\end{REM}

\begin{REM}{RG-1-04-8}{Konstruktion des Tangentialraums}

\end{REM}









\begin{PROP}{RG-1-04-10}{Tangentialbündel ist 2n-dimensional Mfk}

\end{PROP}





\begin{DEF}{RG-1-04-1}{Glatte Abbildung zwischen Mfk}

\end{DEF}







\section{Riemannische Mannigfaltigkeiten Definitionen und Beispiele}

\subsection{Definition}

\begin{DEF}{RG-1-05-18}{Riemannisches Produkt}

\end{DEF}

\begin{DEF}{RG-1-05-2}{Länge von Kurven auf Mfk}

\end{DEF}

\begin{DEF}{RG-1-05-3}{Riemannische Metrik auf Mfk}

\end{DEF}

\begin{DEF}{RG-1-05-6}{Abzählbare Mfk im Unendlichen}

\end{DEF}









\subsection{Example}

\begin{EXA}{RG-1-05-15}{Rotationsfläche}

\end{EXA}

\begin{EXA}{RG-1-05-16}{Hyperbolischer Raum}

\end{EXA}

\begin{EXA}{RG-1-05-17}{Poincaremodell des hyperbolischen Raumes}

\end{EXA}

\begin{EXA}{RG-1-05-19}{RxSn}

\end{EXA}

\begin{EXA}{RG-1-05-20}{Flacher Torus}

\end{EXA}

\begin{EXA}{RG-1-05-22}{Kleinsche Flasche}

\end{EXA}

\begin{EXA}{RG-1-05-9}{R2 in Polarkoordinaten}

\end{EXA}

















\subsection{Proposition}



\begin{PROP}{RG-1-05-21}{Charakterisierung der Isometrien von flachen Tori}

\end{PROP}

\begin{PROP}{RG-1-05-8}{Jede Mfk besitzt eine Riemannische Metrik}

\end{PROP}



\subsection{Remark}



\begin{REM}{RG-1-05-13}{Kompakte Mfk lassen sich in Euklidischen Raum einbetten}

\end{REM}

\begin{REM}{RG-1-05-14}{Unterscheidung zw. Innerer und äußerer Geometrie: Eigenschaften der Mfk vs der Einbettung}

\end{REM}

\begin{REM}{RG-1-05-4}{Pseudo-Riemannische Metrik auf Mfk}

\end{REM}

\begin{REM}{RG-1-05-5}{Lokale Beschreibung von Riemannischer Metrik}

\end{REM}

\begin{REM}{RG-1-05-7}{Relevanz der Abzählbarkeit im Unendlichen
1) Existenz von Verfeinerungen (lokal endlich) für offene Überdeckungen
2) Zerlegung der Eins}

\end{REM}













\begin{EXA}{RG-1-05-12}{Untermannigfaltigkeit mit induzierter Metrik}

\end{EXA}





\begin{DEF}{RG-1-05-1}{Finsler-Metrik auf Mfk}

\end{DEF}

\begin{DEF}{RG-1-05-10}{(Lokale) Isometrie von RMfk}

\end{DEF}

\begin{DEF}{RG-1-05-11}{Isometrische Einbettung von RMfk}

\end{DEF}












\section{Levi-Civita Zusammenhang}



\subsection{Definition}



\begin{DEF}{RG-1-07-2}{Zusammenhang auf Mfk}

\end{DEF}

\begin{DEF}{RG-1-07-4}{Koszulgleichung für Zusammenhänge}

\end{DEF}

\begin{DEF}{RG-1-07-5}{Chirstoffelsymbole als Korrekturterme in lokalen Koordinaten}

\end{DEF}







\subsection{Example}



\begin{EXA}{RG-1-07-8}{Christoffelsymbole für $R^n$ für euklidische Metrik}

\end{EXA}

\begin{EXA}{RG-1-07-9}{Christoffelsymbole für $R^2\backslash 0$ und lokale Darstellung der Metrik zur Polarkoordianten }

\end{EXA}



\subsection{Proposition}



\subsection{Lemma}



\begin{LEM}{RG-1-07-7}{Formel für Christoffelsymbole aus Koszulgleichung}

\end{LEM}

\subsection{Theorem}

\subsection{Remark}



\begin{REM}{RG-1-07-6}{Levi-Civita Zusammenhang $D_XY_p$ hängt nur von $X_p$ ab}

\end{REM}





\begin{THEO}{RG-1-07-3}{Fundamentaltheorem der Riemannischen Geometrie}

\end{THEO}





\begin{PROP}{RG-1-07-10}{Induzierter LC-Zusammenhang auf Untermannigfaltigkeiten von RMfk}

\end{PROP}








\section{Kovariante Ableitung längs einer Kurve}

\subsection{Definition}

\subsection{Proposition}



\begin{PROP}{RG-1-08-2}{Kovariante Ableitungs-Operator längs Kurven induziert durch LC-Zusammenhang der RMfk (EE)}

\end{PROP}

\begin{PROP}{RG-1-08-3}{Kovariante Ableitung der Riemannischen Metrik längs Kurven}

\end{PROP}







\begin{DEF}{RG-1-08-1}{Vektorfelder längs Kurven}

\end{DEF}








\section{Paralleltransport}

\subsection{Definition}

\begin{DEF}{RG-1-09-3}{Parallelverschiebung von Tangentialvektoren bzgl parallelen Vektorfeldern längs Kurven}

\end{DEF}





\subsection{Example}



\begin{EXA}{RG-1-09-6}{Parallelverschiebung im euklidischen Raum}

\end{EXA}

\begin{EXA}{RG-1-09-7}{Parallelverschiebung auf Sn}

\end{EXA}



\subsection{Proposition}

\begin{PROP}{RG-1-09-5}{Parallelverschiebung ist Isometrie zwischen Tangentialräumen}

\end{PROP}



\subsection{Remark}



\begin{REM}{RG-1-09-4}{Abhängigkeit der Parallelverschiebung von Kurve}

\end{REM}





\begin{PROP}{RG-1-09-2}{Eind. Existenz von parallelen Vektorfelder für Anfangswert (Punkt,Tangentialvektor)}

\end{PROP}





\begin{DEF}{RG-1-09-1}{Parallele Vektorfelder längs Kurven}

\end{DEF}








\section{Geodätische und Exponentialabbildung}






\section{Geodätische}






\section{Exponentialabbildung}






\section{Satz von Hopf-Rinow}






\section{Riemannische Geometrie}




\pagebreak
\printbibliography
\end{document}