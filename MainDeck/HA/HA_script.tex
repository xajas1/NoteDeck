\documentclass[10pt, letterpaper]{article}

% Inhaltsverzeichnis für Pakettypen (nur für Übersicht im Header, wird nicht im Dokument angezeigt)
% 1. Seitenlayout und Ränder
% 2. Sprache und Zeichensatz
% 3. Mathematik und Theorem-Umgebungen
% 4. Eigene Makros
% 5. Diagramme und Grafiken
% 6. Tabellen und Aufzählungen
% 7. Inhaltsverzeichnis
% 8. Abschnittsüberschriften
% 9. Abstrakt-Umgebung
% 10. Todos/Notizen
% 11. Rahmen/Box-Umgebungen
% 12. Python-Integration
% 13. Literaturverwaltung
% 14. Hyperlinks
% 15. Absatzeinstellungen
% 16. Umgebungen
% 17. Titel und Autor

% --- 1. Seitenlayout und Ränder ---
\usepackage[margin=3cm]{geometry}

% --- 2. Sprache und Zeichensatz ---
\usepackage[english]{babel}
\usepackage[T1]{fontenc}
\usepackage[utf8]{inputenc}

% --- 3. Mathematik und Theorem-Umgebungen ---
\usepackage{amsmath, amssymb, amsthm}
\usepackage{mathrsfs}
\DeclareMathOperator{\WF}{WF}

% --- 4. Eigene Makros ---
\usepackage{xcolor}
\newcommand{\SKP}{\langle\cdot,\cdot\rangle}
\newcommand{\R}{\mathbb{R}}
\newcommand{\N}{\mathbb{N}}
\newcommand{\Q}{\mathbb{Q}}
\newcommand{\Z}{\mathbb{Z}}
\newcommand{\C}{\mathbb{C}}
\newcommand{\entwurf}[1]{\textcolor{red}{#1}}

% --- 5. Diagramme und Grafiken ---
\usepackage{graphicx}
\usepackage{tikz}
\usetikzlibrary{decorations.pathreplacing, arrows.meta, positioning}
\usepackage{tikz-cd}

% --- 6. Tabellen und Aufzählungen ---
\usepackage{enumitem}
\setlist[itemize]{left=0.5cm}

\newenvironment{romanenum}[1][]
  {%
    \ifx&#1&
    \else
      \textbf{#1}\quad
    \fi
    \begin{enumerate}[label=\roman*)]
  }
  {%
    \end{enumerate}%
  }

% --- 7. Inhaltsverzeichnis ---
\usepackage{tocloft}
\renewcommand{\cftsecfont}{\footnotesize}
\renewcommand{\cftsubsecfont}{\footnotesize}
\renewcommand{\cftsubsubsecfont}{\footnotesize}
\renewcommand{\cftsecpagefont}{\footnotesize}
\renewcommand{\cftsubsecpagefont}{\footnotesize}
\renewcommand{\cftsubsubsecpagefont}{\footnotesize}
\usepackage{etoc}

% --- 8. Abschnittsüberschriften ---
\usepackage{titlesec}
\titleformat{\section}{\normalfont\large\bfseries}{\thesection}{1em}{}
\titleformat{\subsection}{\normalfont\normalsize\bfseries}{\thesubsection}{0.5em}{}
\titleformat{\subsubsection}{\normalfont\normalsize\bfseries}{\thesubsubsection}{0.5em}{}
\setcounter{secnumdepth}{4}

% --- 9. Abstrakt-Umgebung ---
\usepackage{changepage}
\renewenvironment{abstract}
  {
    \begin{adjustwidth}{1.5cm}{1.5cm}
    \small
    \textsc{Abstract. –}%
  }
  {
    \end{adjustwidth}
  }

% --- 10. Todos/Notizen ---
\usepackage{todonotes}

% --- 11. Rahmen/Box-Umgebungen ---
\usepackage{mdframed}
\usepackage{tcolorbox}
\colorlet{shadecolor}{gray!25}

\newenvironment{customTheorem}
  {\vspace{10pt}%
   \begin{mdframed}[
     backgroundcolor=gray!20,
     linewidth=0pt,
     innertopmargin=10pt,
     innerbottommargin=10pt,
     skipabove=\dimexpr\topsep+\ht\strutbox\relax,
     skipbelow=\topsep,
   ]}
  {\end{mdframed}
   \vspace{10pt}%
  }

% --- 12. Python-Integration ---
% (Deaktiviert in dieser Version, aktiviere bei Bedarf)
% \usepackage{pythontex}
% \usepackage[makestderr]{pythontex}

% --- 13. Literaturverwaltung ---
\usepackage{csquotes}
\usepackage[backend=biber, style=alphabetic, citestyle=alphabetic]{biblatex}
\addbibresource{bibliography.bib}

% --- 14. Hyperlinks ---
\usepackage{hyperref}
\hypersetup{
  colorlinks   = true,
  urlcolor     = blue,
  linkcolor    = blue,
  citecolor    = blue,
  frenchlinks  = true
}

% --- 15. Absatzeinstellungen ---
\usepackage[parfill]{parskip}
\sloppy

% --- 16. Umgebungen ---
\usepackage{thmtools}

\newcommand{\CustomHeading}[3]{%
  \par\medskip\noindent%
  \textbf{#1 #2} \textnormal{(#3)}.\enskip%
}

\newenvironment{DEF}[2]{\CustomHeading{Definition}{#1}{#2}}{}
\newenvironment{PROP}[2]{\CustomHeading{Proposition}{#1}{#2}}{}
\newenvironment{THEO}[2]{\CustomHeading{Theorem}{#1}{#2}}{}
\newenvironment{LEM}[2]{\CustomHeading{Lemma}{#1}{#2}}{}
\newenvironment{KORO}[2]{\CustomHeading{Corollar}{#1}{#2}}{}
\newenvironment{REM}[2]{\CustomHeading{Remark}{#1}{#2}}{}
\newenvironment{EXA}[2]{\CustomHeading{Example}{#1}{#2}}{}
\newenvironment{STUD}[2]{\CustomHeading{Study}{#1}{#2}}{}
\newenvironment{CONC}[2]{\CustomHeading{Concept}{#1}{#2}}{}

\newenvironment{PROOF}
  {\begin{proof}}%
{\end{proof}}

% --- 17. Titel und Autor ---
\title{Homologische Algebra}
\author{Tim Jaschik}
\date{\today}

\begin{document}

\maketitle
\rule{\textwidth}{0.5pt}
\begin{abstract}
…
\end{abstract}
\rule{\textwidth}{0.5pt}
\vspace{0.5cm}

\tableofcontents

\pagebreak


\section{Kategorien und Funktoren in HA}

\subsection{Definition}

\subsection{Example}

\subsection{Study}



\begin{STUD}{HA-1-01-22}{3 Grundbegriffe: Kat, Funk, naat Trafo}

\end{STUD}

\begin{STUD}{HA-1-01-23}{Von (kleinen) Funktorkategorien zur Kategorie der Komplexe in der Kategroei der Abelschen Gruppen}

\end{STUD}







\begin{EXA}{HA-1-01-2}{Exa Kategorien}

\end{EXA}

\begin{EXA}{HA-1-01-20}{Exa Kleine Funktorenkategorie: Diag / Seq }

\end{EXA}

\begin{EXA}{HA-1-01-5}{Exa Volle Unterkategorie}

\end{EXA}

\begin{EXA}{HA-1-01-7}{Hom-Funktor}

\end{EXA}











\begin{DEF}{HA-1-01-1}{Kategorie}

\end{DEF}

\begin{DEF}{HA-1-01-10}{Weg in Kategorie}

\end{DEF}

\begin{DEF}{HA-1-01-11}{Gelabelter Weg}

\end{DEF}

\begin{DEF}{HA-1-01-12}{Einfacher Weg}

\end{DEF}

\begin{DEF}{HA-1-01-13}{Kommutatives Diagramm}

\end{DEF}

\begin{DEF}{HA-1-01-14}{Kontravariante Funktoren}

\end{DEF}

\begin{DEF}{HA-1-01-15}{Isopmorphie in Kategorie}

\end{DEF}

\begin{DEF}{HA-1-01-16}{Natürliche Transformation}

\end{DEF}

\begin{DEF}{HA-1-01-17}{Natürlicher Isomorphismus}

\end{DEF}

\begin{DEF}{HA-1-01-18}{Komposition von Natürlichen Transformationen}

\end{DEF}

\begin{DEF}{HA-1-01-19}{(Kleine) Funktorenkategorie}

\end{DEF}

\begin{DEF}{HA-1-01-21}{Komplex und Ketten-Abbildung}

\end{DEF}

\begin{DEF}{HA-1-01-3}{Unterkategorie}

\end{DEF}

\begin{DEF}{HA-1-01-4}{Volle Unterkategorie}

\end{DEF}

\begin{DEF}{HA-1-01-6}{Funktor}

\end{DEF}

\begin{DEF}{HA-1-01-8}{Sequenz in Kategorie}

\end{DEF}

\begin{DEF}{HA-1-01-9}{Diagramm in Katagorie}

\end{DEF}






































\section{Moduln in HA}

\subsection{Definition}

\subsection{Theorem}



\begin{THEO}{HA-1-02-5}{ISO-Sätze}

\end{THEO}





\begin{DEF}{HA-1-02-1}{Links/Rechts Moduln}

\end{DEF}

\begin{DEF}{HA-1-02-2}{Abelsche Gruppe}

\end{DEF}

\begin{DEF}{HA-1-02-4}{Kern / Im / CoKern für R-Hom}

\end{DEF}

\begin{DEF}{HA-1-02-6}{Freie R-Moduln}

\end{DEF}

\begin{DEF}{HA-1-02-7}{Freie Abelsche Gruppe}

\end{DEF}














\section{Tensorprodukt in HA}




\section{Abelsche Kategorien}

\subsection{Definition}

\subsection{Example}

\subsection{Proposition}

\subsection{Remark}

\subsection{Study}



\begin{STUD}{HA-1-04-20}{Herleitung abelscher Kategorien und additiver Funktoren als allg. Rahmen für Komplexe in abelschen Kategorien}

\end{STUD}





\begin{REM}{HA-1-04-15}{Exaktheit in abelschen Kategorien durch Subobjekt in Gadgete}

\end{REM}





\begin{PROP}{HA-1-04-17}{Funktorkategorie zu abelschen Kategorie ist abelsch}

\end{PROP}

\begin{PROP}{HA-1-04-8}{Beziehung Monic / Epic und ker / cokern in additiven Kategorien}

\end{PROP}







\begin{EXA}{HA-1-04-12}{Exa abelsche Kategorie: (Volle Unterkategorien von) Abelsche Gruppen }

\end{EXA}

\begin{EXA}{HA-1-04-14}{Exa Exakte Kategorie}

\end{EXA}







\begin{DEF}{HA-1-04-1}{Additive Kategorie}

\end{DEF}

\begin{DEF}{HA-1-04-10}{Quotienten-Objekt in additiven Kategorien}

\end{DEF}

\begin{DEF}{HA-1-04-11}{Abelsche Kategorie}

\end{DEF}

\begin{DEF}{HA-1-04-13}{Exakte Kategorie}

\end{DEF}

\begin{DEF}{HA-1-04-16}{Abelsche Unterkategorie}

\end{DEF}

\begin{DEF}{HA-1-04-18}{Projektive Objekte in abelschen Kategorien}

\end{DEF}

\begin{DEF}{HA-1-04-19}{Injektive Objekte in abelschen Kategorien}

\end{DEF}

\begin{DEF}{HA-1-04-2}{Additiver Funktor von additiven Kategorien}

\end{DEF}

\begin{DEF}{HA-1-04-3}{Direkte Summe in additiven Kategorien}

\end{DEF}

\begin{DEF}{HA-1-04-4}{Monomorphismen in Kategorien}

\end{DEF}

\begin{DEF}{HA-1-04-5}{Epimorphismen in Kategorien}

\end{DEF}

\begin{DEF}{HA-1-04-6}{Monics / Epics in additiven Kategorien}

\end{DEF}

\begin{DEF}{HA-1-04-7}{Ker / Coker in additiven Kategorien}

\end{DEF}

\begin{DEF}{HA-1-04-9}{Subgadget von Objekten in additiven Katgeorien}

\end{DEF}
































\section{Homologie Funktor in HA}

\subsection{Definition}

\subsection{Example}

\subsection{Proposition}

\subsection{Theorem}

\subsection{Remark}

\subsection{Study}



\begin{STUD}{HA-1-05-26}{Einführung der Homologie-Funktoren}

\end{STUD}

\begin{STUD}{HA-1-05-27}{(Natürlicher) Zusammenhangs-Homomorphismus}

\end{STUD}

\begin{STUD}{HA-1-05-28}{Interpretation des Zusammenhangs-Isomorphismus via Arrow Kategorie}

\end{STUD}

\begin{STUD}{HA-1-05-29}{Was ist die Singuläre Homologie Theorie}

\end{STUD}











\begin{REM}{HA-1-05-1}{Exakte Sequenzen sind Komplexe}

\end{REM}

\begin{REM}{HA-1-05-18}{Interpretation des Zusammenhangs-Homomorphismus}

\end{REM}

\begin{REM}{HA-1-05-2}{Kurze Exakte Sequenzen zu Komplexen erweitern}

\end{REM}

\begin{REM}{HA-1-05-7}{Homologie als Abweichung von Exaktheit eines Komplexes}

\end{REM}











\begin{THEO}{HA-1-05-11}{Zu kurzen exakte Sequenz (K,KAbb) in abel. Kategorie der Komplexe existiert ein Zusammenhangs-Homomorphismus}

\end{THEO}

\begin{THEO}{HA-1-05-13}{Kurze Exakte Sequenz in Kategorie der Komplexe induziert lange exakte Homologie-Sequenz}

\end{THEO}

\begin{THEO}{HA-1-05-15}{Zusammenhangs-Homomorphismus zu kurzen exakten Sequenzen in Kategorie der Komplexe ist natürlich}

\end{THEO}









\begin{PROP}{HA-1-05-22}{Homotope Ketten Abbildungen induzieren gleiche Homologie Abbildungen}

\end{PROP}

\begin{PROP}{HA-1-05-25}{Kontrahierbare Komplexe sind azyklisch}

\end{PROP}

\begin{PROP}{HA-1-05-9}{n-te Homologie ist additiver Funktor}

\end{PROP}









\begin{EXA}{HA-1-05-20}{Exa Grad einer Abbildung zwischen Komplexen}

\end{EXA}

\begin{EXA}{HA-1-05-8}{Fundamentale Exakte Sequenzen für Komplexe: Zyklen Ränder und Homologie}

\end{EXA}







\begin{DEF}{HA-1-05-17}{Arrow Kategorie}

\end{DEF}

\begin{DEF}{HA-1-05-19}{Grad einer Abbildung zwischen Komplexen}

\end{DEF}

\begin{DEF}{HA-1-05-21}{Homotope Ketten Abbildungen (Null-Homotopie)}

\end{DEF}

\begin{DEF}{HA-1-05-24}{Kontrahierbare Komplexe}

\end{DEF}

\begin{DEF}{HA-1-05-3}{Sequenzen von Objekte}

\end{DEF}

\begin{DEF}{HA-1-05-4}{Positive / Negative Komplexe}

\end{DEF}

\begin{DEF}{HA-1-05-5}{Ketten / Zyklen / Ränder in Komplexen}

\end{DEF}

\begin{DEF}{HA-1-05-6}{n-te Homologie in Komplexen}

\end{DEF}




















\section{Komplexe in HA}

\subsection{Definition}

\subsection{Proposition}

\subsection{Study}



\begin{STUD}{HA-1-06-13}{Basics der allg. Komplexe in abelschen Kategorien und die Kategorie der Komplexe}

\end{STUD}





\begin{PROP}{HA-1-06-12}{Kettenabbildung in Quotienten Komplex durch natürliche Abbildung}

\end{PROP}

\begin{PROP}{HA-1-06-5}{Kategorie der Komplexe abelsch, falls Kategorie abelsch}

\end{PROP}







\begin{DEF}{HA-1-06-1}{Komplex in abelschen Kategorien}

\end{DEF}

\begin{DEF}{HA-1-06-10}{Kurze Exakte Sequenzen von Komplexen und Ketten Abbildungen}

\end{DEF}

\begin{DEF}{HA-1-06-11}{Quotienten Komplex}

\end{DEF}

\begin{DEF}{HA-1-06-2}{Ketten Abbildung zwischen Komplexen in abelschen Kategorien}

\end{DEF}

\begin{DEF}{HA-1-06-3}{Kategorie der Komplexe in abelschen Kategorien}

\end{DEF}

\begin{DEF}{HA-1-06-4}{Unterkomplex in abelschen Kategorien}

\end{DEF}

\begin{DEF}{HA-1-06-7}{Isomorphie in Kategorie der Komplexe}

\end{DEF}

\begin{DEF}{HA-1-06-8}{Direkte Summe von Komplexen}

\end{DEF}

\begin{DEF}{HA-1-06-9}{Exaktheit von Sequenze von Komplexen und Ketten Abbildungen}

\end{DEF}






















\section{Hilftslemma für Diagramme der HA}

\subsection{Lemma}



\begin{LEM}{HA-1-07-1}{Basics für Exaktheit von Sequenzen}

\end{LEM}

\begin{LEM}{HA-1-07-2}{Kurze Exakte Sequenzen Basics}

\end{LEM}

\begin{LEM}{HA-1-07-3}{Links/Rechts Vervollständigung von 03-03 Komm Exa}

\end{LEM}

\begin{LEM}{HA-1-07-4}{5-Lemma}

\end{LEM}

\begin{LEM}{HA-1-07-5}{030-030 vert ISO: oben exa gdw unten exa}

\end{LEM}

\begin{LEM}{HA-1-07-6}{3x3 Lemma}

\end{LEM}

\begin{LEM}{HA-1-07-7}{Schlagstock Lemma}

\end{LEM}

\begin{LEM}{HA-1-07-8}{Schlangen Lemma}

\end{LEM}




















\section{Exaktheit von Sequenzen}




\section{Splitting}

\subsection{Definition}

\subsection{Lemma}



\begin{LEM}{HA-1-10-2}{Splitting Cases}

\end{LEM}





\begin{DEF}{HA-1-10-1}{Splitting Basics}

\end{DEF}






\section{Kategorische Formulierung der Homologietheorie}

\subsection{Definition}



\begin{DEF}{HA-1-12-1}{Eilenberg-Stennrod Axiom}

\end{DEF}






\section{Homologie mit Koeffizienten}




\pagebreak
\printbibliography
\end{document}