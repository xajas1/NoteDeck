\documentclass[10pt, letterpaper]{article}

% Inhaltsverzeichnis für Pakettypen (nur für Übersicht im Header, wird nicht im Dokument angezeigt)
% 1. Seitenlayout und Ränder
% 2. Sprache und Zeichensatz
% 3. Mathematik und Theorem-Umgebungen
% 4. Eigene Makros
% 5. Diagramme und Grafiken
% 6. Tabellen und Aufzählungen
% 7. Inhaltsverzeichnis
% 8. Abschnittsüberschriften
% 9. Abstrakt-Umgebung
% 10. Todos/Notizen
% 11. Rahmen/Box-Umgebungen
% 12. Python-Integration
% 13. Literaturverwaltung
% 14. Hyperlinks
% 15. Absatzeinstellungen
% 16. Umgebungen
% 17. Titel und Autor

% --- 1. Seitenlayout und Ränder ---
\usepackage[margin=3cm]{geometry}

% --- 2. Sprache und Zeichensatz ---
\usepackage[english]{babel}
\usepackage[T1]{fontenc}
\usepackage[utf8]{inputenc}

% --- 3. Mathematik und Theorem-Umgebungen ---
\usepackage{amsmath, amssymb, amsthm}
\usepackage{mathrsfs}
\DeclareMathOperator{\WF}{WF}

% --- 4. Eigene Makros ---
\usepackage{xcolor}
\newcommand{\SKP}{\langle\cdot,\cdot\rangle}
\newcommand{\R}{\mathbb{R}}
\newcommand{\N}{\mathbb{N}}
\newcommand{\Q}{\mathbb{Q}}
\newcommand{\Z}{\mathbb{Z}}
\newcommand{\C}{\mathbb{C}}
\newcommand{\entwurf}[1]{\textcolor{red}{#1}}

% --- 5. Diagramme und Grafiken ---
\usepackage{graphicx}
\usepackage{tikz}
\usetikzlibrary{decorations.pathreplacing, arrows.meta, positioning}
\usepackage{tikz-cd}

% --- 6. Tabellen und Aufzählungen ---
\usepackage{enumitem}
\setlist[itemize]{left=0.5cm}

\newenvironment{romanenum}[1][]
  {%
    \ifx&#1&
    \else
      \textbf{#1}\quad
    \fi
    \begin{enumerate}[label=\roman*)]
  }
  {%
    \end{enumerate}%
  }

% --- 7. Inhaltsverzeichnis ---
\usepackage{tocloft}
\renewcommand{\cftsecfont}{\footnotesize}
\renewcommand{\cftsubsecfont}{\footnotesize}
\renewcommand{\cftsubsubsecfont}{\footnotesize}
\renewcommand{\cftsecpagefont}{\footnotesize}
\renewcommand{\cftsubsecpagefont}{\footnotesize}
\renewcommand{\cftsubsubsecpagefont}{\footnotesize}
\usepackage{etoc}

% --- 8. Abschnittsüberschriften ---
\usepackage{titlesec}
\titleformat{\section}{\normalfont\large\bfseries}{\thesection}{1em}{}
\titleformat{\subsection}{\normalfont\normalsize\bfseries}{\thesubsection}{0.5em}{}
\titleformat{\subsubsection}{\normalfont\normalsize\bfseries}{\thesubsubsection}{0.5em}{}
\setcounter{secnumdepth}{4}

% --- 9. Abstrakt-Umgebung ---
\usepackage{changepage}
\renewenvironment{abstract}
  {
    \begin{adjustwidth}{1.5cm}{1.5cm}
    \small
    \textsc{Abstract. –}%
  }
  {
    \end{adjustwidth}
  }

% --- 10. Todos/Notizen ---
\usepackage{todonotes}

% --- 11. Rahmen/Box-Umgebungen ---
\usepackage{mdframed}
\usepackage{tcolorbox}
\colorlet{shadecolor}{gray!25}

\newenvironment{customTheorem}
  {\vspace{10pt}%
   \begin{mdframed}[
     backgroundcolor=gray!20,
     linewidth=0pt,
     innertopmargin=10pt,
     innerbottommargin=10pt,
     skipabove=\dimexpr\topsep+\ht\strutbox\relax,
     skipbelow=\topsep,
   ]}
  {\end{mdframed}
   \vspace{10pt}%
  }

% --- 12. Python-Integration ---
% (Deaktiviert in dieser Version, aktiviere bei Bedarf)
% \usepackage{pythontex}
% \usepackage[makestderr]{pythontex}

% --- 13. Literaturverwaltung ---
\usepackage{csquotes}
\usepackage[backend=biber, style=alphabetic, citestyle=alphabetic]{biblatex}
\addbibresource{bibliography.bib}

% --- 14. Hyperlinks ---
\usepackage{hyperref}
\hypersetup{
  colorlinks   = true,
  urlcolor     = blue,
  linkcolor    = blue,
  citecolor    = blue,
  frenchlinks  = true
}

% --- 15. Absatzeinstellungen ---
\usepackage[parfill]{parskip}
\sloppy

% --- 16. Umgebungen ---
\usepackage{thmtools}

\newcommand{\CustomHeading}[3]{%
  \par\medskip\noindent%
  \textbf{#1 #2} \textnormal{(#3)}.\enskip%
}

\newenvironment{DEF}[2]{\CustomHeading{Definition}{#1}{#2}}{}
\newenvironment{PROP}[2]{\CustomHeading{Proposition}{#1}{#2}}{}
\newenvironment{THEO}[2]{\CustomHeading{Theorem}{#1}{#2}}{}
\newenvironment{LEM}[2]{\CustomHeading{Lemma}{#1}{#2}}{}
\newenvironment{KORO}[2]{\CustomHeading{Corollar}{#1}{#2}}{}
\newenvironment{REM}[2]{\CustomHeading{Remark}{#1}{#2}}{}
\newenvironment{EXA}[2]{\CustomHeading{Example}{#1}{#2}}{}
\newenvironment{STUD}[2]{\CustomHeading{Study}{#1}{#2}}{}
\newenvironment{CONC}[2]{\CustomHeading{Concept}{#1}{#2}}{}

\newenvironment{PROOF}
  {\begin{proof}}%
{\end{proof}}

% --- 17. Titel und Autor ---
\title{Eichfeldtheorie 1}
\author{Tim Jaschik}
\date{\today}

\begin{document}

\maketitle
\rule{\textwidth}{0.5pt}
\begin{abstract}
…
\end{abstract}
\rule{\textwidth}{0.5pt}
\vspace{0.5cm}

\tableofcontents


\pagebreak


\section{Faserbündel}




\subsection{Definition}

\begin{DEF}{EFT1-1-02-1}{Lokale triviale Faserung mit typischen Fasern auf Mfk}
    Seien $E, M$ und $F$ differenzierbare Mannigfaltikeiten und $\pi: E \rightarrow M$ eine differenzierbare Abbildung. Dann heißt $(E, \pi, M)$ eine lokal triviale Faserung mit typischer Faser $F$, wenn es zu jedem $x \in M$ eine offene Umgebung $U$ gibt und einen Diffeomorphismus $\varphi: \pi^{-1}(U):=E \mid U \rightarrow$ $U \times F$, sodass

    $$
    \begin{array}{lll}
    E \mid U & \xrightarrow{\varphi} & U \times F \\
    \pi \searrow & & \swarrow p r_1 \\
    & U &
    \end{array}
    $$
    
    kommutiert. Man spricht auch von der lokal trivialen Faserung $E \rightarrow M$ oder $E$.
\end{DEF}

\begin{DEF}{EFT1-1-02-10}{Bündelatlas für lokale triviale Faserungen}

\end{DEF}

\begin{DEF}{EFT1-1-02-11}{Faserkarte am Punkt x im Basisraum}

\end{DEF}

\begin{DEF}{EFT1-1-02-12}{Bündelkartenwechsel zwischen Bündelkarten}

\end{DEF}

\begin{DEF}{EFT1-1-02-13}{G-Faserbündel mit Liegruppen als Strukturgruppen}

\end{DEF}

\begin{DEF}{EFT1-1-02-14}{Prinzipalbüdel / Hauptfaserbündel}

\end{DEF}

\begin{DEF}{EFT1-1-02-16}{(Differenzierbare) (Lokale) Schnitte in lokal trivialen Faserungen}

\end{DEF}

\begin{DEF}{EFT1-1-02-17}{Raum der differenzierbaren lokalen Schnitte}

\end{DEF}

\begin{DEF}{EFT1-1-02-2}{Vektorraumbündel}

\end{DEF}

\begin{DEF}{EFT1-1-02-37}{Bündelmetrik auf Totalraum ist ein Schnitt in Sym^2(E), sodass g pw. positiv definit }

\end{DEF}

\begin{DEF}{EFT1-1-02-40}{Vektorraumbündel vom endlichen Typ}

\end{DEF}

\begin{DEF}{EFT1-1-02-43}{Bündelisomorphismus}

\end{DEF}

\begin{DEF}{EFT1-1-02-44}{Trivialisierung von Totalraum}

\end{DEF}

\begin{DEF}{EFT1-1-02-45}{Vektorraumbündelabbildung über diff. Abbildungen zwischen Vektorraumbündeln}

\end{DEF}

\begin{DEF}{EFT1-1-02-46}{Vektorraumbündelisomorphismus}

\end{DEF}

\begin{DEF}{EFT1-1-02-48}{Induzierte Bündel durch Abbildungen}

\end{DEF}

\begin{DEF}{EFT1-1-02-54}{Induzierte Bündel bei Einbettungen von UnterMfk}

\end{DEF}

\begin{DEF}{EFT1-1-02-55}{Untervektorraumbündel}

\end{DEF}

\begin{DEF}{EFT1-1-02-63}{Reduktionen von Faserbündeln mit Strukturgruppe bzgl abgeschlossener Untergruppe}

\end{DEF}

\begin{DEF}{EFT1-1-02-7}{Lokale triviale Faserung als Tripel von Totalraum, Basisraum, Bündelprojektion mit typischen Fasern}

\end{DEF}

\begin{DEF}{EFT1-1-02-8}{Reale Fasern in lokal trivialen Faserungen}

\end{DEF}

\begin{DEF}{EFT1-1-02-9}{Bündelkarten für offene Teilmengen der Basis}

\end{DEF}
















































\subsection{Example}

\begin{EXA}{EFT1-1-02-18}{Raum der diff. lokalen Schnitte in Kreuzprodukten}

\end{EXA}

\begin{EXA}{EFT1-1-02-19}{Raum der diff. Lokalen Schnitte im Tangentialbündel}

\end{EXA}

\begin{EXA}{EFT1-1-02-20}{Jedes Vektorraumbündel hat einen lokalen Schnitt x auf O_x in E_x}

\end{EXA}

\begin{EXA}{EFT1-1-02-21}{Im Tangentialbündel existiert kein diff. Schnitt, der nirgends verschwindet}

\end{EXA}

\begin{EXA}{EFT1-1-02-22}{S1 auf S1, z auf z^2 gibt es keinen Schnitt}

\end{EXA}

\begin{EXA}{EFT1-1-02-29}{Bündelstruktur von Tangentialbündel als Ergebnis der Konstruktion von Präbündeln}

\end{EXA}

\begin{EXA}{EFT1-1-02-3}{Projektion von Kreuzprodukt ist eine lokal triviale Faserung}

\end{EXA}

\begin{EXA}{EFT1-1-02-30}{Präbündel zum GL-Prinzipalbündel}

\end{EXA}

\begin{EXA}{EFT1-1-02-31}{Präbündel zum O(n)-Prinzipalbündel für Riemannische Mfk}

\end{EXA}

\begin{EXA}{EFT1-1-02-33}{Hom-Raum für Homomorphismen zwischen Vektorraumbündeln sind Vektorraumbündel}

\end{EXA}

\begin{EXA}{EFT1-1-02-34}{Mult}

\end{EXA}

\begin{EXA}{EFT1-1-02-35}{Sym}

\end{EXA}

\begin{EXA}{EFT1-1-02-36}{Alt}

\end{EXA}

\begin{EXA}{EFT1-1-02-38}{Riemannische Metrik als Bündelmetrik im Tangentialbündel}

\end{EXA}

\begin{EXA}{EFT1-1-02-39}{Gamma (Alt^k(TM))}

\end{EXA}

\begin{EXA}{EFT1-1-02-4}{Tangentialbündel mit differenzierbarer Struktur ist Vektorraumbündel}

\end{EXA}

\begin{EXA}{EFT1-1-02-41}{Tangentialbündel von S^n ist von endlichem Typ}

\end{EXA}

\begin{EXA}{EFT1-1-02-47}{Differential von glatten Abbildungen zw. Tangentialbündel von Mfk ist eine Vektorraumbündelabbildung über glatte Abbildung f}

\end{EXA}

\begin{EXA}{EFT1-1-02-5}{Vektorraumbündel zu S1}

\end{EXA}

\begin{EXA}{EFT1-1-02-50}{Menge der Vektorfelder längs Kurven}

\end{EXA}

\begin{EXA}{EFT1-1-02-51}{Vektorraumbündel bzgl Grassmann-Mfk}

\end{EXA}

\begin{EXA}{EFT1-1-02-6}{Lokale triviale Faserung über S1}

\end{EXA}

\begin{EXA}{EFT1-1-02-64}{Charakterisierung von orientierten Mfk}

\end{EXA}


















































\subsection{Proposition}

\begin{PROP}{EFT1-1-02-28}{Für Präbündel (E,pi,M) existiert auf E genau einem Topologie und differenzierbare Struktur, sodass (E,pi,M) ein Faserbündel mit Strukturgruppe G wird und Präbündelkarten Bündelkarten werden}

\end{PROP}

\begin{PROP}{EFT1-1-02-49}{Schnitte in induzierten Bündeln längs f}

\end{PROP}

\begin{PROP}{EFT1-1-02-61}{Rang-Satz für Vektorraumhomomorphismen: Konstanter Rang impliziert ker und im sind Untervektorraumbündel}

\end{PROP}

\begin{PROP}{EFT1-1-02-65}{Ehresmannscher Faserungssatz: Totalräume mit eigentlich regulären Abbildungen in zusammenhängenden Basisraum implizieren eine lokale triviale Faserung}

\end{PROP}












\subsection{Corollar}

\begin{KORO}{EFT1-1-02-32}{Direkte Summe von Vektorraumbündeln ergeben Prävektorraumbündel}

\end{KORO}

\begin{KORO}{EFT1-1-02-53}{Homotope Abbildungen in Faserbündel induzieren isomorphe Bündel}

\end{KORO}

\begin{KORO}{EFT1-1-02-62}{Charakterisierung von Vektorraumbündeln von endlichem Typ}

\end{KORO}










\subsection{Remark}

\begin{REM}{EFT1-1-02-15}{Beziehung zwischen Vektorraumbündeln und GL-Faserbündeln}

\end{REM}

\begin{REM}{EFT1-1-02-23}{Raum der diff Schnitte in Vektorraumbündeln ist der Vektorraum von glatten Abbildungen auf M}

\end{REM}

\begin{REM}{EFT1-1-02-24}{Für Bündelkarten in Vektorraumbündeln existieren k lokale Schnitte, die an jeder Stelle eine Basis der realen Faser bilden}

\end{REM}

\begin{REM}{EFT1-1-02-25}{k lokale Schnitte, die bei Punkt eine Basis der Faser bilden, induzieren eine Bündelkarte}

\end{REM}

\begin{REM}{EFT1-1-02-26}{Bündelkarten in G-Prinzipalbündeln induzieren lokale Schnitte}

\end{REM}

\begin{REM}{EFT1-1-02-27}{Präbündel mit Strukturgruppe G zu Liegruppe G, Mfk, (disj) Vereinigung von punktweise Mfk und Projektion}

\end{REM}

\begin{REM}{EFT1-1-02-52}{Bündelabbildungen bzgl induzierte Bündel}

\end{REM}

\begin{REM}{EFT1-1-02-56}{Untervektorraumbündel sind Vektorraumbündel}

\end{REM}

\begin{REM}{EFT1-1-02-57}{Quotienten-Räume bzgl Untervektorraumbündel sind Vektorraumbündel}

\end{REM}

\begin{REM}{EFT1-1-02-58}{Untervektorraumbündel bzgl Bündelmetrik}

\end{REM}

\begin{REM}{EFT1-1-02-59}{Tangentialbündel von UnterMfk sind Untervektorraumbündel}

\end{REM}

\begin{REM}{EFT1-1-02-60}{Normalenbündel von UnterMfk}

\end{REM}






























\pagebreak


\printbibliography
\end{document}