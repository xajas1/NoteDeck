\documentclass[10pt, letterpaper]{article}

% Inhaltsverzeichnis für Pakettypen (nur für Übersicht im Header, wird nicht im Dokument angezeigt)
% 1. Seitenlayout und Ränder
% 2. Sprache und Zeichensatz
% 3. Mathematik und Theorem-Umgebungen
% 4. Eigene Makros
% 5. Diagramme und Grafiken
% 6. Tabellen und Aufzählungen
% 7. Inhaltsverzeichnis
% 8. Abschnittsüberschriften
% 9. Abstrakt-Umgebung
% 10. Todos/Notizen
% 11. Rahmen/Box-Umgebungen
% 12. Python-Integration
% 13. Literaturverwaltung
% 14. Hyperlinks
% 15. Absatzeinstellungen
% 16. Umgebungen
% 17. Titel und Autor

% --- 1. Seitenlayout und Ränder ---
\usepackage[margin=3cm]{geometry}

% --- 2. Sprache und Zeichensatz ---
\usepackage[english]{babel}
\usepackage[T1]{fontenc}
\usepackage[utf8]{inputenc}

% --- 3. Mathematik und Theorem-Umgebungen ---
\usepackage{amsmath, amssymb, amsthm}
\usepackage{mathrsfs}
\DeclareMathOperator{\WF}{WF}

% --- 4. Eigene Makros ---
\usepackage{xcolor}
\newcommand{\SKP}{\langle\cdot,\cdot\rangle}
\newcommand{\R}{\mathbb{R}}
\newcommand{\N}{\mathbb{N}}
\newcommand{\Q}{\mathbb{Q}}
\newcommand{\Z}{\mathbb{Z}}
\newcommand{\C}{\mathbb{C}}
\newcommand{\entwurf}[1]{\textcolor{red}{#1}}

% --- 5. Diagramme und Grafiken ---
\usepackage{graphicx}
\usepackage{tikz}
\usetikzlibrary{decorations.pathreplacing, arrows.meta, positioning}
\usepackage{tikz-cd}

% --- 6. Tabellen und Aufzählungen ---
\usepackage{enumitem}
\setlist[itemize]{left=0.5cm}

\newenvironment{romanenum}[1][]
  {%
    \ifx&#1&
    \else
      \textbf{#1}\quad
    \fi
    \begin{enumerate}[label=\roman*)]
  }
  {%
    \end{enumerate}%
  }

% --- 7. Inhaltsverzeichnis ---
\usepackage{tocloft}
\renewcommand{\cftsecfont}{\footnotesize}
\renewcommand{\cftsubsecfont}{\footnotesize}
\renewcommand{\cftsubsubsecfont}{\footnotesize}
\renewcommand{\cftsecpagefont}{\footnotesize}
\renewcommand{\cftsubsecpagefont}{\footnotesize}
\renewcommand{\cftsubsubsecpagefont}{\footnotesize}
\usepackage{etoc}

% --- 8. Abschnittsüberschriften ---
\usepackage{titlesec}
\titleformat{\section}{\normalfont\large\bfseries}{\thesection}{1em}{}
\titleformat{\subsection}[runin]{\normalfont\normalsize\bfseries}{\thesubsection}{0.5em}{}[:]
\titleformat{\subsubsection}[runin]{\normalfont\normalsize\bfseries}{\thesubsubsection}{0.5em}{}[:]
\setcounter{secnumdepth}{4}

% --- 9. Abstrakt-Umgebung ---
\usepackage{changepage}
\renewenvironment{abstract}
  {
    \begin{adjustwidth}{1.5cm}{1.5cm}
    \small
    \textsc{Abstract. –}%
  }
  {
    \end{adjustwidth}
  }

% --- 10. Todos/Notizen ---
\usepackage{todonotes}

% --- 11. Rahmen/Box-Umgebungen ---
\usepackage{mdframed}
\usepackage{tcolorbox}
\colorlet{shadecolor}{gray!25}

\newenvironment{customTheorem}
  {\vspace{10pt}%
   \begin{mdframed}[
     backgroundcolor=gray!20,
     linewidth=0pt,
     innertopmargin=10pt,
     innerbottommargin=10pt,
     skipabove=\dimexpr\topsep+\ht\strutbox\relax,
     skipbelow=\topsep,
   ]}
  {\end{mdframed}
   \vspace{10pt}%
  }

% --- 12. Python-Integration ---
% (Deaktiviert in dieser Version, aktiviere bei Bedarf)
% \usepackage{pythontex}
% \usepackage[makestderr]{pythontex}

% --- 13. Literaturverwaltung ---
\usepackage{csquotes}
\usepackage[backend=biber, style=alphabetic, citestyle=alphabetic]{biblatex}
\addbibresource{bibliography.bib}

% --- 14. Hyperlinks ---
\usepackage{hyperref}
\hypersetup{
  colorlinks   = true,
  urlcolor     = blue,
  linkcolor    = blue,
  citecolor    = blue,
  frenchlinks  = true
}

% --- 15. Absatzeinstellungen ---
\usepackage[parfill]{parskip}
\sloppy

% --- 16. Umgebungen ---
\usepackage{thmtools}

\newcommand{\CustomHeading}[3]{%
  \par\medskip\noindent%
  \textbf{#1 #2} \textnormal{(#3)}.\enskip%
}

\newenvironment{DEF}[2]{\CustomHeading{Definition}{#1}{#2}}{}
\newenvironment{PROP}[2]{\CustomHeading{Proposition}{#1}{#2}}{}
\newenvironment{THEO}[2]{\CustomHeading{Theorem}{#1}{#2}}{}
\newenvironment{LEM}[2]{\CustomHeading{Lemma}{#1}{#2}}{}
\newenvironment{KORO}[2]{\CustomHeading{Corollar}{#1}{#2}}{}
\newenvironment{REM}[2]{\CustomHeading{Remark}{#1}{#2}}{}
\newenvironment{EXA}[2]{\CustomHeading{Example}{#1}{#2}}{}
\newenvironment{STUD}[2]{\CustomHeading{Study}{#1}{#2}}{}
\newenvironment{CONC}[2]{\CustomHeading{Concept}{#1}{#2}}{}

\newenvironment{PROOF}
  {\begin{proof}}%
{\end{proof}}

% --- 17. Titel und Autor ---
\title{Eichfeldtheorie 1}
\author{Tim Jaschik}
\date{\today}

\begin{document}

\maketitle
\rule{\textwidth}{0.5pt}
\begin{abstract}
…
\end{abstract}
\rule{\textwidth}{0.5pt}
\vspace{0.5cm}

\tableofcontents


\pagebreak


\section{Faserbündel}

\subsection{Definitionen}

\begin{DEF}{2.1}{Lokale Trivialisierung mit typischen Fasern auf Mfk}
    Seien $E, M$ und $F$ differenzierbare Mannigfaltikeiten und $\pi: E \rightarrow M$ eine differenzierbare Abbildung. Dann heißt $(E, \pi, M)$ eine lokal triviale Faserung mit typischer Faser $F$, wenn es zu jedem $x \in M$ eine offene Umgebung $U$ gibt und einen Diffeomorphismus $\varphi: \pi^{-1}(U):=E \mid U \rightarrow$ $U \times F$, sodass

    $$
    \begin{array}{lll}
    E \mid U & \xrightarrow{\varphi} & U \times F \\
    \pi \searrow & & \swarrow p r_1 \\
    & U &
    \end{array}
    $$
    
    kommutiert. Man spricht auch von der lokal trivialen Faserung $E \rightarrow M$ oder $E$.
\end{DEF}


\begin{DEF}{2.2}{Vektorraumbündel}
    Sei $(E,\pi,M)$ eine lokale triviale Faserung mit typischer Faser $E$. Ist $F=\mathbb{R}^k$ und ist $\pi^{-1}(x)$ ein $k$-dimensionaler Vektorraum und 
    $$p r_2 \circ \left.\varphi\right|_{\pi^{-1}(x)}: \pi^{-1}(x) \rightarrow \mathbb{R}^k$$ 
    ein Isomorphismus, so heißt $E$ ein Vektorraumbündel der Dimension $k$.
\end{DEF}


\begin{DEF}{2.7}{Lokale Triviale Faser als Tripel (Totalraum,Basisraum,Projektion)}
    Sei $(E, \pi, M)$ eine lokal triviale Faserung wie in 1.1. Dann heißt $E T o$ talraum, $M$ Basis, $\pi$ Bündelprojektion und $F$ typische Faser.
\end{DEF}


\begin{DEF}{2.8}{Reale Fasern in lokal trivialen Bündeln}
    Sei $(E, \pi, M)$ eine lokal triviale Faserung. Für jedes $x \in M$ heißt $E_x=\pi^{-1}(x)$ reale Faser an der Stelle $x$.
    Für $U \subset M$ offen heißt $\varphi: E \mid U \rightarrow U \times F$ Bündelkarte und
    
    $$
    \left\{\left(U_\lambda, \varphi_\lambda\right) \mid\left(U_\lambda, \varphi_\lambda\right) \text { Bündelkarte }, \bigcup_{\lambda \in \Lambda} U_\lambda=M\right\}
    $$
    
    heißt Bündelatlas.
    Die Abbildung $\varphi_x: E_x \rightarrow F, \varphi_x:=p r_2 \circ \varphi \mid E_x$ heißt Faserkarte. Sind $(U, \varphi)$ und $(V, \psi)$ Bündelkarten, so heißt die Abbildung
    
    $$
    \omega: U \cap V \rightarrow \operatorname{Diffeo}(F), x \mapsto \psi_x \circ \varphi_x^{-1}
    $$
    
    der Bündelkartenwechsel zwischen $\varphi$ und $\psi$.
\end{DEF}


\begin{DEF}{2.9}{Bündelkarten für offene Teilmengen der Basis und Bündelatlas}
    Sei $(E, \pi, M)$ eine lokal triviale Faserung. 
    Für $U \subset M$ offen heißt $\varphi: E \mid U \rightarrow U \times F$ Bündelkarte und
    
    $$
    \left\{\left(U_\lambda, \varphi_\lambda\right) \mid\left(U_\lambda, \varphi_\lambda\right) \text { Bündelkarte }, \bigcup_{\lambda \in \Lambda} U_\lambda=M\right\}
    $$

    heißt Bündelatlas.
\end{DEF}


\begin{DEF}{2.11}{Faserkarte am Punkt $x$ im Basisraum}
    Sind $(U, \varphi)$ und $(V, \psi)$ Bündelkarten, so heißt die Abbildung

    $$
    \omega: U \cap V \rightarrow \operatorname{Diffeo}(F), x \mapsto \psi_x \circ \varphi_x^{-1}
    $$
    
    der Bündelkartenwechsel zwischen $\varphi$ und $\psi$.
\end{DEF}


\begin{DEF}{2.12}{Bündelkartenwechsel zwischen Bündelkarten}

\end{DEF}


\begin{DEF}{2.13}{G-Faserbündel mit Liegruppen als Strukturgruppen}

\end{DEF}


\begin{DEF}{2.14}{Prinzipalbündel}

\end{DEF}


\begin{DEF}{2.16}{(Lokale) Schnitte in lokal trivialen Faserungen}

\end{DEF}


\begin{DEF}{2.17}{Raum der differenzierbaren lokalen Schnitte}

\end{DEF}


\begin{DEF}{2.37}{Bündelmetrik auf Totalraum}

\end{DEF}


\begin{DEF}{2.40}{Vektorraumbündel von endlichem Typ}

\end{DEF}


\begin{DEF}{2.43}{Bündelisomorphismus}

\end{DEF}


\begin{DEF}{2.44}{Trivialisierung von Totalraum}

\end{DEF}


\begin{DEF}{2.45}{Vektorraumbündelabbildung über diff. Abbildungen zw. Vektorraumbündeln}

\end{DEF}


\begin{DEF}{2.46}{Vektorraumbündelisomorphismus}

\end{DEF}


\begin{DEF}{2.48}{Induzierte Bündel durch Abbildungen}

\end{DEF}


\begin{DEF}{2.54}{Induzierte Bündel bei Einbettungen von UMfk}

\end{DEF}


\begin{DEF}{2.55}{Untervektorraumbündel}

\end{DEF}


\begin{DEF}{2.63}{Reduktionen von Faserbündeln mit Strukturgruppe bzgl. abgeschlossener Untergruppe}

\end{DEF}



\subsection{Propositions}






\pagebreak


\printbibliography
\end{document}